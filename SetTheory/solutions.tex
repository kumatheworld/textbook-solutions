\documentclass[12pt]{article}
\usepackage[margin=1in]{geometry}
\usepackage{mathtools}
\usepackage{amsthm}
\usepackage{amsfonts}
\DeclarePairedDelimiter\ceil{\lceil}{\rceil}
\DeclarePairedDelimiter\floor{\lfloor}{\rfloor}
\newcommand{\dom}{\mathop{\mathrm{dom}}}
\newcommand{\ran}{\mathop{\mathrm{ran}}}
\newcommand{\type}{\mathop{\mathrm{type}}}
\newcommand{\trcl}{\mathop{\mathrm{trcl}}}
\newcommand{\rank}{\mathop{\mathrm{rank}}}
\newcommand{\val}{\mathop{\mathrm{val}}}

\theoremstyle{definition}
\newtheorem{innercustomthm}{Exercise}
\newenvironment{customthm}[1]
  {\renewcommand\theinnercustomthm{#1}\innercustomthm}
  {\endinnercustomthm}

\title{
  \vspace{-2cm}
  Set Theory \\
  \large by Kenneth Kunen \\
  Exercise Solutions
  \author{Yoshihiro Kumazawa}
}

\begin{document}
\maketitle

\begin{customthm}{I.4.18}
  Assume $y$ is a set such that $y\in y$ and let $x=\{y\}$. This $x$ can be expressed as $x=\{z\in y:z=y\}$ and hence it exists by Comprehension. Now we apply Foundation to $x$, which states that there is no $z$ such that $z=y\wedge z\in y$. That contradicts the assumption $y\in y$. Thus $y\not\in y$ holds for any set $y$.
\end{customthm}

\begin{customthm}{I.6.23}
  Let $X$ be a non-empty subset of $(\mathbb{N}\times\mathbb{N},R)$. Let $x=\min(\dom(X))$ and $y=\min\{v\in\mathbb{N}:(x,v)\in X\}$. Then $(x,y)$ is an $R$-minimal element. $X$ has at most $y$ minimal elements other than $(x,y)$ because $X_v=\{(u,v)\in X\}$ for each $v<y$ has at most 1 minimal element and $(u,v)\in X$ such that $y\leq v$ and $(u,v)\neq(x,y)$ cannot be minimal since $(x,y)<(u,v)$ holds. Hence $X$ has only finitely many $R$-minimal elements. For each $k\in\mathbb{N}$, we have a subset $\{(x,y)\in\mathbb{N}\times\mathbb{N}:x+y+1=k\}$ of size $k$, every element of which is an $R$-minimal element.
\end{customthm}

\begin{customthm}{I.6.26}
  $\in$ totally orders $3=\{0,1,2\}$ because $0\in 1=\{0\}$, $0\in 2=\{0,1\}$ and $1\in 2=\{0,1\}$ hold. Moreover, it is strict because none of $0\in 0$, $1\in 1$ or $2\in 2$ holds. Indeed, $0\not\in 0=\emptyset$ is obvious. $1\in 1$ would imply $1=0$, but we know $0\in 1\setminus0$. $2\in 2$ would imply either $2=0$ or $2=1$, both of which case are contradictory. Now it is clear that $\in$ well-orders $3$ because it is a strict total order and $3$ is finite.
\end{customthm}

\end{document}