\documentclass[12pt]{article}
\usepackage[margin=1in]{geometry}
\usepackage{mathtools}
\usepackage{amsthm}
\usepackage{amsfonts}
\usepackage{amssymb}
\DeclarePairedDelimiter\ceil{\lceil}{\rceil}
\DeclarePairedDelimiter\floor{\lfloor}{\rfloor}
\newcommand{\dom}{\mathop{\mathrm{dom}}}
\newcommand{\ran}{\mathop{\mathrm{ran}}}
\newcommand{\type}{\mathop{\mathrm{type}}}
\newcommand{\trcl}{\mathop{\mathrm{trcl}}}
\newcommand{\rank}{\mathop{\mathrm{rank}}}
\newcommand{\val}{\mathop{\mathrm{val}}}

\theoremstyle{definition}
\newtheorem{innercustomthm}{Exercise}
\newenvironment{customthm}[1]
  {\renewcommand\theinnercustomthm{#1}\innercustomthm}
  {\endinnercustomthm}

\title{
  \vspace{-2cm}
  Set Theory \\
  \large by Kenneth Kunen \\
  Exercise Solutions
  \author{Yoshihiro Kumazawa}
}

\begin{document}
\maketitle

\begin{customthm}{I.4.18}
  Assume $y$ is a set such that $y\in y$ and let $x=\{y\}$. This $x$ can be expressed as $x=\{z\in y:z=y\}$ and hence it exists by Comprehension. Now we apply Foundation to $x$, which states that there is no $z$ such that $z=y\wedge z\in y$. That contradicts the assumption $y\in y$. Thus $y\not\in y$ holds for any set $y$.
\end{customthm}

\begin{customthm}{I.6.23}
  Let $X$ be a non-empty subset of $(\mathbb{N}\times\mathbb{N},R)$. Let $x=\min(\dom(X))$ and $y=\min\{v\in\mathbb{N}:(x,v)\in X\}$. Then $(x,y)$ is an $R$-minimal element. $X$ has at most $y$ minimal elements other than $(x,y)$ because $X_v=\{(u,v)\in X\}$ for each $v<y$ has at most 1 minimal element and $(u,v)\in X$ such that $y\leq v$ and $(u,v)\neq(x,y)$ cannot be minimal since $(x,y)<(u,v)$ holds. Hence $X$ has only finitely many $R$-minimal elements. For each $k\in\mathbb{N}$, we have a subset $\{(x,y)\in\mathbb{N}\times\mathbb{N}:x+y+1=k\}$ of size $k$, every element of which is an $R$-minimal element.
\end{customthm}

\begin{customthm}{I.6.26}
  $\in$ totally orders $3=\{0,1,2\}$ because $0\in 1=\{0\}$, $0\in 2=\{0,1\}$ and $1\in 2=\{0,1\}$ hold. Moreover, it is strict because none of $0\in 0$, $1\in 1$ or $2\in 2$ holds. Indeed, $0\not\in 0=\emptyset$ is obvious. $1\in 1$ would imply $1=0$, but we know $0\in 1\setminus0$. $2\in 2$ would imply either $2=0$ or $2=1$, both of which case are contradictory. Now it is clear that $\in$ well-orders $3$ because it is a strict total order and $3$ is finite.
\end{customthm}

\begin{customthm}{I.6.29}
  Assume there are distinct sets $x,y$ such that $S(x)=S(y)$. Since $x\in S(x)$, we get $x\in y$ from $x\in S(y)$ and $x\neq y$. We get $y\in x$ as well, from which we conclude that the set $\{x,y\}$ has no $\in$-minimal element. Thus $S$ is injective on $V$.
\end{customthm}

\begin{customthm}{I.7.2}
  Let $t$ be a non-empty transitive set. By Foundation, there is $x\in t$ with $x\cap t=\emptyset$. Since $t$ is transitive, $x\subseteq t$ holds and we get $x=\emptyset$ from $x\cap t=\emptyset$. Now assume $t$ has at least 1 element other than $\emptyset$. By applying Foundation to $t\setminus\{\emptyset\}$, we get $y\in t$ such that $y\neq\emptyset\wedge y\cap(t\setminus\{\emptyset\})=\emptyset$. From $y\subseteq t$ (transitivity) and $y\cap(t\setminus\{\emptyset\})=\emptyset$ we get $y\subseteq\{\emptyset\}$, which means nothing but $y=\{\emptyset\}$ since $y$ is non-empty. Now the statements in the exercise are clear.
\end{customthm}

\begin{customthm}{I.7.25}
  Any ordinal is transitive and satisfies trichotomy by definition. Conversely, under Foundation, we prove that any transitive set whose $\in$ satisfies trichotomy also satisfies irreflexivity and transitivity. Irreflexivity is obvious from Exercise I.4.18. To prove transitivity, let $z$ be a transitive set, $w,x,y\in z$ and $w\in x\in y$. If $(z,\in)$ satisfies trichotomy, either of $w\in y$, $w=y$ or $y\in w$ holds. If $w=y$, then $y\in x\in y$ would hold, producing a set $\{x,y\}$ which does not have an $\in$-minimal element, contradicting Foundation. $y\in w$ would produce a contradictory set $\{w,x,y\}$ as well. Thus $w\in y$ holds, which proves transitivity. Hence $z$ is totally ordered by $\in$ and turns out to be an ordinal by Lemma I.7.2.
\end{customthm}

\begin{customthm}{I.7.26}
  Any ordinal is a transitive set by definition and all its elements are transitive because $\in$ on it is a transitive relation by definition. We now assume Foundation to prove the converse. Let $z$ be a transitive set whose elements are all transitive sets and let $\alpha$ be the least ordinal such that $\alpha\not\in z$. Note that such $\alpha$ exists by Theorem I.7.12 and Theorem I.9.3. Since any $\beta\in\alpha$ is an ordinal less than $\alpha$, we have $\beta\in z$ and thus $\alpha\subseteq z$ holds. Assume $\alpha\subsetneq z$ and let $y$ be an $\in$-minimal element of $z\setminus\alpha$. Let $x\in y$. Since $z$ is a transitive set, we have $x\in z$. On the other hand, $x\not\in z\setminus\alpha$ holds since otherwise it would contradict the minimality of $y$. Thus we get $x\in\alpha$, which proves $y\subseteq\alpha$. Now from Theorem I.7.13, $y$ turns out to be an ordinal since it is a transitive set of ordinals. Then either of $y\in\alpha$ or $y=\alpha$ holds from $y\subseteq\alpha$, the first one of which is not the case due to $y\in z\setminus\alpha$. Therefore $y=\alpha$ holds and we get $\alpha\in z$, but that contradicts the definition of $\alpha$. Thus our assumption $\alpha\subsetneq z$ is not correct and hence we conclude $\alpha=z$, which proves that $z$ is an ordinal.
\end{customthm}

\begin{customthm}{I.7.27}
  Skipped for now.
\end{customthm}

\begin{customthm}{I.9.6}
  Skipped for now.
\end{customthm}

\begin{customthm}{I.9.8}
  If $R$ is cyclic on $A$, let $a_0,a_1,\ldots,a_{n-1},a_0$ be a cycle and consider $S=\{a_i:i<n\}$ . Since $S$ is a non-empty subset of $A$ which does not have an $R$-minimal element, $R$ is not well-founded on $A$. Hence any well-founded relation is acyclic. Now we consider the converse when the underlying set is finite. If $R$ is not well-founded on a finite set $A$, there is a non-empty $S\subseteq A$ which does not have an $R$-minimal element. Then $S$ has a cycle because otherwise we would have an infinite sequence of distinct elements in $S$, which cannot happen since $A$ is finite. Thus any acyclic relation on a finite set is well-founded on that set.
\end{customthm}

\end{document}