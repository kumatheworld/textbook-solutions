\documentclass[12pt]{article}
\usepackage[margin=1in]{geometry}
\usepackage{mathtools}
\usepackage{amsthm}
\DeclarePairedDelimiter\ceil{\lceil}{\rceil}
\DeclarePairedDelimiter\floor{\lfloor}{\rfloor}
\newcommand{\dom}{\mathop{\mathrm{dom}}}
\newcommand{\ran}{\mathop{\mathrm{ran}}}
\newcommand{\type}{\mathop{\mathrm{type}}}
\newcommand{\trcl}{\mathop{\mathrm{trcl}}}
\newcommand{\rank}{\mathop{\mathrm{rank}}}
\newcommand{\val}{\mathop{\mathrm{val}}}

\theoremstyle{definition}
\newtheorem{innercustomthm}{Exercise}
\newenvironment{customthm}[1]
  {\renewcommand\theinnercustomthm{#1}\innercustomthm}
  {\endinnercustomthm}

\title{
  \vspace{-2cm}
  Set Theory \\
  \large by Kenneth Kunen \\
  Exercise Solutions
  \author{Yoshihiro Kumazawa}
}

\begin{document}
\maketitle

\begin{customthm}{I.4.18}
  Assume $y$ is a set such that $y\in y$ and let $x=\{y\}$. This $x$ can be expressed as $x=\{z\in y:z=y\}$ and hence it exists by Comprehension. Now we apply Foundation to $x$, which states that there is no $z$ such that $z=y\wedge z\in y$. That contradicts the assumption $y\in y$. Thus $y\not\in y$ holds for any set $y$.
\end{customthm}

\end{document}