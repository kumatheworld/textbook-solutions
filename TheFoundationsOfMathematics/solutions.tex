\documentclass[12pt]{article}
\usepackage{amsthm}
\usepackage{multirow}

\theoremstyle{definition}
\newtheorem{innercustomthm}{Exercise}
\newenvironment{customthm}[1]
  {\renewcommand\theinnercustomthm{#1}\innercustomthm}
  {\endinnercustomthm}

\begin{document}
\title{
  The Foundations of Mathematics \\
  \large by Kenneth Kunen \\
  Exercise Solutions
  \author{Yoshihiro Kumazawa}
}
\maketitle

\begin{customthm}{I.2.1} Below is the table of truth values.
  \begin{center}
    \begin{tabular}{cc|cccc}
      & & \multicolumn{4}{c}{Axiom} \\
      & & 1 & 2 & 4 & 5 \\ \hline
      \multirow{7}{*}{Example}
      & 1 & T & T & F & T \\
      & 2 & T & F & T & T \\
      & 3 & T & T & F & F \\
      & 4 & T & T & F & F \\
      & 5 & F & T & F & T \\
      & 6 & T & T & F & F \\
      & 7 & T & T & F & F \\
    \end{tabular}
  \end{center}
\end{customthm}

\begin{customthm}{I.6.3} Example 2, 3 and 4 are the ones. Those are the ones that satisfy $\neg[\exists x~\mathrm{emp}(x)]$, which also satisfy Extensionality.
\end{customthm}

\begin{customthm}{I.6.11} Example 7 is the one. Example 1, 5 and 6 have pairwise unions. Example 2, 3 and 4 do not have empty sets hence cannot satisfy Comprehension.
\end{customthm}

\begin{customthm}{I.6.13} Example 1 is the one. The other ones do not satisfy $\forall x\neg\exists y[y\in x]$.
\end{customthm}

\begin{customthm}{I.6.15} Assume $\langle x',y'\rangle = \langle x,y\rangle$. Consider the following cases.
  \begin{enumerate}
  \item \underline{$x=y$}. Then it holds that $\langle x,y\rangle = \{\{x\}, \{x,y\}\} = \{\{x\}, \{x,x\}\} = \{\{x\}, \{x\}\} = \{\{x\}\}$. So $\{x',y'\}\in\langle x',y'\rangle$ has to be  identical to $\{x\}$ and we get $x'=y'=x=y$.
  \item \underline{$x\neq y$}. Then $\{x,y\}$ is a doubleton and it has to be identical to $\{x',y'\}$. Since $\{x',y'\}$ is a doubleton, the 2 singletons $\{x\}$ and $\{x'\}$ have to be identical, which leads to $x=x'$. Now we have $\{x,y\} = \{x,y'\}$. Considering $y\in \{x,y'\}$ and $y\neq x$, we get $y=y'$.
  \end{enumerate}
\end{customthm}

\begin{customthm}{I.6.17} $\langle 0,1\rangle = \{\{0\}, \{0,1\}\} = \{1,2\}$. $\langle 1,0\rangle = \{\{1\}, \{1,0\}\} = \{\{1\}, 2\}$.
\end{customthm}

\begin{customthm}{I.7.13} Let $<$ and $\prec$ be are strict total orderings of $S,T$ respectively and let $\triangleleft$ be the lexicographic product on $S\times T$. Check the following properties of $\triangleleft$ one by one.
  \begin{enumerate}
  \item \underline{Transitivity}. Let $\langle s,t\rangle, \langle s',t'\rangle, \langle s'',t''\rangle \in S\times T$ and assume $\langle s,t\rangle\triangleleft\langle s',t'\rangle$ and $\langle s',t'\rangle\triangleleft\langle s'',t''\rangle$. If $s<s'$ and $s'<s''$, then $s<s''$ holds from transitivity of $<$. If $s<s'$ and $s'=s''$, or, $s=s'$ and $s'<s''$, then it is obvious that $s<s''$.
  If $s=s'\wedge t\prec t'$ and $s'=s''\wedge t'\prec t''$, then $s=s''$ and $t\prec t''$ holds from transitivity of $=$ and $\prec$. Those exhaust all the possible cases, in each of which $\langle s,t\rangle\triangleleft\langle s'',t''\rangle$ holds.
  \item \underline{Irreflexivity}. Since $<$ and $\prec$ are irreflexive, we get $\langle s,t\rangle\triangleleft\langle s,t\rangle \leftrightarrow [s<s\vee[s=s\wedge t\prec t]] \leftrightarrow [\mathrm{F} \vee[\mathrm{T}\wedge \mathrm{F}]] \leftrightarrow \mathrm{F}$.
  \item \underline{Trichotomy}. Assume $\langle s,t\rangle\not\triangleleft\langle s',t'\rangle$. That is, $\neg[s<s'\vee[s=s'\wedge t\prec t']] \leftrightarrow [s'\leq s\wedge[s\neq s'\vee t'\preceq t]] \leftrightarrow [s'\leq s\wedge[s\neq s'\vee [s=s'\wedge t'\preceq t]]] \leftrightarrow [s'<s\vee[s=s'\wedge t'\preceq t]]$.
  Here trichotomies of $<$ and $\prec$ are used. Let us further assume $\langle s,t\rangle\neq\langle s',t'\rangle$, which is equivalent to $s=s'\rightarrow t\neq t'$. Combining this with the formula of $\langle s,t\rangle\not\triangleleft\langle s',t'\rangle$, we get $[s'<s\vee[s=s'\wedge t'\prec t]]$. This is nothing but $\langle s',t'\rangle\triangleleft\langle s,t\rangle$, which shows the trichotomy.
  \end{enumerate}
\end{customthm}

\begin{customthm}{I.7.15} Let $R$ be a set. Since  $[(x,y)]=[(x',y')]\rightarrow x=x'$, the formula $\phi(t,x)=\exists y[t=[(x, y)]$ has the property that $\forall t\in R\exists!y\phi(t,y)$. Then the set $\{x:\exists y[(x, y)\in R]\}$ exists by applying Replacement and Comprehension. The existence of the set $\{y:\exists x[(x, y)\in R]\}$ is proved in the same way.
\end{customthm}

\begin{customthm}{I.7.17} Let $(A,\triangleleft_1),(B,\triangleleft_2),(C,\triangleleft_3)$ be arbitrary relations and $F:A\rightarrow B,G:B\rightarrow C$ be isomorphisms. Reflexivity of $\cong$ is easily checked by letting $(A,\triangleleft_1)=(B,\triangleleft_2)$ and $F$ be the identity map $\mathrm{id}_A$. Symmetry follows from taking $F^{-1}:B\rightarrow A$. Transitivity is proved by considering $G\circ F:A\rightarrow C$.
Note that those $\mathrm{id}_A$, $F^{-1}$ and $G\circ F$ are all isomorphisms.
\end{customthm}

\begin{customthm}{I.7.21} Note that $R$ well-orders $A$ if and only if every non-empty subset of $A$ has the $R$-least element. If $R$ well-orders $A$, then it is obvious that $R$ well-orders every $X\subseteq A$ because every non-empty subset of $X$ is a non-empty subset of $A$ as well, which has the $R$-least element.
\end{customthm}

\begin{customthm}{I.7.23} Let $(S,<)$ and $(T,\prec)$ be well-ordered sets and $\triangleleft$ be their lexicographic product. Let $X$ be a non-empty subset of $S\times T$ and $Y = \mathrm{ran}(X)$. Then $Y$ is a non-empty subset of $S$ and it has the $<$-least element $s$. Let $Z=\{y\in T:(s,y)\in X\}$. It is a non-empty subset of $T$, which has the $\prec$-least element $t$. Now we have $(s,t)\in S\times T$ and it is the $\triangleleft$-least element of $X$.
\end{customthm}

\end{document}
