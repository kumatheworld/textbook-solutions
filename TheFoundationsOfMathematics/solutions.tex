\documentclass[12pt]{article}
\usepackage{amsthm}
\usepackage{amsmath}
\usepackage{amssymb}
\usepackage{amsfonts}
\usepackage{multirow}
\usepackage{url}
\usepackage{mathtools}
\DeclarePairedDelimiter\floor{\lfloor}{\rfloor}
\newcommand{\type}{\mathop{\mathrm{type}}}
\newcommand{\trcl}{\mathop{\mathrm{trcl}}}
\newcommand{\rank}{\mathop{\mathrm{rank}}}

\theoremstyle{definition}
\newtheorem{innercustomthm}{Exercise}
\newenvironment{customthm}[1]
  {\renewcommand\theinnercustomthm{#1}\innercustomthm}
  {\endinnercustomthm}

\begin{document}
\sloppy
\title{
  The Foundations of Mathematics \\
  \large by Kenneth Kunen \\
  Exercise Solutions
  \author{Yoshihiro Kumazawa}
}
\maketitle

\begin{customthm}{I.2.1} Below is the table of truth values.
  \begin{center}
    \begin{tabular}{cc|cccc}
      & & \multicolumn{4}{c}{Axiom} \\
      & & 1 & 2 & 4 & 5 \\ \hline
      \multirow{7}{*}{Example}
      & 1 & T & T & F & T \\
      & 2 & T & F & T & T \\
      & 3 & T & T & F & F \\
      & 4 & T & T & F & F \\
      & 5 & F & T & F & T \\
      & 6 & T & T & F & F \\
      & 7 & T & T & F & F \\
    \end{tabular}
  \end{center}
\end{customthm}

\begin{customthm}{I.6.3} Example 2, 3 and 4 are the ones. Those are the ones that satisfy $\neg[\exists x\,\mathrm{emp}(x)]$, which also satisfy Extensionality.
\end{customthm}

\begin{customthm}{I.6.11} Example 7 is the one. Example 1, 5 and 6 have pairwise unions. Example 2, 3 and 4 do not have empty sets hence cannot satisfy Comprehension.
\end{customthm}

\begin{customthm}{I.6.13} Example 1 is the one. The other ones do not satisfy $\forall x\,\neg\exists y\,[y\in x]$.
\end{customthm}

\begin{customthm}{I.6.15} Assume $\langle x',y'\rangle = \langle x,y\rangle$. Consider the following cases.
\begin{enumerate}
\item \underline{$x=y$}. Then it holds that $\langle x,y\rangle = \{\{x\}, \{x,y\}\} = \{\{x\}, \{x,x\}\} = \{\{x\}, \{x\}\} = \{\{x\}\}$. So $\{x',y'\}\in\langle x',y'\rangle$ has to be  identical to $\{x\}$ and we get $x'=y'=x=y$.
\item \underline{$x\neq y$}. Then $\{x,y\}$ is a doubleton and it has to be identical to $\{x',y'\}$. Since $\{x',y'\}$ is a doubleton, the 2 singletons $\{x\}$ and $\{x'\}$ have to be identical, which leads to $x=x'$. Now we have $\{x,y\} = \{x,y'\}$. Considering $y\in \{x,y'\}$ and $y\neq x$, we get $y=y'$.
\end{enumerate}
\end{customthm}

\begin{customthm}{I.6.17} $\langle 0,1\rangle = \{\{0\}, \{0,1\}\} = \{1,2\}$. $\langle 1,0\rangle = \{\{1\}, \{1,0\}\} = \{\{1\}, 2\}$.
\end{customthm}

\begin{customthm}{I.7.13} Let $<$ and $\prec$ be are strict total orderings of $S,T$ respectively and let $\triangleleft$ be the lexicographic product on $S\times T$. Check the following properties of $\triangleleft$ one by one.
\begin{enumerate}
\item\underline{Transitivity}. Let $\langle s,t\rangle, \langle s',t'\rangle, \langle s'',t''\rangle \in S\times T$ and assume $\langle s,t\rangle\triangleleft\langle s',t'\rangle$ and $\langle s',t'\rangle\triangleleft\langle s'',t''\rangle$. If $s<s'$ and $s'<s''$, then $s<s''$ holds from transitivity of $<$. If $s<s'$ and $s'=s''$, or, $s=s'$ and $s'<s''$, then it is obvious that $s<s''$.
If $s=s'\wedge t\prec t'$ and $s'=s''\wedge t'\prec t''$, then $s=s''$ and $t\prec t''$ holds from transitivity of $=$ and $\prec$. Those exhaust all the possible cases, in each of which $\langle s,t\rangle\triangleleft\langle s'',t''\rangle$ holds.
\item\underline{Irreflexivity}. Since $<$ and $\prec$ are irreflexive, we get $\langle s,t\rangle\triangleleft\langle s,t\rangle \leftrightarrow [s<s\vee[s=s\wedge t\prec t]] \leftrightarrow [\mathrm{F} \vee[\mathrm{T}\wedge \mathrm{F}]] \leftrightarrow \mathrm{F}$.
\item\underline{Trichotomy}. Assume $\langle s,t\rangle\not\triangleleft\langle s',t'\rangle$. That is, $\neg[s<s'\vee[s=s'\wedge t\prec t']] \leftrightarrow [s'\leq s\wedge[s\neq s'\vee t'\preccurlyeq t]] \leftrightarrow [s'\leq s\wedge[s\neq s'\vee [s=s'\wedge t'\preccurlyeq t]]] \leftrightarrow [s'<s\vee[s=s'\wedge t'\preccurlyeq t]]$.
Here trichotomies of $<$ and $\prec$ are used. Let us further assume $\langle s,t\rangle\neq\langle s',t'\rangle$, which is equivalent to $s=s'\rightarrow t\neq t'$. Combining this with the formula of $\langle s,t\rangle\not\triangleleft\langle s',t'\rangle$, we get $[s'<s\vee[s=s'\wedge t'\prec t]]$. This is nothing but $\langle s',t'\rangle\triangleleft\langle s,t\rangle$, which shows the trichotomy.
\end{enumerate}
\end{customthm}

\begin{customthm}{I.7.15} Let $R$ be a set. Since  $[(x,y)]=[(x',y')]\rightarrow x=x'$, the formula $\varphi(t,x)=\exists y\,[t=[(x, y)]$ has the property that $\forall t\in R\,\exists!y\,\varphi(t,y)$. Then the set $\{x:\exists y\,[(x, y)\in R]\}$ exists by applying Replacement and Comprehension. The existence of the set $\{y:\exists x\,[(x, y)\in R]\}$ is proved in the same way.
\end{customthm}

\begin{customthm}{I.7.17} Let $(A,\triangleleft_1),(B,\triangleleft_2),(C,\triangleleft_3)$ be arbitrary relations and $F:A\rightarrow B,G:B\rightarrow C$ be isomorphisms. Reflexivity of $\cong$ is easily checked by letting $(A,\triangleleft_1)=(B,\triangleleft_2)$ and $F$ be the identity map $\mathrm{id}_A$. Symmetry follows from taking $F^{-1}:B\rightarrow A$. Transitivity is proved by considering $G\circ F:A\rightarrow C$.
Note that those $\mathrm{id}_A$, $F^{-1}$ and $G\circ F$ are all isomorphisms.
\end{customthm}

\begin{customthm}{I.7.21} Note that $R$ well-orders $A$ if and only if every non-empty subset of $A$ has a $R$-least element. If $R$ well-orders $A$, then it is obvious that $R$ well-orders every $X\subseteq A$ because every non-empty subset of $X$ is a non-empty subset of $A$ as well, which has a $R$-least element.
\end{customthm}

\begin{customthm}{I.7.23} Let $(S,<)$ and $(T,\prec)$ be well-ordered sets and $\triangleleft$ be their lexicographic product. Let $X$ be a non-empty subset of $S\times T$ and $Y = \mathrm{ran}(X)$. Then $Y$ is a non-empty subset of $S$ and it has a $<$-least element $s$. Let $Z=\{y\in T:(s,y)\in X\}$. It is a non-empty subset of $T$, which has a $\prec$-least element $t$. Now we have $(s,t)\in S\times T$ and it is the $\triangleleft$-least element of $X$.
\end{customthm}

\begin{customthm}{I.8.10} Let $X$ be a non-empty set of ordinals. $\bigcap X$ is well-ordered since it is a set of ordinals and $ON$ is well-ordered by $\in$. Let $\alpha\in\bigcap X$. Then $\alpha\subseteq\beta$ holds for every $\beta\in X$ by transitivity of $\beta$ and hence $\alpha\subseteq\bigcap X$ holds, which proves that $\bigcap X$ is a transitive set. Thus $\bigcap X$ is an ordinal.
Now let $\alpha=\min(X)$. $\alpha\subseteq\bigcap X$ follows from the fact that $\alpha\subseteq\beta(\leftrightarrow\alpha\leq\beta)$ for any $\beta\in X$. $\bigcap X\subseteq\alpha$ is obvious since every $\beta\in\bigcap X$ is also a member of $\alpha\in X$. Thus $\bigcap X=\min(X)$ holds. The fact that $\bigcup X$ is an ordinal with $\bigcup X=\sup(X)$ is proved in the same manner.
\end{customthm}

\begin{customthm}{I.8.11} Let $\alpha\in ON$. Every $\beta\in S(\alpha)=\alpha\cup\{\alpha\}$ is either in $\alpha$ or equal to $\alpha$. If $\beta\in\alpha$, $\beta\subseteq\alpha$ follows from transitivity of $\alpha$. If $\beta=\alpha$, then $\beta\subseteq \alpha$ is obvious.
Thus $\beta\subseteq\alpha\subseteq\alpha\cup\{\alpha\}=S(\alpha)$ holds in both cases, which proves $S(\alpha)$'s transitivity. To prove that $S(\alpha)$ is well-ordered, take a non-empty subset $X$ of $S(\alpha)$. If $X=\{\alpha\}$, then $\alpha$ is obviously the $\in$-least element of $X$. If not, $X\setminus\{\alpha\}$ is a non-empty subset of $\alpha$ and since $\alpha$ is well-ordered, $X\setminus\{\alpha\}$ has a $\in$-least element, which is also the least element of $X$.
Hence $S(\alpha)$ is well-ordered and $S(\alpha)\in ON$. The rest is immediate when we replace "$<$" by "$\in$" and "$\leq$" by "$\in$" or "$=$".
\end{customthm}

\begin{customthm}{I.8.13} Let $n$ be a natural number and $\beta\leq S(n)$. Then $\beta$ is either $\beta=S(n)$ or $\beta<S(n)\leftrightarrow\beta\leq n$. If $\beta=S(n)$, then $\beta$ is a successor of $n$. If $\beta\leq n$, then $\beta$ is either 0 or a successor since $n$ is a natural number. In either case, $\beta$ is either 0 or a successor and hence $S(n)$ is a natural number.
Now let $\gamma\in n$. Then for every $\alpha\leq\gamma$, $\alpha$ is either 0 or a successor since $n$ is a natural number and $\alpha\leq\gamma\leq n$. Therefore every element of $n$ turns out to be a natural number.
\end{customthm}

\begin{customthm}{I.8.22} Since every well-ordered set is isomorphic to a unique ordinal, it is sufficient to prove the following; for any ordinal $\alpha$ and $X\subseteq\alpha$, it holds that $\type(X;\in)\leq\alpha$. Now assume $\alpha\in ON$ and  $X\subseteq\alpha$, and let $f$ be the order isomorphism from $X$ to $\type(X)$. Since $f$ is an order-isomorphism, it holds that $f(\xi)=\{f(\mu):\mu\in X\wedge\mu<\xi\}$.
Now let $\Xi = \{\xi\in X:\xi<f(\xi)\}$. Assume $\Xi\neq\emptyset$ and let $\xi$ be the least element of $\Xi$. From $f(\xi)=\{f(\mu):\mu\in X\wedge\mu<\xi\}$ and $\xi\in\Xi$, we get $\xi=f(\mu)$ for some $\mu\in X$ such that $\mu<\xi$.
Since $\xi$ is the least element of $\Xi$, $\mu$ is not in $\Xi$ and hence $f(\mu)\leq\mu$ holds. Then we have $\xi\leq f(\mu)\leq\mu$, which is a contradiction. Thus $\Xi=\emptyset$ and hence $\type(X)\subseteq\alpha$ holds.
\end{customthm}

\begin{customthm}{I.8.23} Skipped for now.
\end{customthm}

\begin{customthm}{I.9.6} Let $x$ be a set.
\begin{enumerate}
\item $\trcl(x)\supseteq\bigcup^0x=x$.
\item Let $y\in\trcl(x)$. Then there is $n\in\omega$ such that $y\in\bigcup^nx$. For such $n$ and $z\in y$, it holds that $z\in\{z:\exists y\in\bigcup^nx\,(z\in y)\}=\bigcup\bigcup^nx=\bigcup^{n+1}x\subseteq\trcl(x)$. Hence $\trcl(x)$ is transitive.
\item Let $t$ be a transitive set. Note that $\bigcup t\subseteq t$ holds by the definition of transitivity. Therefore it can be inductively shown that $\trcl(t)$ is a union of subsets of $t$, which is also a subset of $t$. On the other hand, if $y$ is a superset of $x$, then $\trcl(x)\subseteq\trcl(y)$ holds because we can show that $\bigcup^nx\subseteq\bigcup^ny$ by induction. Thus if $x\subseteq t$, it holds that $\trcl(x)\subseteq\trcl(t)\subseteq t$.
\item For $y\in x$, one can easily show that $\bigcup^ny\subseteq\bigcup^{n+1}x$ by induction. Now let $y\in x$, $z\in\trcl(y)$ and $n\in\omega$ be one with $z\in\bigcup^ny$. Then we get $z\in\bigcup^{n+1}x\subseteq\trcl(x)$, which proves $\trcl(y)\subseteq\trcl(x)$.
\item We show the following proposition $P(n)$: $z\in\bigcup^nx$ iff there is an $\in$-path from $z$ to $x$ of length $n+1$. Note that an $\in$-path from $z$ to $x$ of length $n\in\omega$ is a function $s$ such that $\mathrm{dom}(s)=n+1$, $s(0)=z$, $s(n)=x$, and $s(i)\in s(i+1)$ for all $i<n-1$.
$P(0)$ is obvious by taking the function $s$ such that $s(0)=z=x$. Now assume $P(n)$ holds. Let $z\in\bigcup^{n+1}x=\{z:\exists y\in\bigcup^nx\,(z\in y)\}$ and $y\in\bigcup^n x$ be one such that $z\in y$. By induction hypothesis, we have a function $s$ such that $\mathrm{dom}(s)=n+1$, $s(0)=y$, $s(n)=x$, and $s(i)\in s(i+1)$ for all $i<n-1$.
Now we extend this to a function $s'$ such that $\mathrm{dom}(s')=n+2$, $s(0)=x$, $s'(i+1)=s(i)$ for all $i\leq n$. Then $s'$ turns out to be an $\in$-path from $z$ to $x$ of length $n+1$. Therefore the 'only if' part of $P(n+1)$ is proved. The converse can be proved by considering a sub-path $s$ of an $\in$-path $s'$ from $z$ to $x$ of length $n+1$, where $s$ is the path from $s'(1)$ to $x$ of length $n$.
Thus $P(n+1)$ holds, and so does $\forall n\in\omega\,P(n)$.
\end{enumerate}
\end{customthm}

\begin{customthm}{I.10.5} We can say that $A\times B=\{\{\{x\},\{x,y\}\}\in\mathcal{P(\mathcal{P(A\cup B)})}:x\in A,y\in B\}$ is a set, relying on the specific definition of pairs. $A/R=\{[x]_R\in\mathcal{P}(A):x\in A\}$, where $[x]_R=\{y\in A:yRx\}$, obviously forms a set.
\end{customthm}

\begin{customthm}{I.11.3} $\mathbb{R}\times\mathbb{R}\approx(0,1)\times(0,1)$ holds because $f(x,y)=(\sigma(x),\sigma(y))$, where $\sigma(x)=(1+\exp(x))^{-1}$, is a bijection from $\mathbb{R}\times\mathbb{R}$ to $(0,1)\times(0,1)$.
To prove $(0,1)\times(0,1)\preccurlyeq(0,1)$, we represent $x\in(0,1)$ by an never-ending infinite decimal representation $x=0.x_1x_2\cdots$. Note that any finite decimal $x=0.x_1x_2\cdots x_n$ can be represented that way, i.e. $x=0.x_1x_2\cdots(x_n-1)999\cdots$.
Now we define a function from $(0,1)\times(0,1)$ by $f(0.x1x2\cdots, 0.y1y2\cdots) = 0.x1y1x2y2\cdots$. This is an injection to $(0,1)$ because any number in $(0,1)$ is uniquely represented by the infinite decimal. Thus $(0,1)\times(0,1)\preccurlyeq(0,1)$ holds. $(0,1)\approx\mathbb{R}$ is clear by considering $\sigma^{-1}$.
\end{customthm}

\begin{customthm}{I.11.6} Skipped for now.
\end{customthm}

\begin{customthm}{I.11.12} Let $A,B,C,D$ be sets with $A\preccurlyeq B$, $C\preccurlyeq D$, let $i:A\rightarrow B$, $j:C\rightarrow D$ be injections. Note that if $B=\emptyset$ is allowed, there would be a counterexample of ${}^AC\preccurlyeq{}^BD$. For example, $A=B=C=\emptyset$ and $D=1$ would yield ${}^AC\approx 1\not\preccurlyeq 0={}^BD$. Now Let us further assume that $B\neq\emptyset$ and fix $b\in B$.
We define a function $F:{}^AC\rightarrow{}^BD$ by the following.
\[F(f)(d)=
  \begin{cases}
    i(f(c)) & \mbox{if } d=j(c) \\
    b & \mbox{if } d\not\in\mathrm{ran}(j)
  \end{cases}.
\]
One can easily check that this $F$ is injective and hence ${}^AC\preccurlyeq{}^BD$ holds. The rest of the exercise, namely if $2\prec C$ then $A\prec \mathcal{P}(A)\preccurlyeq{}^AC$, follows from $\mathcal{P}(A)\approx 2^A$ and the above by letting $B=A, C=2, D=C$ in ${}^AC\preccurlyeq{}^BD$. Note that $\mathcal{P}(A)\preccurlyeq{}^AC$ holds even when $A=\emptyset$, where both sides are of cardinality $1$.
\end{customthm}

\begin{customthm}{I.11.15} (1) is obvious from Exercise I.8.22. (2) can be reduced to (1) if we replace $B$ by $i(B)$ where $i:B\rightarrow\alpha$ is an injection. To prove (3), let $\alpha,\beta,\gamma$ be ordinals with $\alpha\leq\beta\leq\gamma$, $\alpha\approx\gamma$ and let $f:\gamma\rightarrow\alpha$ be a bijection.
Since $\alpha\subseteq\beta$ holds, $f$ is a function from $\gamma$ to $\beta$ as well, and since $\beta\subseteq\gamma$, we have $f\upharpoonright\beta:\beta\rightarrow\alpha$. Both of those functions are injective and hence we get $\gamma\preccurlyeq\beta\preccurlyeq\alpha$. Combining these with $\alpha\leq\beta\leq\gamma$ and Schröder-Bernstein Theorem, we get $\alpha\approx\beta\approx\gamma$.
\end{customthm}

\begin{customthm}{I.11.19} The 'only if' part is obvious since any order isomorphism is a bijection. To prove the 'if' part, let $A$ be a set, $\alpha$ be an ordinal and $f:A\rightarrow\alpha$ be a bijection. Then the relation $R=\{\langle x,y\rangle\in A\times A:f(x)\leq f(y)\}$ well-orders $A$.
\end{customthm}

\begin{customthm}{I.11.21} Let $A$ be a well-orderable set and $g:A\rightarrow|A|$ be a bijection. Let $B$ be a set and assume there is a surjection $f:A\rightarrow B$.  Since $f$ is surjective, we can define an injection $i:B\rightarrow|A|$ by $i(y)=\min\{g(x)\in|A|:x\in A\wedge y=f(x)\}$. Then by Exercise I.11.15(2) and Exercise I.11.19, we can say that $B$ is well-orderable and $|B|\leq|A|$.
\end{customthm}

\begin{customthm}{I.11.22} The 'if' part is obvious from Exercise I.11.21. To prove the converse, let $\kappa$ be a cardinal, $B$ be a non-empty set with $B\preccurlyeq\kappa$. For a fixed $b\in B$ and an injection $i:B\rightarrow\kappa$, we can define a surjection $f:\kappa\rightarrow B$ as follows.
  \[f(\alpha)=
    \begin{cases}
      x & \mbox{if } \alpha=i(x) \\
      b & \mbox{if } \alpha\not\in\mathrm{ran}(i)
    \end{cases}.
  \]
\end{customthm}

\begin{customthm}{I.11.23} Let $f:A\rightarrow|A|$, $g:B\rightarrow|B|$ be bijections.
\begin{enumerate}
\item \underline{$|A|$ is a cardinal}. Assume not and let $\xi$ be an ordinal such that $\xi<|A|$ and $\xi\approx|A|$. Then $A\approx\xi$ holds by $A\approx|A|$, contradicting the minimality of $|A|$.
\item \underline{$A\preccurlyeq B$ iff $|A|\leq|B|$}. Assume $A\preccurlyeq B$ and let $i:A\rightarrow B$ be an injection. Then $g\circ i:A\rightarrow|B|$ is an injection, and by Exercise I.11.15(2), there exists an ordinal $\delta\leq|B|$ with a bijection $h:A\rightarrow\delta$. For such $\delta$ and $h$, we can define an injection $h\circ f^{-1}:|A|\rightarrow|B|$. Hence we have $|A|\leq|B|$.
The converse is obvious from an injection $g^{-1}\circ f:A\rightarrow B$.
\item \underline{$A\approx B$ iff $|A|=|B|$}. This is obvious from (2) since both $A\preccurlyeq B\wedge B\preccurlyeq A\leftrightarrow A\approx B$ (Schröder-Bernstein Theorem) and $|A|\leq|B|\wedge|B|\leq|A|\leftrightarrow|A|=|B|$ hold.
\item \underline{$A\prec B$ iff $|A|<|B|$}. This is obvious from (2), (3) and the definitions of $\prec$ and $<$.
\end{enumerate}
\end{customthm}

\begin{customthm}{I.11.24} The 'if' part is obvious since $A\preccurlyeq|A|$ holds. To prove the converse, let $A$ be a finite set and let $n\in\omega$ be one with $A\preccurlyeq n$. Then by Exercise I.11.15(2), $A$ is well-orderable and there exists $m\leq n$ such that $A\approx m$. Now it is obvious that $|A|\leq m\leq n<\omega$.
\end{customthm}

\begin{customthm}{I.11.25} We prove by induction the following proposition $P(n)$: $|A\cup B|\leq m+n$, equality holding iff $A\cap B=\emptyset$. $P(0)$ is obvious since if $n=0$, then we have $B=\emptyset$, $|A\cup B|=|A|=m$ and $A\cap B=\emptyset$. Now assume $P(n)$ and let $B'=B\cup\{b\}$ with $b\not\in B$, $|B|=n$.
If $b\in A$, we have $|A\cup B'|=|A\cup B|\leq m+n<m+n+1$ by induction hypothesis. Note that $A\cap B'\neq\emptyset$ since $A$ and $B'$ has $b$ in common, in which case the equality does not hold. If $b\not\in A$, we can define an injection $i':A\cup B'\rightarrow m+n+1$ by $i\cup\{\langle b',m+n\rangle\}$, where $i$ is an induction $i:A\cup B\rightarrow m+n$.
This is a bijection if $i$ is a bijection, which is by induction hypothesis equivalent to $A\cap B=\emptyset$, which is also equivalent to $A\cap B'=\emptyset$. Thus $P(n+1)$ holds under $P(n)$ and we have $\forall n\in\omega\,P(n)$. One can also prove $|A\times B|=m\cdot n$ by induction on $n$, using the fact that $A\times (B\sqcup\{b\})\approx(A\times B)\sqcup(A\times\{b\})$, where $\sqcup$ denotes the disjoint union operator.
\end{customthm}

\begin{customthm}{I.11.28} Let $\alpha$ be an ordinal. By Exercise I.11.23, we have $\beta\not\preccurlyeq\alpha\leftrightarrow\beta\not\leq\alpha\leftrightarrow\alpha<\beta$ for any ordinal $\beta$. Thus $\alpha^+$, which is by definition the least cardinal $\kappa$ such that $\kappa\not\preccurlyeq\alpha$, is the least cardinal greater than $\alpha$. Now note that $\alpha+1$ is the least ordinal greater than $\alpha$ (See Exercise I.8.11). Therefore $\alpha^+\geq\alpha+1$ holds.
When $\alpha\in\omega$, by Exercise I.8.13 and Theorem I.11.17(2), $\alpha+1\in\omega$ turns out to be a cardinal and hence $\alpha^+=\alpha+1$. When $\alpha\geq\omega$, $\alpha+1$ is not a cardinal since we have an injection $i:\alpha+1\rightarrow\alpha$ defined as follows.
\[i(x)=
  \begin{cases}
    0 & \mbox{if } x=\alpha \\
    x+1 & \mbox{if } x\in\omega \\
    x & \mbox{otherwise}
  \end{cases}.
\]
Thus $\alpha^+>\alpha+1$ holds.
\end{customthm}

\begin{customthm}{I.11.29} Let $A$ be a set and $W,\beta,\kappa$ be the ones occurring in the proof of Theorem I.11.26. Let $\alpha$ be an ordinal such that $\alpha<\beta$. Then there is a pair $(X,R)$ such that $\alpha<\type(X;R)+1$, which is equivalent to $\alpha\leq\type(X;R)$. Therefore $\alpha\preccurlyeq\type(X;R)\approx X\preccurlyeq A$ holds.
If $\alpha\not\prec\beta$, we would have $\beta\preccurlyeq\alpha\preccurlyeq A$ and that contradicts $\beta\not\preccurlyeq A$. Hence $\alpha\prec\beta$ holds and $\beta$ turns out to be a cardinal.
$\beta=\aleph(A)$ is obvious since $\alpha<\beta$ implies $\alpha\preccurlyeq A$ from the above, the contraposition of which is nothing but a proof of $\beta$'s minimality.
\end{customthm}

\begin{customthm}{I.11.30} Let $\xi,\zeta$ be ordinals. We prove $\xi<\zeta\rightarrow\aleph_\xi<\aleph_\zeta$ by induction on $\zeta$. If $\zeta=\xi+1$, which is the least ordinal greater than $\xi$, then $\aleph_\zeta=(\aleph_\xi)^+$ is greater than $\aleph_\xi$ by the definition of $\aleph$.
If $\zeta=\eta+1$ and $\eta>\xi$, assuming the induction hypothesis $\aleph_\eta>\aleph_\xi$, we get $\aleph_\zeta=(\aleph_\eta)^+>(\aleph_\xi)^+>\aleph_\xi$.
If $\zeta$ is a limit ordinal and $\aleph_\xi<\aleph_\eta$ for all $\xi<\eta<\zeta$, then $\aleph_\zeta=\sup\{\aleph_\eta:\eta<\zeta\}\geq\aleph_{\xi+1}>\aleph_\xi$. Hence $\xi<\zeta\rightarrow\aleph_\xi<\aleph_\zeta$ holds.

Now we prove that $\kappa$ is an infinite cardinal iff $\kappa=\aleph_\xi$ for some $\xi$. The 'if' part is obvious from the definition of $\aleph$ and Theorem I.11.17(3). To prove the converse by contradiction, assume there is an infinite cardinal $\kappa$ such that $\forall\xi\in ON\,[\kappa\neq\aleph_\xi]$. Let $\kappa$ be the least cardinal such that $\forall\xi\in ON\,[\kappa\neq\aleph_\xi]$ and let $\xi=\{\eta\in ON:\aleph_\eta<\kappa\}$.
Note that $\xi$ is an an initial segment of $ON$ by the above. Furthermore, $\xi$ forms a set. Otherwise, $\xi=ON$ and $\kappa$ would contain all $\aleph_\eta$, which by Replacement implies $ON$ is a set.
Now we prove $\kappa=\aleph_\xi$. One can easily prove that $\aleph_\xi\leq\kappa$ by splitting $\xi$ into $0$, a successor or a limit ordinal. On the other hand, $\aleph_\xi\not<\kappa$ holds because $\aleph_\xi<\kappa$ would imply $\xi\in\xi$. Thus $\kappa=\aleph_\xi$ holds, which contradicts with the definition of $\kappa$. Hence any infinite cardinal is identical to $\aleph_\xi$ for some $\xi$.
\end{customthm}

\begin{customthm}{I.11.33} We first prove $\aleph_\gamma\leq\gamma$. Since $\gamma$ is a limit ordinal, $\aleph_\gamma=\sup\{\aleph_\xi:\xi<\gamma\}$ holds and $\aleph_\gamma\leq\gamma$ is equivalent to $\forall\xi<\gamma\,[\aleph_\xi\leq\gamma]$.
Let $\xi<\gamma$ and $n\in\omega$ be one such that $\xi<\delta_n$. Then we have $\aleph_\xi<\aleph_{\delta_n}=\delta_{n+1}<\gamma$. Hence $\aleph_\gamma\leq\gamma$ holds.
$\gamma\leq\aleph_\gamma$ is clear from $\forall n\in\omega\,[\delta_n<\aleph_{\delta_n}<\aleph_\gamma]$, and we get $\aleph_\gamma=\gamma$.

Let $\xi$ be an ordinal such that $\aleph_\xi=\xi$. Since $\aleph$ is monotonic by Exercise I.11.32, we get $\forall n\in\omega\,[\delta_n\leq\xi]$ by repeatedly applying $\aleph$ to the both side of $0\leq\xi$.  Hence $\gamma\leq\xi$ holds and $\gamma$ turns out to be the least fixed point of $\aleph$.
\end{customthm}

\begin{customthm}{I.11.34} Let us first note that $\max\{|\alpha|,|\beta|\}=|\max\{\alpha,\beta\}|$. For $|\alpha+\beta|$ and $|\alpha\cdot\beta|$, WLOG we can assume $\alpha\leq\beta$. Then we have $|\alpha+\beta|\leq|\beta+\beta|\leq|\beta\cdot\beta|=|\beta\times\beta|=|\beta|$.
For $|\alpha^\beta|$, let $\gamma=\max\{\alpha,\beta\}$. We prove $|\alpha^\beta|\leq|\gamma|$ by induction on $\beta$. If $\beta=2$, then we have $\omega\leq\alpha$ and hence $|\alpha^\beta|=|\alpha\cdot\alpha|=|\alpha\times\alpha|=|\alpha|\leq|\gamma|$.
Now assume that $|\alpha^\delta|\leq|\gamma|$ for any $\delta<\beta$.
If $\beta=\delta+1$ for some $\delta\geq 2$, then $|\alpha^\beta|=|\alpha^\delta\cdot\alpha|=\max\{|\alpha^\delta|,|\alpha|\}\leq|\gamma|$.
If $\beta$ is a limit ordinal, then $|\alpha^\beta|=|\beta\times\gamma|$ holds since we can define an injection $i:\alpha^\beta\rightarrow\beta\times\gamma$ as $i(\xi)=\langle\delta,i_\delta(\xi)\rangle$,
where $(i_\delta:\alpha^\delta\rightarrow\gamma)_{\delta<\beta}$ a family of injections, whose existence relies on Choice, and $\delta$ is the least ordinal $\eta$ such that $\xi\leq\alpha^\eta$.
Hence we get $|\alpha^\beta|=|\beta\times\gamma|\leq|\gamma\times\gamma|=|\gamma|$. It is easy to see that $|\gamma|\leq|\alpha^\beta|$ and thus we have $|\alpha^\beta|=|\gamma|=\max\{|\alpha|,|\beta|\}$. (Note: I will replace this proof later by one without Choice, if I find any.)
\end{customthm}

\begin{customthm}{I.11.35} Let $W=\{(X,R)\in\mathcal{P}(\omega)\times\mathcal{P(\omega\times\omega)}:R\text{ well-orders }X\}$ and $B=W/\cong$, where $\cong$ is the isomorphism relation. We define a relation $\triangleleft$ on $B$ by $[(X,R)]\triangleleft[(X',R')]$ iff $\type(X;R)<\type(X';R')$.
It is easy to see that $\triangleleft$ is well-defined and it well-orders $B$. Now let us denote $X_{<x}=\{y\in X:y<x\}$ for an ordered set $(X;<)$ and its element $x$. For any $(X,R)\in W$, $(X,R)\cong B_{\triangleleft[(X,R)]}$ holds because there is an order isomorphism $f:(X,R)\rightarrow B_{\triangleleft[(X,R)]}$ defined by $f(x)=[(X_{<x},R\upharpoonright X_{<x})]$.
Therefore if $B$ is countable, it is isomorphic to some $(X,R)\in W$ and is also isomorphic to a proper initial segment of itself (namely $B_{\triangleleft[(X,R)]}$), which is a contradiction. Thus $B$ is uncountable. Note that power sets and quotient sets can be constructed without Replacement by Exercise I.10.5 and hence we have an uncountable well-ordered set within $Z^-$.
\end{customthm}

\begin{customthm}{I.11.36} Let $(A,\triangleleft)$ be a total order with $A=\{a_n:n\in\omega\}$. For $n\in\omega$, let $L_n=\{m\in\omega:m<n\wedge a_m<a_n\}$ and $R_n=\{m\in\omega:m<n\wedge a_n<a_m\}$. we define $f:A\rightarrow\mathbb{Q}$ as follows.
  \[f(a_n)=
    \begin{cases}
      0 & \mbox{if } n=0 \\
      f(\min(R_n))-1 & \mbox{if } n>0 \mbox{ and } L_n=\emptyset \\
      f(\max(L_n))+1 & \mbox{if } n>0 \mbox{ and } R_n=\emptyset \\
      \frac{1}{2}(f(\max(L_n))+f(\min(R_n))) & \mbox{if } n>0 \mbox{, } L_n\neq\emptyset \mbox{ and } R_n\neq\emptyset
    \end{cases}.
  \]
It is easy to see that this $f$ actually is an order-preserving injection.
\end{customthm}

\begin{customthm}{I.11.37} Skipped for now.
\end{customthm}

\begin{customthm}{I.11.38} Skipped for now.
\end{customthm}

\begin{customthm}{I.12.6} Let $\mathcal{F}$ be of finite character, $X\in\mathcal{F}$, and $Y\subseteq X$. $Y\in\mathcal{F}$ holds because every finite subset of $Y$ is in $\mathcal{F}$ since it is a finite subset of $X\in\mathcal{F}$ as well.
\end{customthm}

\begin{customthm}{I.12.13} We define a function $f:\mathcal{P}(\omega\times\omega)\rightarrow\omega_1$ as follows.
  \[f(R)=
    \begin{cases}
      \type(R) & \mbox{if } R \mbox{ well-orders } \omega \\
      |R| & \mbox{if } R \mbox{ is finite} \\
      0 & \mbox{otherwise}
    \end{cases}.
  \]
This $f$ is surjective because $\mathrm{ran}(f)$ includes all countable ordinals. On the other hand, since $\omega\times\omega\preccurlyeq\omega$, we have an injection from $\mathcal{P}(\omega\times\omega)\approx {}^{\omega\times\omega}2$ to $\mathcal{P}(\omega)\approx {}^\omega 2$ by Exercise I.11.12.
Hence we have a surjection from $\mathcal{P}(\omega)$ to $\mathcal{P}(\omega\times\omega)$ and we get a surjection from $\mathcal{P}(\omega)$ to $\omega_1$ composing that surjection and $f$.
\end{customthm}

\begin{customthm}{I.12.14} For each $n\in\omega$, let $\mathcal{F}_n=\{Y\in\mathcal{P}(X):\forall x,y\in Y\,[x\neq y\rightarrow d(x,y)>2^{-n}]\}$. Then $\mathcal{F}_n$ is of finite character and hence it has a maximal element $E_n$ by Tukey's Lemma.
Now let $E=\bigcup_n E_n$. If $E$ is not dense in $X$, there is $x\in X$ and $n\in\omega$ such that the ball of radius $2^{-n}$ centered at $x$ does not intersect $E$, which contradicts $E_n$'s maximality. Thus $E$ is dense and since $X$ is non-separable, $E$ is uncountable. Then there is some uncountable $E_n$ because countable union of countable sets would be countable. That $E_n$ with $\varepsilon=2^{-n}$ is what we want.
\end{customthm}

\begin{customthm}{I.12.16} Let $X,Y$ be compact Hausdorff spaces, $f:X\rightarrow Y$ be a continuous surjection, and $\mathcal{F}=\{Z\in\mathcal{P}(X):Z\text{ is closed and }f\upharpoonright Z\text{ is surjective}\}$. Let $C$ be a chain in $(\mathcal{F},\supsetneq)$. We prove $\bigcap C\in\mathcal{F}$. It is clear that $\bigcap C$ is closed because $C$ is the intersection of closed sets.
Assume $f\upharpoonright\bigcap C$ is not surjective and let $y\in Y\setminus\bigcap C$. Since $f$ is continuous and any singleton in a Hausdorff space is closed, $f^{-1}(y)=\{x\in X:f(x)=y\}$ is a closed subset in $X$, which is compact since $X$ is compact. Then $X\setminus\bigcap C=\bigcup_{Z\in C}(X\setminus Z)$, an open cover of $f^{-1}(y)$, has a finite subcover $(X\setminus Z_1,X\setminus Z_2,\ldots,X\setminus Z_n)$.
We can assume $Z_1\supseteq Z_2\supseteq\ldots\supseteq Z_n$ since $C$ is a chain. Then $X\setminus Z_n=\bigcup_{k=1,\ldots,n}(X\setminus Z_k)\supseteq f^{-1}(y)$ holds. This means that $y\not\in\mathrm{ran}(f\upharpoonright Z_k)$, which contradicts $Z_n\in\mathcal{F}$. Hence $f\upharpoonright\bigcap C$ is surjective and $C\in\mathcal{F}$ holds.
Now we have $\bigcap C$ as an upper bound for any chain $C$, $\mathcal{F}$ turns out to have a maximal set $Z$ by Zorn's Lemma. Such $Z$ is what we want.
\end{customthm}

\begin{customthm}{I.12.17} Let $A$ be a set. To prove the 'only if' part, assume $\omega\preccurlyeq A$ and let $i:\omega\rightarrow A$ be an injection. Then we can define an injection $f:A\rightarrow A$ with $\mathrm{ran}(f)\neq A$ as follows.
  \[f(x)=
    \begin{cases}
      f(i(n+1)) & \mbox{if } x=i(n) \\
      x & \mbox{otherwise}
    \end{cases}.
  \]
This $f$ does not contain $i(0)\in A$ in its range. Hence $A$ is not Dedekind-finite.

To prove the 'if' part, assume $A$ is not Dedekind-finite and let $f:A\rightarrow A$ be an injection with $\mathrm{ran}(f)\neq A$. Then there is $a\in A$ such that $f^n(a)\neq a$ holds for all positive natural number $n$, because otherwise every $a\in A$ would be in the range of $f$ for $a=f(f^{n-1}(a))$. Now fix such $a\in A$ and define $i:\omega\rightarrow A$ by $i(n)=f^n(a)$. This is an injection from $\omega$ to $A$ and we have $\omega\preccurlyeq A$.
\end{customthm}

\begin{customthm}{I.12.18} Let $A$ be dense in $X$. $A$ cannot be finite because $X$ is not finite. (To see that, one can take $x\in X\setminus A$ and consider a ball $B=\{y\in X: |y-x|<\min\{|x-a|:a\in A\}\}$. $B$ does not intersect $A$.) $A$ cannot be countably infinite either because it would give a bijection $i:\omega\rightarrow A$, which gives rise to an injection to $X$, contradicting that $X$ is not Dedekind-finite. This can be checked by Exercise I.12.17. Thus $A$ has to be uncountable, which proves inseparability of $X$.
\end{customthm}

\begin{customthm}{I.13.17} Skipped for now.
\end{customthm}

\begin{customthm}{I.13.18} Let $\delta_0=0$, $\delta_{n+1}=\beth_{\delta_n}$ and $\gamma=\sup\{\delta_n:n\in\omega\}$. Then $\beth_\gamma=\gamma$ holds as is proved in Exercise I.11.33.
\end{customthm}

\begin{customthm}{I.13.20} Skipped for now.
\end{customthm}

\begin{customthm}{I.13.21} We can define an injection $i:A\rightarrow\bigcup_{n\in\omega}(K\times B)^n$ by $i(a_1v_1+\cdots+a_nv_n)=((a_1,v_1),\ldots(a_n,v_n))$. Since $\omega\leq\max(|K|,|B|)$ and $(K\times B)^n\approx\max(|K|,|B|)^n\approx\max(|K|,|B|)$ hold, we have $\bigcup_{n\in\omega}(K\times B)^n\approx\max(|K|,|B|)$ by Theorem I.12.11. Hence $|A|\leq\max(|K|,|B|)$ holds.
On the other hand, $|B|\leq|A|$ is obvious from $B\subseteq A$ and so is $|K|\leq|A|$ because we can define an injection $i:K\rightarrow A$ by $i(a)=av$ for a fixed $v\in B$. Thus we have $|A|=\max(|K|,|B|)$.
Now we view $\mathbb{R}$ and $\mathbb{C}$ as vector spaces over $\mathbb{Q}$ with bases $B_\mathbb{R}$ and $B_\mathbb{C}$ respectively. Then by the above, we have $|\mathbb{R}|=\max(|\mathbb{Q}|, |B_\mathbb{R}|)$.
Since $|\mathbb{Q}|=\aleph_0<2^{\aleph_0}=|\mathbb{R}|$, we get $|B_\mathbb{R}|=2^{\aleph_0}$. The same argument goes for $\mathbb{C}$ and we get $|B_\mathbb{C}|=|B_\mathbb{R}|=2^{\aleph_0}$. Hence $\mathbb{R}$ and $\mathbb{C}$ have the same dimensionality as vector spaces over $Q$, which means they are isomorphic.
Since isomorphisms between vector spaces are group isomorphisms as well, we can conclude that $\mathbb{R}$ and $\mathbb{C}$ are isomorphic as Abelian groups.
\end{customthm}

\begin{customthm}{I.14.4} The list is too long to show here. You can find python code in \url{https://github.com/kumatheworld/math-playground/blob/master/foundations.py}.
\end{customthm}

\begin{customthm}{I.14.9} Let $x\in WF$,  $\beta=\rank(x)$ and $\gamma=\rank(\bigcup x)$. Note that $\gamma\leq\beta$ holds from Lemma I.14.8(4). We prove the following.
\begin{enumerate}
\item\underline{$\gamma=\beta$ if $\beta$ is a limit ordinal or 0}. If $x=0$, then $\bigcup x=\emptyset$ and it is obvious that $\gamma=\beta=0$. Now assume $\beta$ is a limit ordinal. To see $\gamma\geq\beta$, we show that $\xi<\gamma$ for all $\xi<\beta$.
Let $\xi$ be an ordinal such that $\xi<\beta$. Then by Theorem I.14.6, there exists $y\in x$ such that $\rank(y)\geq\xi+1$ and $z\in y$ such that $\rank(z)\geq\xi$. Since $z\in y\in x$ holds, $z\in\bigcup x$ and we have $\gamma\geq\rank(z)+1>\xi$. Thus $\gamma=\beta$ holds.
\item\underline{$\gamma=\alpha$ if $\beta=\alpha+1$}. Let $\beta=\alpha+1$. If $z\in\bigcup x$ and $y$ be one such that $z\in y\in x$, then $\rank(z)<\rank(y)<\rank(x)=\alpha+1$ holds, which means $\rank(z)+1\leq\alpha$. Hence we have $\gamma\leq\alpha$.
On the other hand, there exists $y\in z$ such that $\rank(y)=\alpha$. Then $y\subseteq\bigcup x$ holds and we have $\alpha\leq\gamma$ by Lemma I.14.7. Thus $\gamma=\alpha$ holds.
\end{enumerate}
\end{customthm}

\begin{customthm}{I.14.14} Let $f:\omega\rightarrow HF$ be the one defined recursively by $f(m)=\{f(n):nEm\}$ and $g:HF\rightarrow\omega$ be the one defined recursively by $g(x)=\sum_{y\in x}2^{g(y)}$. To see $(\omega;E)\cong(HF;\in)$, we show the following.
\begin{enumerate}
\item\underline{$g\circ f = \mathrm{id}_\omega$}. We show this inductively. Let $m\in\omega$ and assume $g(f(n))=n$ for all $n<m$. Then we have $g(f(m))=g(\{f(n):nEm\})=\sum_{n:nEm}2^{g(f(n))}=\sum_{n:nEm}2^n=m$.
\item\underline{$f\circ g = \mathrm{id}_{HF}$}. We show this inductively too. Let $x\in HF$ and assume $f(g(y))=y$ for all $y\in x$. Then we have $f(g(x))=\{f(n):nE_{g(x)}\}=\{f(n):nE_{\sum_{y\in x}2^{g(y)}}\}=\{f(g(y)):y\in x\}=\{y:y\in x\}=x$.
\item\underline{$nEm\leftrightarrow f(n)\in f(m)$}. This is obvious from the definition of $E$.
\end{enumerate}
From the above $f$ turns out to be an order isomorphism and we get $(\omega;E)\cong(HF;\in)$.
\end{customthm}

\begin{customthm}{I.14.15} Assume $WF\subsetneq K$ and let $x\in WF\setminus K$ be one with the lowest rank. Since $x\not\in K$, $x\subsetneq K$ holds and hence there is $y\in x\setminus K$. From Lemma I.14.4(5), $y\in WF$ and $\rank(y)<\rank(x)$ hold, which contradicts the minimality of $\rank(x)$. Thus $WF\subseteq K$ holds.
\end{customthm}

\begin{customthm}{I.14.16} We show $|R(\omega+\xi)|=\beth_\xi$ by induction. The equation holds when $\xi=0$ because we have $|R(\omega)|=\omega$ from Exercise I.14.14.
When $\xi$ is a successor, let $\xi=\eta+1$ and assume $|R(\omega+\eta)|=\beth_\eta$. Then we have $|R(\omega+\xi)|=|2^{R(\omega+\eta)}|=2^{|R(\omega+\eta)|}=2^{\beth_\eta}=\beth_\xi$.
Now assume $\xi$ is a limit ordinal and $|R(\omega+\eta)|=\beth_\eta$ holds for all $\eta<\xi$. It is obvious that $|R(\omega+\xi)|\geq\beth_\xi=\sup\{\beth_\eta:\eta<\xi\}$ because $|R(\omega+\xi)|\geq|R(\omega+\eta)|=\beth_\eta$ holds for all $\eta<\xi$.
To see $|R(\omega+\xi)|\leq\beth_\xi$, we first note that $R(\omega+\xi)=\bigcup_{\zeta<\omega+\xi}R(\zeta)=\bigcup_{\eta<\xi}R(\omega+\eta)$ holds since $\omega+\xi$ is a limit ordinal as well. The last equality follows from $R(n)\subseteq R(\omega)$ for all $n<\omega$.
On the other hand, $\omega+\xi\leq\beth_\xi$ and $R(\omega+\eta)\leq\beth_\xi$ hold for any $\xi\in ON$. One can easily show the first inequality by induction and the second one is obvious from our induction hypothesis. Then by Theorem I.12.11, we get $|R(\omega+\xi)|=|\bigcup_{\eta<\xi}R(\omega+\eta)|\leq\beth_\xi$. Thus $|R(\omega+\xi)|=\beth_\xi$ holds for all ordinals $\xi$.
\end{customthm}

\begin{customthm}{I.14.17} Assume $\forall x\in A\,\exists y\,\varphi(x,y)$. Let $\hat\varphi(x,\alpha)$ say that $\alpha$ is the least ordinal such that $\exists y\in R(\alpha)\,\varphi(x,y)$.
Then $\forall x\in A\,\exists!\alpha\,\hat\varphi(x,\alpha)$ holds and we get $\exists\hat B\,\forall x\in A\,\exists\alpha\in\hat B\,\hat\varphi(x,\alpha)$ from Replacement.
For such $\hat B$, let $B=\bigcup_{\alpha\in B\cap ON}R(\alpha)$. Then we have $\forall x\in A\,\exists y\in B\,\varphi(x,y)$.
\end{customthm}

\begin{customthm}{I.14.19} Skipped for now.
\end{customthm}

\begin{customthm}{I.14.20} Skipped for now.
\end{customthm}

\begin{customthm}{I.14.21} Skipped for now.
\end{customthm}

\begin{customthm}{I.14.22} Skipped for now.
\end{customthm}

\begin{customthm}{I.14.23} Let $\psi(y)$ be a formula such that $\forall y\,[(\forall x\in y\,\psi(x))\rightarrow\psi(y)]$. Assume $\neg\forall y\,\psi(y)$ and let $y$ be a set of least rank such that $\neg\psi(y)$.
Then by the contraposition of $(\forall x\in y\,\psi(x))\rightarrow\psi(y)$, there is a set $x\in y$ such that $\neg\psi(x)$. Here we have $\rank(x)<\rank(y)$ since $x\in y$, which contradicts the minimality of $\rank(y)$.
\end{customthm}

\begin{customthm}{I.14.24} Skipped for now.
\end{customthm}

\begin{customthm}{I.15.2} $\mathbb{Z}\subseteq\mathbb{Q}$ is obvious from their definitions. To see $\mathbb{Q}\subseteq HF$, we show $\rank(x)<\omega$ for all $x$. If $x\in\omega$, then $\rank(x)<\rank(\omega)=\omega$.
If $x=\langle i,\langle m,n\rangle\rangle$ for some $i,m,n\in\omega$, then $\rank(x)=\max\{\rank(i),\max\{\rank(m),\rank(n)\}+2\}+2\in\omega$. (For example, we get $\rank(-2/3)=\rank(\langle 1,\langle 2,3\rangle\rangle)=7$ from this.) Hence $\mathbb{Q}\subseteq HF$ holds.
$\rank(\mathbb{Z})\leq\omega$ follows from $\mathbb{Z}\subseteq HF$, and we have $\rank(\mathbb{Z})>n$ for any $n\in\omega$ since $n\in\mathbb{Z}$ and $\rank(n)=n$. Thus $\rank(\mathbb{Z})$ is equal to $\omega$ and so is $\rank(\mathbb{Q})$.
\end{customthm}

\begin{customthm}{I.15.5} Let $x\in\mathbb{R}$. $\rank(x)\leq\omega$ holds because $\rank(q)<\omega$ holds for all $q\in x$. Now let $q\in x$ and let $n\in\omega$ be one such that $-n<q$. Then $-n\in x$ and we have $\rank(x)>\rank(-n)=\rank(\langle 1,\langle n,1\rangle\rangle)=n+4$.
Since $n$ can be arbitrarily large, we have $\rank(x)\geq\omega$. Hence $\rank(x)=\omega$ for all $x\in\mathbb{R}$ and we get $\rank(\mathbb{R})=\omega+1$.
For $z=\langle x,y\rangle\in\mathbb{C}$, we have $\rank(z)=\max\{\rank(x),\rank(y)\}+2=\omega+2$, which proves $\rank(\mathbb{C})=\omega+3$.
\end{customthm}

\begin{customthm}{I.15.8} We prove the following.
\begin{enumerate}
\item\underline{$|\mathbb{R}|=2^{\aleph_0}$}. $|\mathbb{R}|\leq2^{\aleph_0}$ is clear from $\mathbb{R}\subseteq{\mathcal{P}(\mathbb{Q})}$ and $\mathcal{P}(\mathbb{Q})\approx2^\mathbb{Q}\approx2^{\aleph_0}$.
However, the opposite is not straightforward from our definition of $\mathbb{R}$. One can prove it by treating $\mathbb{R}$ as the set of infinite decimals and using the diagonal argument. But we would need to prove that the set of infinite decimals is isomorphic to our $\mathbb{R}$ for that, which would be too long to write down here.
\item\underline{$|C(\mathbb{R},\mathbb{R})|=2^{\aleph_0}$}. $|C(\mathbb{R},\mathbb{R})|\geq2^{\aleph_0}$ is obvious because $\mathbb{R}\succcurlyeq2^{\aleph_0}$ and we have a constant function $c_x:\mathbb{R}\rightarrow\mathbb{R}$ for each $x\in\mathbb{R}$.
To prove the opposite, we use the well-known fact from topology that a continuous function from a Hausdorff space is unique up to the restriction to a dense subspace of their domain. Since $\mathbb{Q}$ is dense in $\mathbb{R}$, we have $C(\mathbb{R},\mathbb{R})\subseteq\mathbb{R}^\mathbb{Q}$ and $\mathbb{R}^\mathbb{Q}\approx(2^{\aleph_0})^\mathbb{Q}\approx2^{\aleph_0\times\mathbb{Q}}\approx2^{\aleph_0}$.
Hence $|C(\mathbb{R},\mathbb{R})|=2^{\aleph_0}$ holds.
\item\underline{$|\mathbb{R}^\mathbb{R}|=2^{2^{\aleph_0}}$}. $\mathbb{R}^\mathbb{R}\approx (2^{\aleph_0})^\mathbb{R}\approx 2^{\aleph_0\times\mathbb{R}}\approx2^\mathbb{R}\approx2^{2^{\aleph_0}}$.
\end{enumerate}
\end{customthm}

\begin{customthm}{I.15.9} The set $\mathcal{B}=\mathcal{B}_{\omega_1}\subseteq\mathcal{P}(\mathbb{R})$ of Borel subsets is defined recursively,
where $\mathcal{B}_0$ is the set of open sets and closed sets of $\mathbb{R}$, $\mathcal{B}_{\alpha+1}$ is the family of all countable unions and intersections of sets in $\mathcal{B}_\alpha$, and $\mathcal{B}_\gamma=\bigcup\{\mathcal{B}_\alpha:\alpha<\gamma\}$ for limit $\gamma$.
Therefore we prove $|\mathcal{B_\alpha}|=2^{\aleph_0}$ by induction on $\alpha\leq\omega_1$ as follows.
\begin{enumerate}
\item\underline{$|\mathcal{B}_0|=2^{\aleph_0}$}. Let $\mathcal{O}$ be the set of open sets. $2^{\aleph_0}\leq|\mathcal{O}|$ is obvious because we have a different open interval $(x,x+1)$ for each $x\in\mathbb{R}$.
To prove $|\mathcal{O}|\leq2^{\aleph_0}$, we use the fact that $\mathbb{R}$ is second-countable. Let $B=\{B_n:n\in\omega\}$ be a countable base of $\mathbb{R}$. Then we have a surjection from ${}^\omega\omega$ to $\mathcal{O}$ defined by $f\mapsto\bigcup_{n\in\omega}B_{f(n)}$.
Therefore $|\mathcal{O}|\leq|{}^\omega\omega|=2^{\aleph_0}$ holds and so does $|\mathcal{O}|=2^{\aleph_0}$.
On the other hand, since there is a 1-to-1 correspondence between closed sets and open sets by taking the complement, we get $\mathcal{B}_0=|2^{\aleph_0}+2^{\aleph_0}|=2^{\aleph_0}$.
\item\underline{$|\mathcal{B}_{\alpha+1}|=2^{\aleph_0}$ under $|\mathcal{B}_\alpha|=2^{\aleph_0}$}.  $|\mathcal{B}_{\alpha+1}|\geq2^{\aleph_0}$ is obvious from $\mathcal{B}_{\alpha+1}\supseteq\mathcal{B}_\alpha$ and $|\mathcal{B}_\alpha|=2^{\aleph_0}$.
There is a surjection from $2\times{}^\omega\mathcal{B}_{\alpha}$ to $\mathcal{B}_{\alpha+1}$ defined by $(0,f)\mapsto\bigcup_{n\in\omega}f(n)$ and $(1,f)\mapsto\bigcap_{n\in\omega}f(n)$,
which proves $|\mathcal{B}_{\alpha+1}|\leq|2\times{}^\omega\mathcal{B}_{\alpha}|=|2\times2^{\aleph_0\times\omega}|=2^{\aleph_0}$. Thus $|\mathcal{B}_{\alpha+1}|=2^{\aleph_0}$ holds.
\item\underline{$|\mathcal{B}_\gamma|=2^{\aleph_0}$ under $|\mathcal{B}_\alpha|=2^{\aleph_0}$ for all $\alpha<\gamma$}.
By Theorem I.12.11 and $\gamma\leq\omega_1\leq2^{\aleph_0}$, we have $|\mathcal{B}_\gamma|=|\bigcup\{\mathcal{B}_\alpha:\alpha<\gamma\}|=2^{\aleph_0}$.
\end{enumerate}
From the above, we get $|\mathcal{B}|=2^{\aleph_0}$.

For the set $\mathcal{L}$ of Lebesgue measurable sets, $|\mathcal{L}|\leq2^{2^{\aleph_0}}$ is obvious from $\mathcal{L}\subseteq\mathcal{P}(\mathbb{R})$.
To prove $\mathcal{L}\supseteq\mathcal{P}(\mathbb{R})$, we make use of the Cantor set, which is of cardinality $2^{\aleph_0}$ and has measure $0$. Since every subset of a null set is measurable, we have at least $2^{2^{\aleph_0}}$ measurable sets. Hence $|\mathcal{L}|=2^{2^{\aleph_0}}$ holds.
\end{customthm}

\begin{customthm}{I.15.10} Skipped for now.
\end{customthm}

\begin{customthm}{I.15.11} Skipped for now.
\end{customthm}

\begin{customthm}{I.15.12} Skipped for now.
\end{customthm}

\begin{customthm}{I.15.14} Let $x\in HF^{<\omega}$. Every $y=(n,z)\in x$ has finite rank and so does $x$. Hence we have $x\in HF$ and $HF^{<\omega}\subseteq HF$.
\end{customthm}

\begin{customthm}{I.15.15} Let $A\subseteq\omega\setminus\{0\}$. Assume $A\cap A^{<\omega}\neq\emptyset$ and let $x\in A\cap A^{<\omega}$. Since $x\in\omega$ and $x\neq 0$, we have $0\in x$. On the other hand, since $A^{<\omega}$ is a set of functions, $x$ is a function and hence is a set of doubletons, which contradicts $0\in x$.
Thus $A\cap A^{<\omega}=\emptyset$ holds.
\end{customthm}

\end{document}
