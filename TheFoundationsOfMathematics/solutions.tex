\documentclass[12pt]{article}
\usepackage{amsthm}
\usepackage{amsmath}
\usepackage{amssymb}
\usepackage{amsfonts}
\usepackage{multirow}

\theoremstyle{definition}
\newtheorem{innercustomthm}{Exercise}
\newenvironment{customthm}[1]
  {\renewcommand\theinnercustomthm{#1}\innercustomthm}
  {\endinnercustomthm}

\begin{document}
\title{
  The Foundations of Mathematics \\
  \large by Kenneth Kunen \\
  Exercise Solutions
  \author{Yoshihiro Kumazawa}
}
\maketitle

\begin{customthm}{I.2.1} Below is the table of truth values.
  \begin{center}
    \begin{tabular}{cc|cccc}
      & & \multicolumn{4}{c}{Axiom} \\
      & & 1 & 2 & 4 & 5 \\ \hline
      \multirow{7}{*}{Example}
      & 1 & T & T & F & T \\
      & 2 & T & F & T & T \\
      & 3 & T & T & F & F \\
      & 4 & T & T & F & F \\
      & 5 & F & T & F & T \\
      & 6 & T & T & F & F \\
      & 7 & T & T & F & F \\
    \end{tabular}
  \end{center}
\end{customthm}

\begin{customthm}{I.6.3} Example 2, 3 and 4 are the ones. Those are the ones that satisfy $\neg[\exists x~\mathrm{emp}(x)]$, which also satisfy Extensionality.
\end{customthm}

\begin{customthm}{I.6.11} Example 7 is the one. Example 1, 5 and 6 have pairwise unions. Example 2, 3 and 4 do not have empty sets hence cannot satisfy Comprehension.
\end{customthm}

\begin{customthm}{I.6.13} Example 1 is the one. The other ones do not satisfy $\forall x\neg\exists y[y\in x]$.
\end{customthm}

\begin{customthm}{I.6.15} Assume $\langle x',y'\rangle = \langle x,y\rangle$. Consider the following cases.
  \begin{enumerate}
  \item \underline{$x=y$}. Then it holds that $\langle x,y\rangle = \{\{x\}, \{x,y\}\} = \{\{x\}, \{x,x\}\} = \{\{x\}, \{x\}\} = \{\{x\}\}$. So $\{x',y'\}\in\langle x',y'\rangle$ has to be  identical to $\{x\}$ and we get $x'=y'=x=y$.
  \item \underline{$x\neq y$}. Then $\{x,y\}$ is a doubleton and it has to be identical to $\{x',y'\}$. Since $\{x',y'\}$ is a doubleton, the 2 singletons $\{x\}$ and $\{x'\}$ have to be identical, which leads to $x=x'$. Now we have $\{x,y\} = \{x,y'\}$. Considering $y\in \{x,y'\}$ and $y\neq x$, we get $y=y'$.
  \end{enumerate}
\end{customthm}

\begin{customthm}{I.6.17} $\langle 0,1\rangle = \{\{0\}, \{0,1\}\} = \{1,2\}$. $\langle 1,0\rangle = \{\{1\}, \{1,0\}\} = \{\{1\}, 2\}$.
\end{customthm}

\begin{customthm}{I.7.13} Let $<$ and $\prec$ be are strict total orderings of $S,T$ respectively and let $\triangleleft$ be the lexicographic product on $S\times T$. Check the following properties of $\triangleleft$ one by one.
  \begin{enumerate}
  \item\underline{Transitivity}. Let $\langle s,t\rangle, \langle s',t'\rangle, \langle s'',t''\rangle \in S\times T$ and assume $\langle s,t\rangle\triangleleft\langle s',t'\rangle$ and $\langle s',t'\rangle\triangleleft\langle s'',t''\rangle$. If $s<s'$ and $s'<s''$, then $s<s''$ holds from transitivity of $<$. If $s<s'$ and $s'=s''$, or, $s=s'$ and $s'<s''$, then it is obvious that $s<s''$.
  If $s=s'\wedge t\prec t'$ and $s'=s''\wedge t'\prec t''$, then $s=s''$ and $t\prec t''$ holds from transitivity of $=$ and $\prec$. Those exhaust all the possible cases, in each of which $\langle s,t\rangle\triangleleft\langle s'',t''\rangle$ holds.
  \item\underline{Irreflexivity}. Since $<$ and $\prec$ are irreflexive, we get $\langle s,t\rangle\triangleleft\langle s,t\rangle \leftrightarrow [s<s\vee[s=s\wedge t\prec t]] \leftrightarrow [\mathrm{F} \vee[\mathrm{T}\wedge \mathrm{F}]] \leftrightarrow \mathrm{F}$.
  \item\underline{Trichotomy}. Assume $\langle s,t\rangle\not\triangleleft\langle s',t'\rangle$. That is, $\neg[s<s'\vee[s=s'\wedge t\prec t']] \leftrightarrow [s'\leq s\wedge[s\neq s'\vee t'\preceq t]] \leftrightarrow [s'\leq s\wedge[s\neq s'\vee [s=s'\wedge t'\preceq t]]] \leftrightarrow [s'<s\vee[s=s'\wedge t'\preceq t]]$.
  Here trichotomies of $<$ and $\prec$ are used. Let us further assume $\langle s,t\rangle\neq\langle s',t'\rangle$, which is equivalent to $s=s'\rightarrow t\neq t'$. Combining this with the formula of $\langle s,t\rangle\not\triangleleft\langle s',t'\rangle$, we get $[s'<s\vee[s=s'\wedge t'\prec t]]$. This is nothing but $\langle s',t'\rangle\triangleleft\langle s,t\rangle$, which shows the trichotomy.
  \end{enumerate}
\end{customthm}

\begin{customthm}{I.7.15} Let $R$ be a set. Since  $[(x,y)]=[(x',y')]\rightarrow x=x'$, the formula $\phi(t,x)=\exists y[t=[(x, y)]$ has the property that $\forall t\in R\exists!y\phi(t,y)$. Then the set $\{x:\exists y[(x, y)\in R]\}$ exists by applying Replacement and Comprehension. The existence of the set $\{y:\exists x[(x, y)\in R]\}$ is proved in the same way.
\end{customthm}

\begin{customthm}{I.7.17} Let $(A,\triangleleft_1),(B,\triangleleft_2),(C,\triangleleft_3)$ be arbitrary relations and $F:A\rightarrow B,G:B\rightarrow C$ be isomorphisms. Reflexivity of $\cong$ is easily checked by letting $(A,\triangleleft_1)=(B,\triangleleft_2)$ and $F$ be the identity map $\mathrm{id}_A$. Symmetry follows from taking $F^{-1}:B\rightarrow A$. Transitivity is proved by considering $G\circ F:A\rightarrow C$.
Note that those $\mathrm{id}_A$, $F^{-1}$ and $G\circ F$ are all isomorphisms.
\end{customthm}

\begin{customthm}{I.7.21} Note that $R$ well-orders $A$ if and only if every non-empty subset of $A$ has a $R$-least element. If $R$ well-orders $A$, then it is obvious that $R$ well-orders every $X\subseteq A$ because every non-empty subset of $X$ is a non-empty subset of $A$ as well, which has a $R$-least element.
\end{customthm}

\begin{customthm}{I.7.23} Let $(S,<)$ and $(T,\prec)$ be well-ordered sets and $\triangleleft$ be their lexicographic product. Let $X$ be a non-empty subset of $S\times T$ and $Y = \mathrm{ran}(X)$. Then $Y$ is a non-empty subset of $S$ and it has a $<$-least element $s$. Let $Z=\{y\in T:(s,y)\in X\}$. It is a non-empty subset of $T$, which has a $\prec$-least element $t$. Now we have $(s,t)\in S\times T$ and it is the $\triangleleft$-least element of $X$.
\end{customthm}

\begin{customthm}{I.8.10} Let $X$ be a non-empty set of ordinals. $\cap X$ is well-ordered since it is a set of ordinals and $ON$ is well-ordered by $\in$. Let $\alpha\in\cap X$. Then $\alpha\subseteq\beta$ holds for every $\beta\in X$ by transitivity of $\beta$ and hence $\alpha\subseteq\cap X$ holds, which proves that $\cap X$ is a transitive set. Thus $\cap X$ is an ordinal.
Now let $\alpha=\min(X)$. $\alpha\subseteq\cap X$ follows from the fact that $\alpha\subseteq\beta(\leftrightarrow\alpha\leq\beta)$ for any $\beta\in X$. $\cap X\subseteq\alpha$ is obvious since every $\beta\in\cap X$ is also a member of $\alpha\in X$. Thus $\cap X=\min(X)$ holds. The fact that $\cup X$ is an ordinal with $\cup X=\sup(X)$ is proved in the same manner.
\end{customthm}

\begin{customthm}{I.8.11} Let $\alpha\in ON$. Every $\beta\in S(\alpha)=\alpha\cup\{\alpha\}$ is either in $\alpha$ or equal to $\alpha$. If $\beta\in\alpha$, $\beta\subseteq\alpha$ follows from transitivity of $\alpha$. If $\beta=\alpha$, then $\beta\subseteq \alpha$ is obvious.
Thus $\beta\subseteq\alpha\subseteq\alpha\cup\{\alpha\}=S(\alpha)$ holds in both cases, which proves $S(\alpha)$'s transitivity. To prove that $S(\alpha)$ is well-ordered, take a non-empty subset $X$ of $S(\alpha)$. If $X=\{\alpha\}$, then $\alpha$ is obviously the $\in$-least element of $X$. If not, $X\setminus\{\alpha\}$ is a non-empty subset of $\alpha$ and since $\alpha$ is well-ordered, $X\setminus\{\alpha\}$ has a $\in$-least element, which is also the least element of $X$.
Hence $S(\alpha)$ is well-ordered and $S(\alpha)\in ON$. The rest is immediate when we replace "$<$" by "$\in$" and "$\leq$" by "$\in$" or "$=$".
\end{customthm}

\begin{customthm}{I.8.13} Let $n$ be a natural number and $\beta\leq S(n)$. Then $\beta$ is either $\beta=S(n)$ or $\beta<S(n)\leftrightarrow\beta\leq n$. If $\beta=S(n)$, then $\beta$ is a successor of $n$. If $\beta\leq n$, then $\beta$ is either 0 or a successor since $n$ is a natural number. In either case, $\beta$ is either 0 or a successor and hence $S(n)$ is a natural number.
Now let $\gamma\in n$. Then for every $\alpha\leq\gamma$, $\alpha$ is either 0 or a successor since $n$ is a natural number and $\alpha\leq\gamma\leq n$. Therefore every element of $n$ turns out to be a natural number.
\end{customthm}

\begin{customthm}{I.8.22} Since every well-ordered set is isomorphic to a unique ordinal, it is sufficient to prove the following; for any ordinal $\alpha$ and $X\subseteq\alpha$, it holds that $\mathrm{type}(X;\in)\leq\alpha$. Now assume $\alpha\in ON$ and  $X\subseteq\alpha$, and let $f$ be the order isomorphism from $X$ to $\mathrm{type}(X)$. Since $f$ is an order-isomorphism, it holds that $f(\xi)=\{f(\mu):\mu\in X\wedge\mu<\xi\}$.
Now let $\Xi = \{\xi\in X:\xi<f(\xi)\}$. Assume $\Xi\neq\emptyset$ and let $\xi$ be the least element of $\Xi$. From $f(\xi)=\{f(\mu):\mu\in X\wedge\mu<\xi\}$ and $\xi\in\Xi$, we get $\xi=f(\mu)$ for some $\mu\in X$ such that $\mu<\xi$.
Since $\xi$ is the least element of $\Xi$, $\mu$ is not in $\Xi$ and hence $f(\mu)\leq\mu$ holds. Then we have $\xi\leq f(\mu)\leq\mu$, which is a contradiction. Thus $\Xi=\emptyset$ and hence $\mathrm{type}(X)\subseteq\alpha$ holds.
\end{customthm}

\begin{customthm}{I.8.23} Skipped for now.
\end{customthm}

\begin{customthm}{I.9.6} Let $x$ be a set.
  \begin{enumerate}
  \item $\mathrm{trcl}(x)\supseteq\cup^0x=x$.
  \item Let $y\in\mathrm{trcl}(x)$. Then there is $n\in\omega$ such that $y\in\cup^nx$. For such $n$ and $z\in y$, it holds that $z\in\{z:\exists y\in\cup^nx(z\in y)\}=\cup\cup^nx=\cup^{n+1}x\subseteq\mathrm{trcl}(x)$. Hence $\mathrm{trcl}(x)$ is transitive.
  \item Let $t$ be a transitive set. Note that $\cup t\subseteq t$ holds by definition of transitivity. Therefore it can be inductively shown that $\mathrm{trcl}(t)$ is a union of subsets of $t$, which is also a subset of $t$. On the other hand, if $y$ is a superset of $x$, then $\mathrm{trcl}(x)\subseteq\mathrm{trcl}(y)$ holds because we can show that $\cup^nx\subseteq\cup^ny$ by induction. Thus if $x\subseteq t$, it holds that $\mathrm{trcl}(x)\subseteq\mathrm{trcl}(t)\subseteq t$.
  \item For $y\in x$, one can easily show that $\cup^ny\subseteq\cup^{n+1}x$ by induction. Now let $y\in x$, $z\in\mathrm{trcl}(y)$ and $n\in\omega$ be one with $z\in\cup^ny$. Then we get $z\in\cup^{n+1}x\subseteq\mathrm{trcl}(x)$, which proves $\mathrm{trcl}(y)\subseteq\mathrm{trcl}(x)$.
  \item We show the following proposition $P(n)$: $z\in\cup^nx$ iff there is an $\in$-path from $z$ to $x$ of length $n+1$. Note that an $\in$-path from $z$ to $x$ of length $n\in\omega$ is a function $s$ such that $\mathrm{dom}(s)=n+1$, $s(0)=z$, $s(n)=x$, and $s(i)\in s(i+1)$ for all $i<n-1$.
  $P(0)$ is obvious by taking the function $s$ such that $s(0)=z=x$. Now assume $P(n)$ holds. Let $z\in\cup^{n+1}x=\{z:\exists y\in\cup^n x(z\in y)\}$ and $y\in\cup^n x$ be one such that $z\in y$. By induction hypothesis, we have a function $s$ such that $\mathrm{dom}(s)=n+1$, $s(0)=y$, $s(n)=x$, and $s(i)\in s(i+1)$ for all $i<n-1$.
  Now we extend this to a function $s'$ such that $\mathrm{dom}(s')=n+2$, $s(0)=x$, $s'(i+1)=s(i)$ for all $i\leq n$. Then $s'$ turns out to be an $\in$-path from $z$ to $x$ of length $n+1$. Therefore the 'only if' part of $P(n+1)$ is proved. The converse can be proved by considering a sub-path $s$ of an $\in$-path $s'$ from $z$ to $x$ of length $n+1$, where $s$ is the path from $s'(1)$ to $x$ of length $n$.
  Thus $P(n+1)$ holds, and so does $\forall n\in\omega.P(n)$.
  \end{enumerate}
\end{customthm}

\begin{customthm}{I.10.5} We can say that $A\times B=\{\{\{x\},\{x,y\}\}\in\mathcal{P(\mathcal{P(A\cup B)})}:x\in A,y\in B\}$ is a set, relying on the specific definition of pairs. $A/R=\{[x]_R\in\mathcal{P}(A):x\in A\}$, where $[x]_R=\{y\in A:yRx\}$, obviously forms a set.
\end{customthm}

\begin{customthm}{I.11.3} $\mathbb{R}\times\mathbb{R}\approx(0,1)\times(0,1)$ holds because $f(x,y)=(\sigma(x),\sigma(y))$, where $\sigma(x)=(1+\exp(x))^{-1}$, is a bijection from $\mathbb{R}\times\mathbb{R}$ to $(0,1)\times(0,1)$.
To prove $(0,1)\times(0,1)\preceq(0,1)$, we represent $x\in(0,1)$ by an never-ending infinite decimal representation $x=0.x_1x_2\cdots$. Note that any finite decimal $x=0.x_1x_2\cdots x_n$ can be represented that way, i.e. $x=0.x_1x_2\cdots(x_n-1)999\cdots$.
Now we define a function from $(0,1)\times(0,1)$ by $f(0.x1x2\cdots, 0.y1y2\cdots) = 0.x1y1x2y2\cdots$. This is an injection to $(0,1)$ because any number in $(0,1)$ is uniquely represented by the infinite decimal. Thus $(0,1)\times(0,1)\preceq(0,1)$ holds. $(0,1)\approx\mathbb{R}$ is clear by considering $\sigma^{-1}$.
\end{customthm}

\begin{customthm}{I.11.6} Skipped for now.
\end{customthm}

\begin{customthm}{I.11.12} Let $A,B,C,D$ be sets with $A\preceq B$, $C\preceq D$, let $i:A\rightarrow B$, $j:C\rightarrow D$ be injections. Note that if $B=\emptyset$ is allowed, there would be a counterexample of ${}^AC\preceq{}^BD$. For example, $A=B=C=\emptyset$ and $D=1$ would yield ${}^AC\approx 1\not\preceq 0={}^BD$. Now Let us further assume that $B\neq\emptyset$ and fix $b\in B$.
We define a function $F:{}^AC\rightarrow{}^BD$ by the following.
\[F(f)(d)=
  \begin{cases}
    i(f(c)) & \mbox{if } d=j(c) \\
    b & \mbox{if } d\not\in\mathrm{ran}(j)
  \end{cases}
\]
One can easily check that this $F$ is injective and hence ${}^AC\preceq{}^BD$ holds. The rest of the exercise, namely if $2\prec C$ then $A\prec \mathcal{P}(A)\preceq{}^AC$, follows from $\mathcal{P}(A)\approx 2^A$ and the above by letting $B=A, C=2, D=C$ in ${}^AC\preceq{}^BD$. Note that $\mathcal{P}(A)\preceq{}^AC$ holds even when $A=\emptyset$, where both sides are of cardinality $1$.
\end{customthm}

\begin{customthm}{I.11.15} (1) is obvious from Exercise I.8.22. (2) can be reduced to (1) if we replace $B$ by $i(B)$ where $i:B\rightarrow\alpha$ is an injection. To prove (3), let $\alpha,\beta,\gamma$ be ordinals with $\alpha\leq\beta\leq\gamma$, $\alpha\approx\gamma$ and let $f:\gamma\rightarrow\alpha$ be a bijection.
Since $\alpha\subseteq\beta$ holds, $f$ is a function from $\gamma$ to $\beta$ as well, and since $\beta\subseteq\gamma$, we have $f\upharpoonright\beta:\beta\rightarrow\alpha$. Both of those functions are injective and hence we get $\gamma\preceq\beta\preceq\alpha$. Combining these with $\alpha\leq\beta\leq\gamma$ and Schröder-Bernstein Theorem, we get $\alpha\approx\beta\approx\gamma$.
\end{customthm}

\end{document}
