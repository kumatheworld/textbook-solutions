\documentclass[12pt]{article}
\usepackage{../kuma}

\title{
  \vspace{-2cm}
  The Foundations of Mathematics \\
  \large by Kenneth Kunen \\
  Exercise Solutions
}

\begin{document}
\maketitle

\begin{customthm}{I.2.1}
  Below is the table of the truth values.
  \begin{center}
    \begin{tabular}{cc|cccc}
      & & \multicolumn{4}{c}{Axiom} \\
      & & 1 & 2 & 4 & 5 \\ \hline
      \multirow{7}{*}{Example}
      & 1 & T & T & F & T \\
      & 2 & T & F & T & T \\
      & 3 & T & T & F & F \\
      & 4 & T & T & F & F \\
      & 5 & F & T & F & T \\
      & 6 & T & T & F & F \\
      & 7 & T & T & F & F \\
    \end{tabular}
  \end{center}
\end{customthm}

\begin{customthm}{I.6.3}
  Example 2, 3 and 4 are the ones.
\end{customthm}

\begin{customthm}{I.6.11}
  Example 7 is the one. Example 1, 5 and 6 have pairwise unions. Example 2, 3 and 4 do not have empty sets hence cannot satisfy Comprehension.
\end{customthm}

\begin{customthm}{I.6.13}
  Example 1 is the one. The other ones do not satisfy $\forall x\,\neg\exists y\,[y\in x]$.
\end{customthm}

\begin{customthm}{I.6.15}
  Assume $\langle x',y'\rangle = \langle x,y\rangle$. Consider the following cases.
  \begin{enumerate}
    \item\underline{$x=y$}. Then it holds that $\langle x,y\rangle = \{\{x\}, \{x,y\}\} = \{\{x\}, \{x,x\}\} = \{\{x\}, \{x\}\} = \{\{x\}\}$. So $\{x',y'\}\in\langle x',y'\rangle$ has to be  identical to $\{x\}$ and we get $x'=y'=x=y$.
    \item\underline{$x\neq y$}. Then $\{x,y\}$ is a doubleton and it has to be identical to $\{x',y'\}$. Since $\{x',y'\}$ is a doubleton, the 2 singletons $\{x\}$ and $\{x'\}$ have to be identical, which leads to $x=x'$. Now we have $\{x,y\} = \{x,y'\}$. Considering $y\in \{x,y'\}$ and $y\neq x$, we get $y=y'$.
  \end{enumerate}
\end{customthm}

\begin{customthm}{I.6.17}
  $\langle 0,1\rangle = \{\{0\}, \{0,1\}\} = \{1,2\}$. $\langle 1,0\rangle = \{\{1\}, \{1,0\}\} = \{\{1\}, 2\}$.
\end{customthm}

\begin{customthm}{I.7.13}
  Let $<,\prec$ be strict total orderings of $S,T$ respectively and let $\triangleleft$ be the lexicographic product on $S\times T$. Check the following properties of $\triangleleft$ one by one.
  \begin{enumerate}
    \item\underline{Transitivity}. Let $\langle s,t\rangle, \langle s',t'\rangle, \langle s'',t''\rangle \in S\times T$ and assume $\langle s,t\rangle\triangleleft\langle s',t'\rangle$ and $\langle s',t'\rangle\triangleleft\langle s'',t''\rangle$. If $s<s'$ and $s'<s''$, then $s<s''$ holds from transitivity of $<$. If $s<s'$ and $s'=s''$, or, $s=s'$ and $s'<s''$, then it is obvious that $s<s''$. If $s=s'\wedge t\prec t'$ and $s'=s''\wedge t'\prec t''$, then $s=s''$ and $t\prec t''$ holds from transitivity of $=$ and $\prec$. Those exhaust all the possible cases, in each of which $\langle s,t\rangle\triangleleft\langle s'',t''\rangle$ holds.
    \item\underline{Irreflexivity}. Since $<$ and $\prec$ are irreflexive, we get $\langle s,t\rangle\triangleleft\langle s,t\rangle \leftrightarrow [s<s\vee[s=s\wedge t\prec t]] \leftrightarrow [\mathrm{F} \vee[\mathrm{T}\wedge \mathrm{F}]] \leftrightarrow \mathrm{F}$.
    \item\underline{Trichotomy}. Assume $\langle s,t\rangle\not\triangleleft\langle s',t'\rangle$. That is, $\neg[s<s'\vee[s=s'\wedge t\prec t']] \leftrightarrow [s'\leq s\wedge[s\neq s'\vee t'\preccurlyeq t]] \leftrightarrow [s'\leq s\wedge[s\neq s'\vee [s=s'\wedge t'\preccurlyeq t]]] \leftrightarrow [s'<s\vee[s=s'\wedge t'\preccurlyeq t]]$. Here trichotomies of $<$ and $\prec$ are used. Let us further assume $\langle s,t\rangle\neq\langle s',t'\rangle$, which is equivalent to $s=s'\rightarrow t\neq t'$. Combining this with the formula of $\langle s,t\rangle\not\triangleleft\langle s',t'\rangle$, we get $[s'<s\vee[s=s'\wedge t'\prec t]]$. This is nothing but $\langle s',t'\rangle\triangleleft\langle s,t\rangle$, which shows the trichotomy.
  \end{enumerate}
\end{customthm}

\begin{customthm}{I.7.15}
  Let $R$ be a set. Since  $[(x,y)]=[(x',y')]\rightarrow x=x'$, the formula $\varphi(t,x)=\exists y\,[t=[(x, y)]$ has the property that $\forall t\in R\,\exists!y\,\varphi(t,y)$. Then the set $\{x:\exists y\,[(x, y)\in R]\}$ exists by applying Replacement and Comprehension. The existence of the set $\{y:\exists x\,[(x, y)\in R]\}$ is proved in the same way.
\end{customthm}

\begin{customthm}{I.7.17}
  Let $(A,\triangleleft_1),(B,\triangleleft_2),(C,\triangleleft_3)$ be arbitrary relations and $F:A\rightarrow B,G:B\rightarrow C$ be isomorphisms. Reflexivity of $\cong$ is easily checked by letting $(A,\triangleleft_1)=(B,\triangleleft_2)$ and $F$ be the identity map $\mathrm{id}_A$. Symmetry follows from taking $F^{-1}:B\rightarrow A$. Transitivity is proved by considering $G\circ F:A\rightarrow C$. Note that those $\mathrm{id}_A$, $F^{-1}$ and $G\circ F$ are all isomorphisms.
\end{customthm}

\begin{customthm}{I.7.21}
  Note that $R$ well-orders $A$ if and only if every non-empty subset of $A$ has a $R$-least element. If $R$ well-orders $A$, then it is obvious that $R$ well-orders every $X\subseteq A$ because every non-empty subset of $X$ is a non-empty subset of $A$ as well, which has a $R$-least element.
\end{customthm}

\begin{customthm}{I.7.23}
  Let $(S,<)$ and $(T,\prec)$ be well-ordered sets and $\triangleleft$ be their lexicographic product. Let $X$ be a non-empty subset of $S\times T$ and $Y = \ran(X)$. Then $Y$ is a non-empty subset of $S$ and it has a $<$-least element $s$. Let $Z=\{y\in T:(s,y)\in X\}$. It is a non-empty subset of $T$, which has a $\prec$-least element $t$. Now we have $(s,t)\in S\times T$ and it is the $\triangleleft$-least element of $X$.
\end{customthm}

\begin{customthm}{I.8.10}
  Let $X$ be a non-empty set of ordinals. $\bigcap X$ is well-ordered since it is a set of ordinals and $ON$ is well-ordered by $\in$. Let $\alpha\in\bigcap X$. Then $\alpha\subseteq\beta$ holds for every $\beta\in X$ by transitivity of $\beta$ and hence $\alpha\subseteq\bigcap X$ holds, which proves that $\bigcap X$ is a transitive set. Thus $\bigcap X$ is an ordinal. Now let $\alpha=\min(X)$. $\alpha\subseteq\bigcap X$ follows from the fact that $\alpha\subseteq\beta(\leftrightarrow\alpha\leq\beta)$ for any $\beta\in X$. $\bigcap X\subseteq\alpha$ is obvious since every $\beta\in\bigcap X$ is also a member of $\alpha\in X$. Thus $\bigcap X=\min(X)$ holds. The fact that $\bigcup X$ is an ordinal with $\bigcup X=\sup(X)$ is proved in the same manner.
\end{customthm}

\begin{customthm}{I.8.11}
  Let $\alpha\in ON$. Every $\beta\in S(\alpha)=\alpha\cup\{\alpha\}$ is either in $\alpha$ or equal to $\alpha$. If $\beta\in\alpha$, $\beta\subseteq\alpha$ follows from transitivity of $\alpha$. If $\beta=\alpha$, then $\beta\subseteq \alpha$ is obvious. Thus $\beta\subseteq\alpha\subseteq\alpha\cup\{\alpha\}=S(\alpha)$ holds in both cases, which proves $S(\alpha)$'s transitivity. To prove that $S(\alpha)$ is well-ordered, take a non-empty subset $X$ of $S(\alpha)$. If $X=\{\alpha\}$, then $\alpha$ is obviously the $\in$-least element of $X$. If not, $X\setminus\{\alpha\}$ is a non-empty subset of $\alpha$ and since $\alpha$ is well-ordered, $X\setminus\{\alpha\}$ has a $\in$-least element, which is also the least element of $X$. Hence $S(\alpha)$ is well-ordered and $S(\alpha)\in ON$. The rest is immediate when we replace "$<$" by "$\in$" and "$\leq$" by "$\in$" or "$=$".
\end{customthm}

\begin{customthm}{I.8.13}
  Let $n$ be a natural number and $\beta\leq S(n)$. Then $\beta$ is either $\beta=S(n)$ or $\beta<S(n)\leftrightarrow\beta\leq n$. If $\beta=S(n)$, then $\beta$ is a successor of $n$. If $\beta\leq n$, then $\beta$ is either 0 or a successor since $n$ is a natural number. In either case, $\beta$ is either 0 or a successor and hence $S(n)$ is a natural number. Now let $\gamma\in n$. Then for every $\alpha\leq\gamma$, $\alpha$ is either 0 or a successor since $n$ is a natural number and $\alpha\leq\gamma\leq n$. Therefore every element of $n$ turns out to be a natural number.
\end{customthm}

\begin{customthm}{I.8.22}
  Since every well-ordered set is isomorphic to a unique ordinal, it is sufficient to prove the following; for any ordinal $\alpha$ and $X\subseteq\alpha$, it holds that $\type(X;\in)\leq\alpha$. Now assume $\alpha\in ON$ and  $X\subseteq\alpha$, and let $f$ be the order isomorphism from $X$ to $\type(X)$. Since $f$ is an order-isomorphism, it holds that $f(\xi)=\{f(\mu):\mu\in X\wedge\mu<\xi\}$. Now let $\Xi = \{\xi\in X:\xi<f(\xi)\}$. Assume $\Xi\neq\emptyset$ and let $\xi$ be the least element of $\Xi$. From $f(\xi)=\{f(\mu):\mu\in X\wedge\mu<\xi\}$ and $\xi\in\Xi$, we get $\xi=f(\mu)$ for some $\mu\in X$ such that $\mu<\xi$. Since $\xi$ is the least element of $\Xi$, $\mu$ is not in $\Xi$ and hence $f(\mu)\leq\mu$ holds. Then we have $\xi\leq f(\mu)\leq\mu$, which is a contradiction. Thus $\Xi=\emptyset$ and hence $\type(X)\subseteq\alpha$ holds.
\end{customthm}

\begin{customthm}{I.8.23}
  We work in ZF to prove the following. Let $Z=\{\{\},\{\{\}\},\ldots\}$.
  \begin{itemize}
    \item\underline{$(1)\rightarrow(2)$}. Assume the Axiom of Infinity. We have $\omega$. Let $\varphi(x,y)=x\in\omega\wedge\exists f[\dom(f)=x+1\wedge f(x)=y\wedge f(0)=\emptyset\wedge\forall n\in x[f(n+1)=\{f(n)\}]]$. By Replacement, we get $Z$.
    \item\underline{$(2)\rightarrow(1)$}. Assume the Axiom des Unendlichen of Zermelo. We have $Z$. Let $\psi(y,x)=y\in Z\wedge[\chi(y,\dom(g))\wedge g(y)=x\wedge g(\emptyset)=\emptyset\wedge\forall w\in\dom(g)\setminus\{y\}[g(\{w\})=g(w)\cup\{w\}]]$, where $\chi(y, S)=S\subseteq Z\wedge y\in S\wedge\{y\}\not\in S\wedge\forall v\in S\,\forall u\in v[u\in S]$, which means that $S$ is the "initial segment" of $Z$ whose "maximum" is $y$. By Replacement, we get $\omega$.
  \end{itemize}
\end{customthm}

\begin{customthm}{I.9.6}
  Let $x$ be a set.
  \begin{enumerate}
    \item $\trcl(x)\supseteq\bigcup^0x=x$.
    \item Let $y\in\trcl(x)$. Then there is $n\in\omega$ such that $y\in\bigcup^nx$. For such $n$ and $z\in y$, it holds that $z\in\{z:\exists y\in\bigcup^nx\,(z\in y)\}=\bigcup\bigcup^nx=\bigcup^{n+1}x\subseteq\trcl(x)$. Hence $\trcl(x)$ is transitive.
    \item Let $t$ be a transitive set. Note that $\bigcup t\subseteq t$ holds by the definition of transitivity. Therefore it can be inductively shown that $\trcl(t)$ is a union of subsets of $t$, which is also a subset of $t$. On the other hand, if $y$ is a superset of $x$, then $\trcl(x)\subseteq\trcl(y)$ holds because we can show that $\bigcup^nx\subseteq\bigcup^ny$ by induction. Thus if $x\subseteq t$, it holds that $\trcl(x)\subseteq\trcl(t)\subseteq t$.
    \item For $y\in x$, one can easily show that $\bigcup^ny\subseteq\bigcup^{n+1}x$ by induction. Now let $y\in x$, $z\in\trcl(y)$ and $n\in\omega$ be one with $z\in\bigcup^ny$. Then we get $z\in\bigcup^{n+1}x\subseteq\trcl(x)$, which proves $\trcl(y)\subseteq\trcl(x)$.
    \item We show the following proposition $P(n)$: $z\in\bigcup^nx$ iff there is an $\in$-path from $z$ to $x$ of length $n+1$. Note that an $\in$-path from $z$ to $x$ of length $n\in\omega$ is a function $s$ such that $\dom(s)=n+1$, $s(0)=z$, $s(n)=x$, and $s(i)\in s(i+1)$ for all $i<n-1$. $P(0)$ is obvious by taking the function $s$ such that $s(0)=z=x$. Now assume $P(n)$ holds. Let $z\in\bigcup^{n+1}x=\{z:\exists y\in\bigcup^nx\,(z\in y)\}$ and $y\in\bigcup^n x$ be one such that $z\in y$. By inductive hypothesis, we have a function $s$ such that $\dom(s)=n+1$, $s(0)=y$, $s(n)=x$, and $s(i)\in s(i+1)$ for all $i<n-1$. Now we extend this to a function $s'$ such that $\dom(s')=n+2$, $s(0)=x$, $s'(i+1)=s(i)$ for all $i\leq n$. Then $s'$ turns out to be an $\in$-path from $z$ to $x$ of length $n+1$. Therefore the 'only if' part of $P(n+1)$ is proved. The converse can be proved by considering a sub-path $s$ of an $\in$-path $s'$ from $z$ to $x$ of length $n+1$, where $s$ is the path from $s'(1)$ to $x$ of length $n$. Thus $P(n+1)$ holds, and so does $\forall n\in\omega\,P(n)$.
  \end{enumerate}
\end{customthm}

\begin{customthm}{I.10.5}
  We can say that $A\times B=\{\{\{x\},\{x,y\}\}\in\mathcal{P(\mathcal{P(A\cup B)})}:x\in A,y\in B\}$ is a set, relying on the specific definition of pairs. $A/R=\{[x]_R\in\mathcal{P}(A):x\in A\}$, where $[x]_R=\{y\in A:yRx\}$, obviously forms a set.
\end{customthm}

\begin{customthm}{I.11.3}
  $\mathbb{R}\times\mathbb{R}\approx(0,1)\times(0,1)$ holds because $f(x,y)=(\sigma(x),\sigma(y))$, where $\sigma(x)=(1+\exp(x))^{-1}$, is a bijection from $\mathbb{R}\times\mathbb{R}$ to $(0,1)\times(0,1)$. To prove $(0,1)\times(0,1)\preccurlyeq(0,1)$, we represent $x\in(0,1)$ by an never-ending infinite decimal representation $x=0.x_1x_2\cdots$. Note that any finite decimal $x=0.x_1x_2\cdots x_n$ can be represented that way, i.e. $x=0.x_1x_2\cdots(x_n-1)999\cdots$. Now we define a function from $(0,1)\times(0,1)$ by $f(0.x1x2\cdots, 0.y1y2\cdots) = 0.x1y1x2y2\cdots$. This is an injection to $(0,1)$ because any number in $(0,1)$ is uniquely represented by the infinite decimal. Thus $(0,1)\times(0,1)\preccurlyeq(0,1)$ holds. $(0,1)\approx\mathbb{R}$ is clear by considering $\sigma^{-1}$.
\end{customthm}

\begin{customthm}{I.11.6}
  Skipped for now.
\end{customthm}

\begin{customthm}{I.11.12}
  Let $A,B,C,D$ be sets with $A\preccurlyeq B$, $C\preccurlyeq D$, let $i:A\rightarrow B$, $j:C\rightarrow D$ be injections. Note that if $B=\emptyset$ is allowed, there would be a counterexample of ${}^AC\preccurlyeq{}^BD$. For example, $A=B=C=\emptyset$ and $D=1$ would yield ${}^AC\approx 1\not\preccurlyeq 0={}^BD$. Now Let us further assume that $B\neq\emptyset$ and fix $b\in B$. We define a function $F:{}^AC\rightarrow{}^BD$ by the following.
  \[F(f)(d)=
    \begin{cases}
      i(f(c)) & \text{if } d=j(c) \\
      b & \text{if } d\not\in\ran(j)
    \end{cases}.
  \]
  One can easily check that this $F$ is injective and hence ${}^AC\preccurlyeq{}^BD$ holds. The rest of the exercise, namely if $2\prec C$ then $A\prec \mathcal{P}(A)\preccurlyeq{}^AC$, follows from $\mathcal{P}(A)\approx 2^A$ and the above by letting $B=A, C=2, D=C$ in ${}^AC\preccurlyeq{}^BD$. Note that $\mathcal{P}(A)\preccurlyeq{}^AC$ holds even when $A=\emptyset$, where both sides are of cardinality $1$.
\end{customthm}

\begin{customthm}{I.11.15}
  (1) is obvious from Exercise I.8.22. (2) can be reduced to (1) if we replace $B$ by $i(B)$ where $i:B\rightarrow\alpha$ is an injection. To prove (3), let $\alpha,\beta,\gamma$ be ordinals with $\alpha\leq\beta\leq\gamma$, $\alpha\approx\gamma$ and let $f:\gamma\rightarrow\alpha$ be a bijection. Since $\alpha\subseteq\beta$ holds, $f$ is a function from $\gamma$ to $\beta$ as well, and since $\beta\subseteq\gamma$, we have $f\upharpoonright\beta:\beta\rightarrow\alpha$. Both of those functions are injective and hence we get $\gamma\preccurlyeq\beta\preccurlyeq\alpha$. Combining these with $\alpha\leq\beta\leq\gamma$ and Schröder-Bernstein Theorem, we get $\alpha\approx\beta\approx\gamma$.
\end{customthm}

\begin{customthm}{I.11.19}
  The 'only if' part is obvious since any order isomorphism is a bijection. To prove the 'if' part, let $A$ be a set, $\alpha$ be an ordinal and $f:A\rightarrow\alpha$ be a bijection. Then the relation $R=\{\langle x,y\rangle\in A\times A:f(x)\leq f(y)\}$ well-orders $A$.
\end{customthm}

\begin{customthm}{I.11.21}
  Let $A$ be a well-orderable set and $g:A\rightarrow|A|$ be a bijection. Let $B$ be a set and assume there is a surjection $f:A\rightarrow B$.  Since $f$ is surjective, we can define an injection $i:B\rightarrow|A|$ by $i(y)=\min\{g(x)\in|A|:x\in A\wedge y=f(x)\}$. Then by Exercise I.11.15(2) and Exercise I.11.19, we can say that $B$ is well-orderable and $|B|\leq|A|$.
\end{customthm}

\begin{customthm}{I.11.22}
  The 'if' part is obvious from Exercise I.11.21. To prove the converse, let $\kappa$ be a cardinal, $B$ be a non-empty set with $B\preccurlyeq\kappa$. For a fixed $b\in B$ and an injection $i:B\rightarrow\kappa$, we can define a surjection $f:\kappa\rightarrow B$ as follows.
  \[
    f(\alpha)=
    \begin{cases}
      x & \text{if } \alpha=i(x) \\
      b & \text{if } \alpha\not\in\ran(i)
    \end{cases}.
  \]
\end{customthm}

\begin{customthm}{I.11.23}
  Let $f:A\rightarrow|A|$, $g:B\rightarrow|B|$ be bijections.
  \begin{enumerate}
    \item\underline{$|A|$ is a cardinal}. Assume not and let $\xi$ be an ordinal such that $\xi<|A|$ and $\xi\approx|A|$. Then $A\approx\xi$ holds by $A\approx|A|$, contradicting the minimality of $|A|$.
    \item\underline{$A\preccurlyeq B$ iff $|A|\leq|B|$}. Assume $A\preccurlyeq B$ and let $i:A\rightarrow B$ be an injection. Then $g\circ i:A\rightarrow|B|$ is an injection, and by Exercise I.11.15(2), there exists an ordinal $\delta\leq|B|$ with a bijection $h:A\rightarrow\delta$. For such $\delta$ and $h$, we can define an injection $h\circ f^{-1}:|A|\rightarrow|B|$. Hence we have $|A|\leq|B|$. The converse is obvious from an injection $g^{-1}\circ f:A\rightarrow B$.
    \item\underline{$A\approx B$ iff $|A|=|B|$}. This is obvious from (2) since both $A\preccurlyeq B\wedge B\preccurlyeq A\leftrightarrow A\approx B$ (Schröder-Bernstein Theorem) and $|A|\leq|B|\wedge|B|\leq|A|\leftrightarrow|A|=|B|$ hold.
    \item\underline{$A\prec B$ iff $|A|<|B|$}. This is obvious from (2), (3) and the definitions of $\prec$ and $<$.
  \end{enumerate}
\end{customthm}

\begin{customthm}{I.11.24}
  The 'if' part is obvious since $A\preccurlyeq|A|$ holds. To prove the converse, let $A$ be a finite set and let $n\in\omega$ be one with $A\preccurlyeq n$. Then by Exercise I.11.15(2), $A$ is well-orderable and there exists $m\leq n$ such that $A\approx m$. Now it is obvious that $|A|\leq m\leq n<\omega$.
\end{customthm}

\begin{customthm}{I.11.25}
  We prove by induction the following proposition $P(n)$: $|A\cup B|\leq m+n$, equality holding iff $A\cap B=\emptyset$. $P(0)$ is obvious since if $n=0$, then we have $B=\emptyset$, $|A\cup B|=|A|=m$ and $A\cap B=\emptyset$. Now assume $P(n)$ and let $B'=B\cup\{b\}$ with $b\not\in B$, $|B|=n$. If $b\in A$, we have $|A\cup B'|=|A\cup B|\leq m+n<m+n+1$ by inductive hypothesis. Note that $A\cap B'\neq\emptyset$ since $A$ and $B'$ has $b$ in common, in which case the equality does not hold. If $b\not\in A$, we can define an injection $i':A\cup B'\rightarrow m+n+1$ by $i\cup\{\langle b',m+n\rangle\}$, where $i$ is an induction $i:A\cup B\rightarrow m+n$. This is a bijection if $i$ is a bijection, which is by inductive hypothesis equivalent to $A\cap B=\emptyset$, which is also equivalent to $A\cap B'=\emptyset$. Thus $P(n+1)$ holds under $P(n)$ and we have $\forall n\in\omega\,P(n)$. One can also prove $|A\times B|=m\cdot n$ by induction on $n$, using the fact that $A\times (B\sqcup\{b\})\approx(A\times B)\sqcup(A\times\{b\})$, where $\sqcup$ denotes the disjoint union operator.
\end{customthm}

\begin{customthm}{I.11.28}
  Let $\alpha$ be an ordinal. By Exercise I.11.23, we have $\beta\not\preccurlyeq\alpha\leftrightarrow\beta\not\leq\alpha\leftrightarrow\alpha<\beta$ for any ordinal $\beta$. Thus $\alpha^+$, which is by definition the least cardinal $\kappa$ such that $\kappa\not\preccurlyeq\alpha$, is the least cardinal greater than $\alpha$. Now note that $\alpha+1$ is the least ordinal greater than $\alpha$ (See Exercise I.8.11). Therefore $\alpha^+\geq\alpha+1$ holds. When $\alpha\in\omega$, by Exercise I.8.13 and Theorem I.11.17(2), $\alpha+1\in\omega$ turns out to be a cardinal and hence $\alpha^+=\alpha+1$. When $\alpha\geq\omega$, $\alpha+1$ is not a cardinal since we have an injection $i:\alpha+1\rightarrow\alpha$ defined as follows.
  \[
    i(x)=
    \begin{cases}
      0 & \text{if } x=\alpha \\
      x+1 & \text{if } x\in\omega \\
      x & \text{otherwise}
    \end{cases}.
  \]
  Thus $\alpha^+>\alpha+1$ holds.
\end{customthm}

\begin{customthm}{I.11.29}
  Let $A$ be a set and $W,\beta,\kappa$ be the ones occurring in the proof of Theorem I.11.26. Let $\alpha$ be an ordinal such that $\alpha<\beta$. Then there is a pair $(X,R)$ such that $\alpha<\type(X;R)+1$, which is equivalent to $\alpha\leq\type(X;R)$. Therefore $\alpha\preccurlyeq\type(X;R)\approx X\preccurlyeq A$ holds. If $\alpha\not\prec\beta$, we would have $\beta\preccurlyeq\alpha\preccurlyeq A$ and that contradicts $\beta\not\preccurlyeq A$. Hence $\alpha\prec\beta$ holds and $\beta$ turns out to be a cardinal. $\beta=\aleph(A)$ is obvious since $\alpha<\beta$ implies $\alpha\preccurlyeq A$ from the above, the contraposition of which is nothing but a proof of $\beta$'s minimality.
\end{customthm}

\begin{customthm}{I.11.30}
  Let $\xi,\zeta$ be ordinals. We prove $\xi<\zeta\rightarrow\aleph_\xi<\aleph_\zeta$ by induction on $\zeta$. If $\zeta=\xi+1$, which is the least ordinal greater than $\xi$, then $\aleph_\zeta=(\aleph_\xi)^+$ is greater than $\aleph_\xi$ by the definition of $\aleph$. If $\zeta=\eta+1$ and $\eta>\xi$, assuming the inductive hypothesis $\aleph_\eta>\aleph_\xi$, we get $\aleph_\zeta=(\aleph_\eta)^+>(\aleph_\xi)^+>\aleph_\xi$. If $\zeta$ is a limit ordinal and $\aleph_\xi<\aleph_\eta$ for all $\xi<\eta<\zeta$, then $\aleph_\zeta=\sup\{\aleph_\eta:\eta<\zeta\}\geq\aleph_{\xi+1}>\aleph_\xi$. Hence $\xi<\zeta\rightarrow\aleph_\xi<\aleph_\zeta$ holds.

  Now we prove that $\kappa$ is an infinite cardinal iff $\kappa=\aleph_\xi$ for some $\xi$. The 'if' part is obvious from the definition of $\aleph$ and Theorem I.11.17(3). To prove the converse by contradiction, assume there is an infinite cardinal $\kappa$ such that $\forall\xi\in ON\,[\kappa\neq\aleph_\xi]$. Let $\kappa$ be the least cardinal such that $\forall\xi\in ON\,[\kappa\neq\aleph_\xi]$ and let $\xi=\{\eta\in ON:\aleph_\eta<\kappa\}$. Note that $\xi$ is an initial segment of $ON$ by the above. Furthermore, $\xi$ forms a set. Otherwise, $\xi=ON$ and $\kappa$ would contain all $\aleph_\eta$, which by Replacement implies $ON$ is a set. Now we prove $\kappa=\aleph_\xi$. One can easily prove that $\aleph_\xi\leq\kappa$ by splitting $\xi$ into $0$, a successor or a limit ordinal. On the other hand, $\aleph_\xi\not<\kappa$ holds because $\aleph_\xi<\kappa$ would imply $\xi\in\xi$. Thus $\kappa=\aleph_\xi$ holds, which contradicts with the definition of $\kappa$. Hence any infinite cardinal is identical to $\aleph_\xi$ for some $\xi$.
\end{customthm}

\begin{customthm}{I.11.33}
  We first prove $\aleph_\gamma\leq\gamma$. Since $\gamma$ is a limit ordinal, $\aleph_\gamma=\sup\{\aleph_\xi:\xi<\gamma\}$ holds and $\aleph_\gamma\leq\gamma$ is equivalent to $\forall\xi<\gamma\,[\aleph_\xi\leq\gamma]$. Let $\xi<\gamma$ and $n\in\omega$ be one such that $\xi<\delta_n$. Then we have $\aleph_\xi<\aleph_{\delta_n}=\delta_{n+1}<\gamma$. Hence $\aleph_\gamma\leq\gamma$ holds. $\gamma\leq\aleph_\gamma$ is clear from $\forall n\in\omega\,[\delta_n<\aleph_{\delta_n}<\aleph_\gamma]$, and we get $\aleph_\gamma=\gamma$.

  Let $\xi$ be an ordinal such that $\aleph_\xi=\xi$. Since $\aleph$ is monotonic by Exercise I.11.32, we get $\forall n\in\omega\,[\delta_n\leq\xi]$ by repeatedly applying $\aleph$ to the both side of $0\leq\xi$.  Hence $\gamma\leq\xi$ holds and $\gamma$ turns out to be the least fixed point of $\aleph$.
\end{customthm}

\begin{customthm}{I.11.34}
  Let $\kappa=\max\{|\alpha|,|\beta|\}$. We first note that $\kappa=|\max\{\alpha,\beta\}|$. For $|\alpha+\beta|$ and $|\alpha\cdot\beta|$, if $\alpha\leq\beta$, it holds that $|\beta|=\kappa\geq\aleph_0$ and we have $|\beta|\leq|\alpha+\beta|\leq|\beta+\beta|=|\beta|$ and $|\beta|\leq|\alpha\cdot\beta|\leq|\beta\cdot\beta|=|\beta\times\beta|=|\beta|$. One can prove in a similar way that $|\alpha+\beta|=|\alpha\cdot\beta|=|\alpha|$ holds when $\alpha\geq\beta$. Thus $|\alpha+\beta|=|\alpha\cdot\beta|=\kappa$ holds.

  For $|\alpha^\beta|$, we first show that $|\alpha^\beta|\leq\kappa$. To prove it, we fix a $\delta<\kappa^+$ and show that $|\alpha^\beta|\leq\kappa$ for $\alpha,\beta<\delta$. In order to do that, we fix an injection $i:\delta\rightarrow\kappa$, another injection $j:\kappa\times\kappa\rightarrow\kappa$ and define an injection $f_\beta:\alpha^\beta\rightarrow\kappa$ by recursion on $\beta<\delta$. If $\beta=0$, then $\alpha^\beta=1$ holds and there is an obvious injection $f_0:1\rightarrow\kappa$ defined by $f_0(0)=0$. If $\beta=\xi+1$ and we have $f_\xi$, then there is an injection $g:\alpha\times\alpha^\xi\rightarrow\kappa$ defined by $g(x,y)=j(x,f_\xi(y))$ and we define $f_\beta$ by composing $g$ and the isomorphism from $\alpha^\beta=\type(\alpha\times\alpha^\xi)$ to $\alpha\times\alpha^\xi$. If $\beta$ is a limit ordinal, we define $f_\beta$ by $f_\beta(x)=j(i(\xi),f_\xi(x))$ where $\xi<\beta$ is the least ordinal such that $x<\alpha^\xi$. Thus we have shown $|\alpha^\beta|\leq\kappa$ for $\alpha,\beta<\delta$. Since $\delta<\kappa^+$ is arbitrary, we have $|\alpha^\beta|\leq\kappa$ whenever $\max\{|\alpha|,|\beta|\}=\kappa$.

  Now we prove $|\alpha^\beta|\geq\kappa$. If $\alpha\geq\beta$, we have $|\alpha^\beta|\geq|\alpha|=\kappa$. If $\alpha\leq\beta$, we have $|\alpha^\beta|\geq|2^\beta|\geq|\beta|=\kappa$, where one can easily prove $\beta\leq2^\beta$ by recursively defining injections. From both $|\alpha^\beta|\leq\kappa$ and $|\alpha^\beta|\geq\kappa$, we conclude that $|\alpha^\beta|=\kappa$.
\end{customthm}

\begin{customthm}{I.11.35}
  Let $W=\{(X,R)\in\mathcal{P}(\omega)\times\mathcal{P}(\omega\times\omega):R\text{ well-orders }X\}$ and $B=W/\cong$, where $\cong$ is the isomorphism relation. We define a relation $\triangleleft$ on $B$ by $[(X,R)]\triangleleft[(X',R')]$ iff $\type(X;R)<\type(X';R')$. It is easy to see that $\triangleleft$ is well-defined and it well-orders $B$. Now let us denote $X_{<x}=\{y\in X:y<x\}$ for an ordered set $(X;<)$ and its element $x$. For any $(X,R)\in W$, $(X,R)\cong B_{\triangleleft[(X,R)]}$ holds because there is an order isomorphism $f:(X,R)\rightarrow B_{\triangleleft[(X,R)]}$ defined by $f(x)=[(X_{<x},R\upharpoonright X_{<x})]$. Therefore if $B$ is countable, it is isomorphic to some $(X,R)\in W$ and is also isomorphic to a proper initial segment of itself (namely $B_{\triangleleft[(X,R)]}$), which is a contradiction. Thus $B$ is uncountable. Note that power sets and quotient sets can be constructed without Replacement by Exercise I.10.5 and hence we have an uncountable well-ordered set within $Z^-$.
\end{customthm}

\begin{customthm}{I.11.36}
  Let $(A,\triangleleft)$ be a total order with $A=\{a_n:n\in\omega\}$. For $n\in\omega$, let $L_n=\{m\in\omega:m<n\wedge a_m<a_n\}$ and $R_n=\{m\in\omega:m<n\wedge a_n<a_m\}$. we define $f:A\rightarrow\mathbb{Q}$ as follows.
  \[
    f(a_n)=
    \begin{cases}
      0 & \text{if } n=0 \\
      f(\min(R_n))-1 & \text{if } n>0 \text{ and } L_n=\emptyset \\
      f(\max(L_n))+1 & \text{if } n>0 \text{ and } R_n=\emptyset \\
      \frac{1}{2}(f(\max(L_n))+f(\min(R_n))) & \text{if } n>0 \text{, } L_n\neq\emptyset \text{ and } R_n\neq\emptyset
    \end{cases}.
  \]
  It is easy to see that this $f$ actually is an order-preserving injection.
\end{customthm}

\begin{customthm}{I.11.37}
  We prove the following.
  \begin{itemize}
    \item\underline{(1)$\rightarrow$(2)}. This is trivial from $\mathbb{Q}\subseteq\mathbb{R}$.
    \item\underline{(2)$\rightarrow$(3)}. Let $A\subseteq\mathbb{R}$ be a well-ordered set of order type $\alpha$ and let $a:\alpha\rightarrow A$ be the isomorphism. We define a family of disjoint open intervals $(I_\xi)_{\xi\in\alpha}$ by $I_\xi=(a(\xi),a(\xi+1))$, where $a$ is extended by $a_\alpha=\infty$ when $\alpha$ is a successor. If $\alpha$ is uncountable, then at least one of $(I_\xi)_{\xi\in\alpha}$ does not have any rational numbers in it since $\mathbb{Q}$ is countable, but it is impossible because $\mathbb{Q}$ is dense in $\mathbb{R}$. Thus $\alpha$ has to be countable.
    \item\underline{(3)$\rightarrow$(4)}. Let $P(\alpha,f)$ say that $f$ is an isomorphism from $\alpha$ to a subset of $\mathbb{Q}$ which has $0$ in it and is a closed subset of $[0,1)$. We show that for any $0<\alpha<\omega_1$, there is an $f$ such that $P(\alpha,f)$. To prove it, we fix a non-zero ordinal $\delta<\omega_1$, fix an injection $i:\delta\rightarrow\omega$ and recursively define a sequence of functions $(f_\alpha)_{0<\alpha<\delta}$ such that $P(\alpha,f_\alpha)$ holds for any $0<\alpha<\delta$. When $\alpha=1$, then the function $f_1$ defined by $f_1(0)=0$ satisfies $P(1,f_1)$. When $\alpha=\xi+1$, assuming the inductive hypothesis, we have an $f_\xi$ that satisfies $P(\xi,f_\xi)$. Then the following function $f_\alpha$ defined by $f_\alpha(\eta)=f_\xi(\eta)/2$ for $\eta<\xi$ and $f_\alpha(\xi)=1/2$ satisfies $P(\alpha,f_\alpha)$. When $\alpha$ is a limit ordinal, we first define a countably infinite increasing sequence of ordinals $0=\xi_0<\xi_1<\cdots<\alpha$ by $\xi_{n+1}=\min_{\eta<\alpha}(i(\xi_n)<i(\eta))$. Then by assuming the inductive hypothesis, we have a sequence of functions $(f_n)_{n<\omega}$ each of which satisfies $P(\type(\xi_{n+1}\setminus\xi_n),f_n)$. Now for $n<\omega$, let $j_n:(\xi_{n+1}\setminus\xi_n)\rightarrow\type(\xi_{n+1}\setminus\xi_n)$ be the isomorphism and we define a function $f:\alpha\rightarrow[0,\infty)$ by $f(\xi)=f_n(j_n(\xi))+n$ where $n<\omega$ is the least ordinal such that $\xi<\xi_{n+1}$. Since the image of $f$ has every natural number and is closed in every half-open interval $[n,n+1)$, it is closed in the whole $[0,\infty)$. By considering another function $g:[0,\infty)\rightarrow[0,1)$ defined by $g(x)=x/(x+1)$, we get an $f_\alpha=g\circ f$ that satisfies $P(\alpha,f_\alpha)$. Now we have shown that our $(f_\alpha)_{0<\alpha<\delta}$ satisfies $P(\alpha,f_\alpha)$. Since $0<\delta<\omega_1$ is arbitrary, we can say that for any $0<\alpha<\omega_1$, there is an $f$ which satisfies $P(\alpha,f)$. Now it is obvious that any countable ordinal greater than $0$ satisfies (4). It is trivial that $0$ also satisfies (4).
    \item\underline{(4)$\rightarrow$(1)}. This is obvious.
  \end{itemize}
\end{customthm}

\begin{customthm}{I.11.38}
  Skipped for now.
\end{customthm}

\begin{customthm}{I.12.6}
  Let $\mathcal{F}$ be of finite character, $X\in\mathcal{F}$, and $Y\subseteq X$. $Y\in\mathcal{F}$ holds because every finite subset of $Y$ is in $\mathcal{F}$ since it is a finite subset of $X\in\mathcal{F}$ as well.
\end{customthm}

\begin{customthm}{I.12.13}
  We define a function $f:\mathcal{P}(\omega\times\omega)\rightarrow\omega_1$ as follows.
  \[
    f(R)=
    \begin{cases}
      \type(R) & \text{if } R \text{ well-orders } \omega \\
      |R| & \text{if } R \text{ is finite} \\
      0 & \text{otherwise}
    \end{cases}.
  \]
  This $f$ is surjective because $\ran(f)$ includes all countable ordinals. On the other hand, since $\omega\times\omega\preccurlyeq\omega$, we have an injection from $\mathcal{P}(\omega\times\omega)\approx {}^{\omega\times\omega}2$ to $\mathcal{P}(\omega)\approx {}^\omega 2$ by Exercise I.11.12. Hence we have a surjection from $\mathcal{P}(\omega)$ to $\mathcal{P}(\omega\times\omega)$ and we get a surjection from $\mathcal{P}(\omega)$ to $\omega_1$ composing that surjection and $f$.
\end{customthm}

\begin{customthm}{I.12.14}
  For each $n\in\omega$, let $\mathcal{F}_n=\{Y\in\mathcal{P}(X):\forall x,y\in Y\,[x\neq y\rightarrow d(x,y)>2^{-n}]\}$. Then $\mathcal{F}_n$ is of finite character and hence it has a maximal element $E_n$ by Tukey's Lemma. Now let $E=\bigcup_n E_n$. If $E$ is not dense in $X$, there is $x\in X$ and $n\in\omega$ such that the ball of radius $2^{-n}$ centered at $x$ does not intersect $E$, which contradicts $E_n$'s maximality. Thus $E$ is dense and since $X$ is non-separable, $E$ is uncountable. Then there is some uncountable $E_n$ because countable union of countable sets would be countable. That $E_n$ with $\varepsilon=2^{-n}$ is what we want.
\end{customthm}

\begin{customthm}{I.12.16}
  Let $X,Y$ be compact Hausdorff spaces, $f:X\rightarrow Y$ be a continuous surjection, and $\mathcal{F}=\{Z\in\mathcal{P}(X):Z\text{ is closed and }f\upharpoonright Z\text{ is surjective}\}$. Let $C$ be a chain in $(\mathcal{F},\supsetneq)$. We prove $\bigcap C\in\mathcal{F}$. It is clear that $\bigcap C$ is closed because $C$ is the intersection of closed sets. Assume $f\upharpoonright\bigcap C$ is not surjective and let $y\in Y\setminus\bigcap C$. Since $f$ is continuous and any singleton in a Hausdorff space is closed, $f^{-1}(y)=\{x\in X:f(x)=y\}$ is a closed subset in $X$, which is compact since $X$ is compact. Then $X\setminus\bigcap C=\bigcup_{Z\in C}(X\setminus Z)$, an open cover of $f^{-1}(y)$, has a finite subcover $(X\setminus Z_1,X\setminus Z_2,\ldots,X\setminus Z_n)$. We can assume $Z_1\supseteq Z_2\supseteq\ldots\supseteq Z_n$ since $C$ is a chain. Then $X\setminus Z_n=\bigcup_{k=1,\ldots,n}(X\setminus Z_k)\supseteq f^{-1}(y)$ holds. This means that $y\not\in\ran(f\upharpoonright Z_k)$, which contradicts $Z_n\in\mathcal{F}$. Hence $f\upharpoonright\bigcap C$ is surjective and $C\in\mathcal{F}$ holds. Now we have $\bigcap C$ as an upper bound for any chain $C$, $\mathcal{F}$ turns out to have a maximal set $Z$ by Zorn's Lemma. Such $Z$ is what we want.
\end{customthm}

\begin{customthm}{I.12.17}
  Let $A$ be a set. To prove the 'only if' part, assume $\omega\preccurlyeq A$ and let $i:\omega\rightarrow A$ be an injection. Then we can define an injection $f:A\rightarrow A$ with $\ran(f)\neq A$ as follows.
  \[
    f(x)=
    \begin{cases}
      f(i(n+1)) & \text{if } x=i(n) \\
      x & \text{otherwise}
    \end{cases}.
  \]
  This $f$ does not contain $i(0)\in A$ in its range. Hence $A$ is not Dedekind-finite.

  To prove the 'if' part, assume $A$ is not Dedekind-finite and let $f:A\rightarrow A$ be an injection with $\ran(f)\neq A$. Then there is $a\in A$ such that $f^n(a)\neq a$ holds for all positive natural number $n$, because otherwise every $a\in A$ would be in the range of $f$ for $a=f(f^{n-1}(a))$. Now fix such $a\in A$ and define $i:\omega\rightarrow A$ by $i(n)=f^n(a)$. This is an injection from $\omega$ to $A$ and we have $\omega\preccurlyeq A$.
\end{customthm}

\begin{customthm}{I.12.18}
  Let $A$ be dense in $X$. $A$ cannot be finite because $X$ is not finite. (To see that, one can take $x\in X\setminus A$ and consider a ball $B=\{y\in X: |y-x|<\min\{|x-a|:a\in A\}\}$. $B$ does not intersect $A$.) $A$ cannot be countably infinite either because it would give a bijection $i:\omega\rightarrow A$, which is an injection to $X$, contradicting that $X$ is not Dedekind-finite. This can be checked by Exercise I.12.17. Thus $A$ has to be uncountable, which proves the inseparability of $X$.
\end{customthm}

\begin{customthm}{I.13.17}
  Let $\kappa$ be a weakly inaccessible cardinal and let $A=\{\alpha:\alpha=\aleph_\alpha\}$. We first show that $\kappa\in A$. Let $\kappa=\aleph_\lambda$. It is obvious that $\lambda\leq\kappa$. Note that $\lambda$ is a limit ordinal since $\kappa$ is not $\omega$ and is regular. Exploiting the regularity of $\kappa$ again, we get $\kappa=\cf(\kappa)=\cf(\aleph_\lambda)=\cf(\lambda)\leq\lambda$. Thus $\lambda=\kappa$ holds and we have $\kappa\in A$. To see that $\kappa$ is the $\kappa$-th element of $A$, let $\nu=\type(A\cap\kappa)$. That is, $\kappa$ is the $\nu$-th element of $A$. Here $\sup(A\cap\kappa)=\kappa$ holds since for a given $\alpha\in A\cap\kappa$, we can get a larger cardinal $\beta=\sup\{\beta_n:n\in\omega\}$ in $A\cap\kappa$ where $\beta_0=\aleph_{\alpha+1}$ and $\beta_{n+1}=\aleph_{\beta_n}$. Therefore we have $\cf(\kappa)=\cf(\nu)\leq\nu$ and we get $\nu=\kappa$.

  Next, let $\kappa$ be a strongly inaccessible cardinal and let $\lambda$ be the least ordinal such that $\kappa\leq\beth_\lambda$. $\lambda$ is obviously not $0$ from $\kappa\geq\omega$. It is not a successor either since we have $\beth_{\xi+1}<\kappa$ whenever $\beth_{\xi}<\kappa$ holds. Therefore $\lambda$ is a limit ordinal. If $\kappa<\beth_\lambda$, there is some ordinal $\mu<\lambda$ such that $\kappa<\beth_\mu$, contradicting the minimality of $\lambda$. Thus $\kappa=\beth_\lambda$ holds and from the cofinality argument like above, we get $\kappa=\beth_\kappa$. One can prove that $\kappa$ is actually the $\kappa$-th element of $\{\alpha:\alpha=\beth_\alpha\}$ in the same way as we did for weakly inaccessible cardinals.
\end{customthm}

\begin{customthm}{I.13.18}
  Let $\delta_0=0$, $\delta_{n+1}=\beth_{\delta_n}$ and $\gamma=\sup\{\delta_n:n\in\omega\}$. Then $\beth_\gamma=\gamma$ holds as is proved in Exercise I.11.33.
\end{customthm}

\begin{customthm}{I.13.20}
  Skipped for now.
\end{customthm}

\begin{customthm}{I.13.21}
  We can define an injection $i:A\rightarrow\bigcup_{n\in\omega}(K\times B)^n$ by $i(a_1v_1+\cdots+a_nv_n)=((a_1,v_1),\ldots(a_n,v_n))$. Since $\omega\leq\max(|K|,|B|)$ and $(K\times B)^n\approx\max(|K|,|B|)^n\approx\max(|K|,|B|)$ hold, we have $\bigcup_{n\in\omega}(K\times B)^n\approx\max(|K|,|B|)$ by Theorem I.12.11. Hence $|A|\leq\max(|K|,|B|)$ holds. On the other hand, $|B|\leq|A|$ is obvious from $B\subseteq A$ and so is $|K|\leq|A|$ because we can define an injection $i:K\rightarrow A$ by $i(a)=av$ for a fixed $v\in B$. Thus we have $|A|=\max(|K|,|B|)$. Now we view $\mathbb{R}$ and $\mathbb{C}$ as vector spaces over $\mathbb{Q}$ with bases $B_\mathbb{R}$ and $B_\mathbb{C}$ respectively. Then by the above, we have $|\mathbb{R}|=\max(|\mathbb{Q}|, |B_\mathbb{R}|)$. Since $|\mathbb{Q}|=\aleph_0<2^{\aleph_0}=|\mathbb{R}|$, we get $|B_\mathbb{R}|=2^{\aleph_0}$. The same argument goes for $\mathbb{C}$ and we get $|B_\mathbb{C}|=|B_\mathbb{R}|=2^{\aleph_0}$. Hence $\mathbb{R}$ and $\mathbb{C}$ have the same dimensionality as vector spaces over $Q$, which means they are isomorphic. Since isomorphisms between vector spaces are group isomorphisms as well, we can conclude that $\mathbb{R}$ and $\mathbb{C}$ are isomorphic as Abelian groups.
\end{customthm}

\begin{customthm}{I.14.4}
  The list is too long to show here. You can find Python code in \url{https://github.com/kumatheworld/math-playground/blob/master/foundations.py}.
\end{customthm}

\begin{customthm}{I.14.9}
  Let $x\in WF$,  $\beta=\rank(x)$ and $\gamma=\rank(\bigcup x)$. Note that $\gamma\leq\beta$ holds from Lemma I.14.8(4). We prove the following.
  \begin{enumerate}
    \item\underline{$\gamma=\beta$ if $\beta$ is a limit ordinal or 0}. If $x=0$, then $\bigcup x=\emptyset$ and it is obvious that $\gamma=\beta=0$. Now assume $\beta$ is a limit ordinal. To see $\gamma\geq\beta$, we show that $\xi<\gamma$ for all $\xi<\beta$. Let $\xi$ be an ordinal such that $\xi<\beta$. Then by Theorem I.14.6, there exists $y\in x$ such that $\rank(y)\geq\xi+1$ and $z\in y$ such that $\rank(z)\geq\xi$. Since $z\in y\in x$ holds, $z\in\bigcup x$ and we have $\gamma\geq\rank(z)+1>\xi$. Thus $\gamma=\beta$ holds.
    \item\underline{$\gamma=\alpha$ if $\beta=\alpha+1$}. Let $\beta=\alpha+1$. If $z\in\bigcup x$ and $y$ be one such that $z\in y\in x$, then $\rank(z)<\rank(y)<\rank(x)=\alpha+1$ holds, which means $\rank(z)+1\leq\alpha$. Hence we have $\gamma\leq\alpha$. On the other hand, there exists $y\in z$ such that $\rank(y)=\alpha$. Then $y\subseteq\bigcup x$ holds and we have $\alpha\leq\gamma$ by Lemma I.14.7. Thus $\gamma=\alpha$ holds.
  \end{enumerate}
\end{customthm}

\begin{customthm}{I.14.14}
  Let $f:\omega\rightarrow HF$ be the one defined recursively by $f(m)=\{f(n):nEm\}$ and $g:HF\rightarrow\omega$ be the one defined recursively by $g(x)=\sum_{y\in x}2^{g(y)}$. To see $(\omega;E)\cong(HF;\in)$, we show the following.
  \begin{enumerate}
    \item\underline{$g\circ f = \mathrm{id}_\omega$}. We show this inductively. Let $m\in\omega$ and assume $g(f(n))=n$ for all $n<m$. Then we have $g(f(m))=g(\{f(n):nEm\})=\sum_{n:nEm}2^{g(f(n))}=\sum_{n:nEm}2^n=m$.
    \item\underline{$f\circ g = \mathrm{id}_{HF}$}. We show this inductively too. Let $x\in HF$ and assume $f(g(y))=y$ for all $y\in x$. Then we have $f(g(x))=\{f(n):nE_{g(x)}\}=\{f(n):nE_{\sum_{y\in x}2^{g(y)}}\}=\{f(g(y)):y\in x\}=\{y:y\in x\}=x$. Note here that $nE_{\sum_{y\in x}2^{g(y)}}\leftrightarrow n=g(y)$ holds since $g\upharpoonright x$ is injective by inductive hypothesis.
    \item\underline{$nEm\leftrightarrow f(n)\in f(m)$}. This is obvious from the definition of $E$.
  \end{enumerate}
  From the above, $f$ turns out to be an order isomorphism and we get $(\omega;E)\cong(HF;\in)$.
\end{customthm}

\begin{customthm}{I.14.15}
  Assume $WF\not\subseteq K$ and let $x\in WF\setminus K$ be one with the lowest rank. Since $x\not\in K$, $x\not\subseteq K$ holds and hence there is $y\in x\setminus K$. From Lemma I.14.4(5), $y\in WF$ and $\rank(y)<\rank(x)$ hold, which contradicts the minimality of $\rank(x)$. Thus $WF\subseteq K$ holds.
\end{customthm}

\begin{customthm}{I.14.16}
  We show $|R(\omega+\xi)|=\beth_\xi$ by induction. The equation holds when $\xi=0$ because we have $|R(\omega)|=\omega$ from Exercise I.14.14. When $\xi$ is a successor, let $\xi=\eta+1$ and assume $|R(\omega+\eta)|=\beth_\eta$. Then we have $|R(\omega+\xi)|=|2^{R(\omega+\eta)}|=2^{|R(\omega+\eta)|}=2^{\beth_\eta}=\beth_\xi$. Now assume $\xi$ is a limit ordinal and $|R(\omega+\eta)|=\beth_\eta$ holds for all $\eta<\xi$. It is obvious that $|R(\omega+\xi)|\geq\beth_\xi=\sup\{\beth_\eta:\eta<\xi\}$ because $|R(\omega+\xi)|\geq|R(\omega+\eta)|=\beth_\eta$ holds for all $\eta<\xi$. To see $|R(\omega+\xi)|\leq\beth_\xi$, we first note that $R(\omega+\xi)=\bigcup_{\zeta<\omega+\xi}R(\zeta)=\bigcup_{\eta<\xi}R(\omega+\eta)$ holds since $\omega+\xi$ is a limit ordinal as well. The last equality follows from $R(n)\subseteq R(\omega)$ for all $n<\omega$. On the other hand, $\omega+\xi\leq\beth_\xi$ and $R(\omega+\eta)\leq\beth_\xi$ hold for any $\xi\in ON$. One can easily show the first inequality by induction and the second one is obvious from our inductive hypothesis. Then by Theorem I.12.11, we get $|R(\omega+\xi)|=|\bigcup_{\eta<\xi}R(\omega+\eta)|\leq\beth_\xi$. Thus $|R(\omega+\xi)|=\beth_\xi$ holds for all ordinals $\xi$.
\end{customthm}

\begin{customthm}{I.14.17}
  Assume $\forall x\in A\,\exists y\,\varphi(x,y)$. Let $\hat\varphi(x,\alpha)$ say that $\alpha$ is the least ordinal such that $\exists y\in R(\alpha)\,\varphi(x,y)$. Then $\forall x\in A\,\exists!\alpha\,\hat\varphi(x,\alpha)$ holds and we get $\exists\hat B\,\forall x\in A\,\exists\alpha\in\hat B\,\hat\varphi(x,\alpha)$ from Replacement. For such $\hat B$, let $B=\bigcup_{\alpha\in B\cap ON}R(\alpha)$. Then we have $\forall x\in A\,\exists y\in B\,\varphi(x,y)$.
\end{customthm}

\begin{customthm}{I.14.19}
  Let $\kappa$ be an infinite cardinal.
  \begin{enumerate}
    \item Let $y\in H(\kappa)$ and consider a set of ordinals $\alpha=\{\rank(x):x\in \trcl(y)\}$. By Lemma I.14.11, we have $\alpha\in ON$. Moreover, $\alpha<\kappa$ holds because $|\alpha|\leq|\trcl(y)|<\kappa$ holds. Since $\rank(x)<\alpha$ holds for all $x\in\trcl(y)$, we have $\rank(\trcl(y))\leq\alpha<\kappa$. From $\rank(y)=\rank(\trcl(y))$, we get $\rank(y)<\kappa$, which means $y\in R(\kappa)$. Thus $H(\kappa)\subseteq R(\kappa)$ holds.
    \item $|H(\kappa)|\geq2^{<\kappa}$ holds because $\mathcal{P}(\alpha)\subseteq H(\kappa)$ holds for all $\alpha<\kappa$. When $\kappa=\omega$, the inequality is nothing but $|HF|=\aleph_0$, which we know already. To prove $|H(\kappa)|\leq2^{<\kappa}$ in general, we first consider the cases when $\kappa$ is a successor ordinal. Let $\kappa=\theta^+$. Then $2^{<\kappa}=2^\theta$ holds. Now let $S(\alpha)=H(\theta^+)\cap R(\alpha)$ for $\alpha\in ON$. We show that $|S(\alpha)|\leq2^\theta$ by induction on $\alpha\leq\theta^+$. If $\alpha=0$, $S(\alpha)=0$ holds and we obviously have $|S(\alpha)|\leq2^\theta$. If $\alpha=\xi+1$, consider the relation $S(\alpha)\subseteq[S(\xi)]^{<\theta^+}$, which means any $y\in S(\alpha)$ is a set of elements in $S(\xi)$ and its has cardinality less than $\theta^+$. By this inclusion, the inductive hypothesis and Exercise I.13.20, we get $|S(\alpha)|\leq(2^\theta)^\theta=2^\theta$. If $\alpha$ is a limit ordinal, $S(\alpha)=\bigcup_{\xi<\alpha}S(\xi)$ holds and we get $|S(\alpha)|\leq2^\theta$ from the inductive hypothesis and $|\alpha|\leq\theta^+\leq2^\theta$. Now we have $|S(\theta^+)|\leq2^\theta$. Since $H(\theta^+)\subseteq R(\theta^+)$ holds from the above, we get $H(\theta^+)=S(\theta^+)$ and hence $|H(\theta^+)|=|S(\theta^+)|\leq2^\theta$. If $\kappa$ is a limit cardinal, it holds that $H(\kappa)=\bigcup_{\theta<\kappa}H(\theta)$. We can think of this union as the union only taken among the successor cardinals below $\kappa$. Then from what we proved above, we get $|H(\kappa)|\leq2^\kappa=2^{<\kappa}$. Thus $|H(\kappa)|=2^{<\kappa}$ holds for any infinite cardinal $\kappa$.
    \end{enumerate}
\end{customthm}

\begin{customthm}{I.14.20}
  Let $y\subseteq H(\kappa)$ and $|y|<\kappa$. We show that $|\trcl(y)|<\kappa$. It suffices to show that $|\bigcup\{\trcl(z):z\in y\}|<\kappa$ since $\trcl(y)=y\cup\bigcup\{\trcl(z):z\in y\}$ holds and we know that $|y|<\kappa$ by assumption and union does not raise cardinality. Here $|\bigcup\{\trcl(z):z\in y\}|<\kappa$ follows from Theorem I.13.12 since $\kappa$ is regular, $|y|<\kappa$ holds and we have $|\trcl(z)|<\kappa$ for any $z\in y$ since $y\subseteq H(\kappa)$. Thus $|\trcl(y)|<\kappa$ holds, which is nothing but $y\in H(\kappa)$.

  Now let $z\in H(\kappa)$ and $f:z\rightarrow H(\kappa)$. We prove that $f\in H(\kappa)$. Note that $\trcl(f)=f\cup\bigcup\{\trcl(\langle x,y\rangle):\langle x,y\rangle\in f\}$ holds. Here one can easily break down every $\trcl(\langle x,y\rangle)$ into $\{x,y,\{x\},\{x,y\}\}\cup\trcl(x)\cup\trcl(y)$, which has cardinality less than $\kappa$ since $|\trcl(x)|<\kappa$ and $|\trcl(y)|<\kappa$ hold from $x,y\in H(\kappa)$. On the other hand, we have $|f|\leq|z|\leq|\trcl(z)|<\kappa$. Thus by Theorem I.13.12, $|\trcl(f)|<\kappa$ holds, which means $f\in H(\kappa)$.
\end{customthm}

\begin{customthm}{I.14.21}
  Let $\gamma>\omega$ be a limit ordinal. We first prove that $R(\gamma)$ is a model for $ZC$ by showing that it satisfies each of the following.
  \begin{enumerate}
    \setcounter{enumi}{-1}
    \item\textbf{Set Existence.} We have $0\in R(\gamma)$.
    \item\textbf{Extensionality.} Let $x,y\in R(\gamma)$ and assume that $z\in x\leftrightarrow z\in y$ holds for all $z\in R(\gamma)$. It implies that $z\in x\leftrightarrow z\in y$ holds for all $z\in V$ since $R(\gamma)$ is transitive. Hence $x=y$ holds in $V$, which also means $x=y$ in $R(\gamma)$.
    \item\textbf{Foundation.} For a given non-empty $x\in R(\gamma)$, an element $y\in x$ of the least rank has the property. Note that such a $y$ is also in $R(\gamma)$.
    \item\textbf{Comprehension Scheme.} For a given $z\in R(\gamma)$ and a formula $\varphi$ without $y$ free, the set $\{x\in z:\varphi(x)\}$ has the property. Since it is a subset of $z$, it has rank no greater than $\rank(z)$, which shows that it is also in $R(\gamma)$.
    \item\textbf{Pairing.} For given $x,y\in R(\gamma)$, the pair $\{x,y\}$ has the property. To see that $\{x,y\}\in R(\gamma)$, consider its rank, which is $\max\{\rank(x),\rank(y)\}+1$. Since $\gamma$ is a limit ordinal, $\rank(\{x,y\})<\gamma$ holds whenever $\rank(x),\rank(y)<\gamma$.
    \item\textbf{Union.} For a given $x\in R(\gamma)$, the union $\bigcup x$ has the property. Note that $\rank(\bigcup x)\leq\rank(x)$ holds and hence $\bigcup x\in R(\gamma)$.
    \item\textbf{Replacement Scheme.} Some instances might fail. See below for a counterexample when $\gamma=\omega+\omega$.
    \item\textbf{Infinity.} $\omega\in R(\gamma)$ has the property. Note that $\emptyset$ in $R(\gamma)$ is the empty set (of $V$, i.e. $0$), which is in $R(\gamma)$. Also, $S(x)$ in $R(\gamma)$ for a given $x\in R(\gamma)$ is the successor of $x$ (in $V$, i.e. $S(x)$), which is in $R(\gamma)$.
    \item\textbf{Power Set.} For a given $x\in R(\gamma)$, the power set $\mathcal{P}(x)$, which is also in $R(\gamma)$ since $\rank(\mathcal{P}(x))=\rank(x)+1<\gamma$ holds, has the property. To see that, let $z\in R(\gamma)$ and assume that $z\subseteq x$ holds in $R(\gamma)$. We show that $z\subseteq x$ holds in $V$. Let $w\in z$. Since $R(\gamma)$ is transitive, $w\in R(\gamma)$ holds and we have $w\in x$ in $R(\gamma)$ from $z\subseteq x$ in $R(\gamma)$. Thus $z\subseteq x$ also holds in $V$, which means $z\in\mathcal{P}(x)$.
    \item\textbf{Choice.} For a family $F\in R(\gamma)$ of mutually exclusive non-empty sets, let $C\subseteq\bigcup F$ be a set (in $V$) which has exactly one element in each $x\in F$. Note that $C$ is in $R(\gamma)$ since it has rank at most $\rank(\bigcup F)$. To prove that $C$ really is a "choice set" in $R(\gamma)$, one can again use the transitivity of $R(\gamma)$ to see the notions in $R(\gamma)$ that appear in the axiom, such as $\forall x\in F$, $x\cap y=\emptyset$ and $\textrm{SING}$, really have the same meaning as what they mean in $V$.
  \end{enumerate}
  To see that Replacement fails in $R(\omega+\omega)$, consider the formula $\varphi(x,y)=x\in\omega\wedge\exists f[\dom(f)=x+1\wedge f(x)=y\wedge f(0)=\omega\wedge\forall n\in x[f(n+1)=\{f(n)\}]]$. Applying Replacement to $\varphi$, we would get $\omega+\omega$, which does not exist in $R(\omega+\omega)$. Thus Replacement is false in $R(\omega+\omega)$ and $\varphi$ is a counterexample.
\end{customthm}

\begin{customthm}{I.14.22}
  Let $\kappa>\omega$ be a regular cardinal. For $H(\kappa)$ and $H(\omega)$, most of the axioms can be proved similarly as in Exercise I.14.21, based on facts such as that $H(\kappa)$ is a transitive set or that $x\subseteq y\in H(\kappa)\rightarrow x\in H(\kappa)$. It is obvious that the Axiom of Infinity in $H(\omega)$ fails because any set having the property of the axiom would have an infinite rank. Here we only consider the Replacement Schemes for $H(\kappa)$ and $H(\omega)$, and the Power Set Axiom for $H(\kappa)$.

  We first show that Replacement holds in $H(\theta)$ when $\theta$ is a regular cardinal, which means that it is true in both $H(\kappa)$ and $H(\omega)$. Let $A\in H(\theta)$ and let $\varphi$ be a formula without $B$ free which satisfies $\forall x\in A\,\exists!y\,\varphi(x,y)$. Note that the meanings of $\forall$ and $\exists!$ are the same in both $V$ and $H(\theta)$ since $H(\theta)$ is transitive. By Replacement in $V$, we have a set $B=\{y:\exists x\,\varphi(x,y)\}$. Note here that $B\subseteq H(\theta)$ and $|B|\leq|A|\leq|\trcl(A)|<\theta$ hold. Thus by Exercise I.14.20, we have $B\in H(\theta)$, which shows that Replacement is true.

  Now assume that the Power Set Axiom holds in $H(\kappa)$. Then all the ZFC axioms hold and we have $2^\lambda\in H(\kappa)$ whenever $\lambda\in H(\kappa)$ is a cardinal. That is, $2^\lambda<\kappa$ holds whenever $\lambda<\kappa$ holds, which shows the strong inaccessibility of $\kappa$.

  Conversely, assume that $\kappa$ is strongly inaccessible and consider any $y\in H(\kappa)$ and its power set $\mathcal{P}(y)$. Since $\trcl(\mathcal{P}(y))=\mathcal{P}(y)\cup\bigcup\{\trcl(x):x\subseteq y\}=\mathcal{P}(y)\cup\trcl(y)$ holds, we have $|\trcl(\mathcal{P}(y))|=2^{|\trcl(y)|}<\kappa$ by the strong inaccessibility of $\kappa$. Hence we get $\mathcal{P}(y)\in H(\kappa)$. Note that $\mathcal{P}(y)$ is really the power set of $y$ in $H(\kappa)$ because $H(\kappa)$ is transitive. Thus the Power Set Axiom holds.

  Furthermore, if $\kappa$ is strongly inaccessible, $H(\kappa)=R(\kappa)$ holds. Since $H(\kappa)\subseteq R(\kappa)$ is proven for any cardinal $\kappa$ so far, we only prove the opposite direction here. Let $y\in R(\kappa)$ and let $\gamma=\rank(y)$. Considering $\trcl(y)\subseteq R(\gamma)$, we have $|\trcl(y)|\leq|R(\gamma)|\leq\beth_\gamma<\beth_\kappa=\kappa$, where $|R(\gamma)|\leq\beth_\gamma$ and $\beth_\kappa=\kappa$ are already proved in Exercise I.14.17 and Exercise I.13.17 respectively. Thus $y\in H(\kappa)$ holds and we get $H(\kappa)=R(\kappa)$.
\end{customthm}

\begin{customthm}{I.14.23}
  Let $\psi(y)$ be a formula such that $\forall y\,[(\forall x\in y\,\psi(x))\rightarrow\psi(y)]$. Assume $\neg\forall y\,\psi(y)$ and let $y$ be a set of the least rank such that $\neg\psi(y)$. Then by the contraposition of $(\forall x\in y\,\psi(x))\rightarrow\psi(y)$, there is a set $x\in y$ such that $\neg\psi(x)$. Here we have $\rank(x)<\rank(y)$ since $x\in y$, which contradicts the minimality of $\rank(y)$.
\end{customthm}

\begin{customthm}{I.14.24}
  Skipped for now.
\end{customthm}

\begin{customthm}{I.15.2}
  $\mathbb{Z}\subseteq\mathbb{Q}$ is obvious from their definitions. To see $\mathbb{Q}\subseteq HF$, we show $\rank(x)<\omega$ for all $x$. If $x\in\omega$, then $\rank(x)<\rank(\omega)=\omega$. If $x=\langle i,\langle m,n\rangle\rangle$ for some $i,m,n\in\omega$, then $\rank(x)=\max\{\rank(i),\max\{\rank(m),\rank(n)\}+2\}+2\in\omega$. (For example, we get $\rank(-2/3)=\rank(\langle 1,\langle 2,3\rangle\rangle)=7$ from this.) Hence $\mathbb{Q}\subseteq HF$ holds. $\rank(\mathbb{Z})\leq\omega$ follows from $\mathbb{Z}\subseteq HF$, and we have $\rank(\mathbb{Z})>n$ for any $n\in\omega$ since $n\in\mathbb{Z}$ and $\rank(n)=n$. Thus $\rank(\mathbb{Z})$ is equal to $\omega$ and so is $\rank(\mathbb{Q})$.
\end{customthm}

\begin{customthm}{I.15.5}
  Let $x\in\mathbb{R}$. $\rank(x)\leq\omega$ holds because $\rank(q)<\omega$ holds for all $q\in x$. Now let $q\in x$ and let $n\in\omega$ be one such that $-n<q$. Then $-n\in x$ and we have $\rank(x)>\rank(-n)=\rank(\langle 1,\langle n,1\rangle\rangle)=n+4$. Since $n$ can be arbitrarily large, we have $\rank(x)\geq\omega$. Hence $\rank(x)=\omega$ for all $x\in\mathbb{R}$ and we get $\rank(\mathbb{R})=\omega+1$. For $z=\langle x,y\rangle\in\mathbb{C}$, we have $\rank(z)=\max\{\rank(x),\rank(y)\}+2=\omega+2$, which proves $\rank(\mathbb{C})=\omega+3$.
\end{customthm}

\begin{customthm}{I.15.8}
  We prove the following.
  \begin{enumerate}
    \item\underline{$|\mathbb{R}|=2^{\aleph_0}$}. $|\mathbb{R}|\leq2^{\aleph_0}$ is clear from $\mathbb{R}\subseteq{\mathcal{P}(\mathbb{Q})}$ and $\mathcal{P}(\mathbb{Q})\approx2^\mathbb{Q}\approx2^{\aleph_0}$. However, the opposite is not straightforward from our definition of $\mathbb{R}$. One can prove it by treating $\mathbb{R}$ as the set of infinite decimals and using the diagonal argument. But we would need to prove that the set of infinite decimals is isomorphic to our $\mathbb{R}$ for that, which would be too long to write down here.
    \item\underline{$|C(\mathbb{R},\mathbb{R})|=2^{\aleph_0}$}. $|C(\mathbb{R},\mathbb{R})|\geq2^{\aleph_0}$ is obvious because $\mathbb{R}\succcurlyeq2^{\aleph_0}$ and we have a constant function $c_x:\mathbb{R}\rightarrow\mathbb{R}$ for each $x\in\mathbb{R}$. To prove the opposite, we use the well-known fact from topology that a continuous function from a Hausdorff space is unique up to the restriction to a dense subspace of their domain. Since $\mathbb{Q}$ is dense in $\mathbb{R}$, we have $C(\mathbb{R},\mathbb{R})\subseteq\mathbb{R}^\mathbb{Q}$ and $\mathbb{R}^\mathbb{Q}\approx(2^{\aleph_0})^\mathbb{Q}\approx2^{\aleph_0\times\mathbb{Q}}\approx2^{\aleph_0}$. Hence $|C(\mathbb{R},\mathbb{R})|=2^{\aleph_0}$ holds.
    \item\underline{$|\mathbb{R}^\mathbb{R}|=2^{2^{\aleph_0}}$}. $\mathbb{R}^\mathbb{R}\approx (2^{\aleph_0})^\mathbb{R}\approx 2^{\aleph_0\times\mathbb{R}}\approx2^\mathbb{R}\approx2^{2^{\aleph_0}}$.
  \end{enumerate}
\end{customthm}

\begin{customthm}{I.15.9}
  The set $\mathcal{B}=\mathcal{B}_{\omega_1}\subseteq\mathcal{P}(\mathbb{R})$ of Borel subsets is defined recursively, where $\mathcal{B}_0$ is the set of open sets and closed sets of $\mathbb{R}$, $\mathcal{B}_{\alpha+1}$ is the family of all countable unions and intersections of sets in $\mathcal{B}_\alpha$, and $\mathcal{B}_\gamma=\bigcup\{\mathcal{B}_\alpha:\alpha<\gamma\}$ for limit $\gamma$. Therefore we prove $|\mathcal{B_\alpha}|=2^{\aleph_0}$ by induction on $\alpha\leq\omega_1$ as follows.
  \begin{enumerate}
    \item\underline{$|\mathcal{B}_0|=2^{\aleph_0}$}. Let $\mathcal{O}$ be the set of open sets. $2^{\aleph_0}\leq|\mathcal{O}|$ is obvious because we have a different open interval $(x,x+1)$ for each $x\in\mathbb{R}$. To see that $|\mathcal{O}|\leq2^{\aleph_0}$, consider the following mapping $f:\mathcal{O}\rightarrow\mathcal{P}(\mathbb{Q}\times\mathbb{Q})$ defined by $f(U)=\{\langle q,r\rangle\in\mathbb{Q}\times\mathbb{Q}:(q-r,q+r)\subseteq U\}$. This is injective since there is always an interval $(q-r,q+r)$ where $q,r\in\mathbb{Q}$, contained in the difference in 2 different open sets. Thus we have $|\mathcal{O}|=2^{\aleph_0}$. On the other hand, since there is a 1-to-1 correspondence between closed sets and open sets by taking the complement, we get $\mathcal{B}_0=|2^{\aleph_0}+2^{\aleph_0}|=2^{\aleph_0}$.
    \item\underline{$|\mathcal{B}_{\alpha+1}|=2^{\aleph_0}$ under $|\mathcal{B}_\alpha|=2^{\aleph_0}$}.  $|\mathcal{B}_{\alpha+1}|\geq2^{\aleph_0}$ is obvious from $\mathcal{B}_{\alpha+1}\supseteq\mathcal{B}_\alpha$ and $|\mathcal{B}_\alpha|=2^{\aleph_0}$. There is a surjection from $2\times{}^\omega\mathcal{B}_{\alpha}$ to $\mathcal{B}_{\alpha+1}$ defined by $(0,f)\mapsto\bigcup_{n\in\omega}f(n)$ and $(1,f)\mapsto\bigcap_{n\in\omega}f(n)$, which proves $|\mathcal{B}_{\alpha+1}|\leq|2\times{}^\omega\mathcal{B}_{\alpha}|=|2\times2^{\aleph_0\times\omega}|=2^{\aleph_0}$. Thus $|\mathcal{B}_{\alpha+1}|=2^{\aleph_0}$ holds.
    \item\underline{$|\mathcal{B}_\gamma|=2^{\aleph_0}$ under $|\mathcal{B}_\alpha|=2^{\aleph_0}$ for all $\alpha<\gamma$}. By Theorem I.12.11 and $\gamma\leq\omega_1\leq2^{\aleph_0}$, we have $|\mathcal{B}_\gamma|=|\bigcup\{\mathcal{B}_\alpha:\alpha<\gamma\}|=2^{\aleph_0}$.
  \end{enumerate}
  From the above, we get $|\mathcal{B}|=2^{\aleph_0}$.

  For the set $\mathcal{L}$ of Lebesgue measurable sets, $|\mathcal{L}|\leq2^{2^{\aleph_0}}$ is obvious from $\mathcal{L}\subseteq\mathcal{P}(\mathbb{R})$. To prove $\mathcal{L}\supseteq\mathcal{P}(\mathbb{R})$, we make use of the Cantor set, which is of cardinality $2^{\aleph_0}$ and has measure $0$. Since every subset of a null set is measurable, we have at least $2^{2^{\aleph_0}}$ measurable sets. Hence $|\mathcal{L}|=2^{2^{\aleph_0}}$ holds.
\end{customthm}

\begin{customthm}{I.15.10} (half-done)
  We first note that the set of all closed nowhere dense sets, which we denote by $C$ here, has cardinality $\mathfrak{c}=2^{\aleph_0}$. This is because the set of all singletons, which is a subset of $C$, has that cardinality and the set of all closed sets, which is a superset of $C$, has that cardinality too. Let $U$ be the set of all countable unions of $C$. We have $|U|=\mathfrak{c}^{\aleph_0}=\mathfrak{c}$ from the above. Therefore we can list $U$ as $U=\{B_\alpha:\alpha<2^{\omega}\}$. Now we well-order $\mathbb{R}$ and define a family of sets $\{X_\alpha:\alpha\leq2^{\omega}\}$ by $X_0=\emptyset$, $X_{\xi+1}=X_\xi\cup\min(\mathbb{R}\setminus\{X_\xi\cup\bigcup_{\eta<\xi}B_\eta\})$, and $X_{\alpha}=\bigcup_{\xi<\alpha}X_\xi$ for a limit ordinal $\alpha$. \textbf{This definition works if it holds $\forall\xi<2^\omega[\bigcup_{\eta<\xi}B_\eta\neq\mathbb{R}]$, which we currently do not provide a proof for}. Then $X_{2^\omega}$ has cardinality $\mathfrak{c}$, which is uncountable. It has at most $\alpha$ elements from each $B_\alpha$, which is countable if we assume CH. Thus under CH, this $X_{2^\omega}$ is a Luzin set.

  The rest of the exercise is skipped for now.
\end{customthm}

\begin{customthm}{I.15.11} (half-done)
  The existence of a Sierpiński set under CH is skipped for now.

  Let $X\subseteq\mathbb{R}$ be a Sierpiński set. To prove that $X$ cannot be measurable, we show that the measure of $X$ cannot be neither $0$ or positive. It is clear that the measure cannot be 0 since $X$ would be countable and uncountable at the same time. To see that $X$ cannot have positive measure, we show that any set of positive measure has a null subset of the same cardinality. The proof is omitted for now.
\end{customthm}

\begin{customthm}{I.15.12}
  Skipped for now.
\end{customthm}

\begin{customthm}{I.15.14}
  Let $x\in HF^{<\omega}$. Every $y=(n,z)\in x$ has finite rank and so does $x$. Hence we have $x\in HF$ and $HF^{<\omega}\subseteq HF$.
\end{customthm}

\begin{customthm}{I.15.15}
  Let $A\subseteq\omega\setminus\{0\}$. Assume $A\cap A^{<\omega}\neq\emptyset$ and let $x\in A\cap A^{<\omega}$. Since $x\in\omega$ and $x\neq 0$, we have $0\in x$. On the other hand, since $A^{<\omega}$ is a set of functions, $x$ is a function and hence is a set of doubletons, which contradicts $0\in x$. Thus $A\cap A^{<\omega}=\emptyset$ holds.
\end{customthm}

\begin{customthm}{II.4.7}
  We prove that by induction on $|\sigma|=k$. If $k=0$, then $\sigma=()$. This is not an expression and we have $\mathrm{count}(\sigma)=0$. Hence both sides of the 'iff' are false, in which case the statement holds. Now assume that $k>0$ and that the statement holds for all the sequences.

  To prove the 'if' part, assume $\sigma$ is one with $\mathrm{count}(\sigma)=-1$ and $\mathrm{count}(\sigma\upharpoonright l)\geq0$ whenever $l<k$. Let $n=\alpha(s_0)$. Note that $\mathrm{count}(\sigma\upharpoonright1)=n-1$ holds. Since the function $\mathrm{count}()$ decreases at most 1 at a step, we have an increasing sequence $1=j_0<j_1<\ldots<j_{n-1}<j_n=k$ of integers such that each $j_i$ is the first one such that $\mathrm{count}(\sigma\upharpoonright j_i)=n-i-1$. Then each consecutive sequence $\tau_i=\sigma\upharpoonright(j_{i+1}\setminus j_i)$ satisfies $\mathrm{count}(\tau_i)=-1$ and $\mathrm{count}(\tau_i\upharpoonright l)\geq0$ whenever $l<|\tau_i|$, which turns out to be a subexpression by the inductive hypothesis. Thus $\sigma=s_0\tau_0\cdots\tau_{n-1}$ is an expression.

  To prove the 'only if' part, assume $\sigma$ is an expression and let $\sigma=s\tau_0\cdots\tau_{n-1}$. Then each $\tau_i$ is an expression shorter than $\sigma$ and we can apply the inductive hypothesis to get $\mathrm{count}(\tau_i)=-1$. Since it holds that $\mathrm{count}(\sigma)=\sum_{j<k}(\alpha(s_j)-1)=n-1+\sum_{i=0}^{n-1}\mathrm{count}(\tau_i)$, we have $\mathrm{count}(\sigma)=-1$. If $\mathrm{count}(\sigma\upharpoonright l)<0$ for some $l<k$, let $l$ be the least one. Then $\mathrm{count}(\sigma\upharpoonright m)\geq0$ holds for $m<l$ and $\mathrm{count}(l)=-1$ holds because the function $\mathrm{count}()$ decreases at most 1 at a step. Then from what we proved in the above (the 'if' part), $\sigma\upharpoonright l$ is an expression, which contradicts Lemma II.4.3. Thus $\mathrm{count}(\sigma\upharpoonright l)\geq0$ holds for all $l<k$.
\end{customthm}

\begin{customthm}{II.7.5}
  We prove that by induction on $\tau$. If $\tau=x\in\dom(\sigma)$, then $V(\tau)=\{x\}$ and it is obvious that $\val_\mathfrak{A}(\tau)[\sigma]=\sigma(x)$ only depends on $\sigma\upharpoonright V(\tau)$. If $\tau=c\in\mathcal{F}_0$, then $\val_\mathfrak{A}(\tau)[\sigma]=c_\mathfrak{A}$ holds for any assignment $\sigma$. Now, let $\tau=f\tau_1\cdots\tau_n$ and let $\sigma,\sigma'$ be assignments such that $\sigma\upharpoonright V(\tau)=\sigma'\upharpoonright V(\tau)$. For each $\tau_i$, since $V(\tau_i)\subseteq V(\tau)$ holds, we have $\val_\mathfrak{A}(\tau_i)[\sigma]=\val_\mathfrak{A}(\tau_i)[\sigma']$ by the inductive hypothesis. Thus we get $\val_\mathfrak{A}(\tau)[\sigma]=f_\mathfrak{A}(\val_\mathfrak{A}(\tau_1)[\sigma],\ldots,\val_\mathfrak{A}(\tau_n)[\sigma])=f_\mathfrak{A}(\val_\mathfrak{A}(\tau_1)[\sigma'],\ldots,\val_\mathfrak{A}(\tau_n)[\sigma'])=\val_\mathfrak{A}(\tau)[\sigma']$.
\end{customthm}

\begin{customthm}{II.7.9}
  We prove that by induction on the number of the logical symbols of $\varphi$. If $\varphi=p\in\mathcal{P}_0$, then $V(\varphi)=\emptyset$ and $\val_\mathfrak{A}(\varphi)[\sigma]=p_\mathfrak{A}$ holds for any assignment $\sigma$. If $\varphi=p\tau_1\cdots\tau_n$ and $\sigma\upharpoonright V(\varphi)=\sigma'\upharpoonright V(\varphi)$, then from Exercise II.7.5, we have $(\val_\mathfrak{A}(\tau_1)[\sigma],\ldots,\val_\mathfrak{A}(\tau_n)[\sigma])=(\val_\mathfrak{A}(\tau_1)[\sigma'],\ldots,\val_\mathfrak{A}(\tau_n)[\sigma'])$ and hence $\val_\mathfrak{A}(\varphi)[\sigma]=\val_\mathfrak{A}(\varphi)[\sigma']$ holds. The case when $\varphi$ is $=\tau_1\tau_2$ is almost the same as when $\varphi=p\tau_1\cdots\tau_n$. If $\varphi=\neg\psi$, the proof is obvious from $V(\varphi)=V(\psi)$ and the inductive hypothesis. If $\varphi$ is either $\wedge\varphi_1\varphi_2$, $\vee\varphi_1\varphi_2$, $\rightarrow\varphi_1\varphi_2$ or $\leftrightarrow\varphi_1\varphi_2$, the proof is done by using $V(\varphi_1)\subseteq V(\varphi)$, $V(\varphi_2)\subseteq V(\varphi)$ and the inductive hypothesis. Now assume $\varphi=\exists y\psi$. Then $V(\varphi)=V(\psi)\setminus\{y\}$ holds. For assignments $\sigma,\sigma'$ in $A$ such that $\sigma\upharpoonright V(\varphi)=\sigma'\upharpoonright V(\varphi)$, it holds that $(\sigma+(y/a))\upharpoonright V(\psi)=(\sigma'+(y/a))\upharpoonright V(\psi)$ because $(\sigma+(y/a))(x)=\sigma(x)=\sigma'(x)=(\sigma'+(y/a))(x)$ holds for all $x\in V(\varphi)\setminus\{y\}$ and $a\in A$. We also have $(\sigma+(y/a))\upharpoonright\{y\}=(\sigma'+(y/a))\upharpoonright\{y\}=a$ for all $a\in A$. Hence we get $\val_\mathfrak{A}(\varphi)[\sigma]=\val_\mathfrak{A}(\varphi)[\sigma']$ from the inductive hypothesis. The same proof goes for $\varphi=\forall y\psi$.
\end{customthm}

\begin{customthm}{II.7.10}
  $\val_\mathfrak{A}(0<x)[2]=T$ iff $(\val_\mathfrak{A}[0](2), \val_\mathfrak{A}[x][2])\in<_\mathfrak{A}\leftrightarrow(0,2)\in\{(x,y)\in\mathbb{Q}^2:x<y\}$ holds, which is true. $\val_\mathfrak{A}(\exists y(x=y\cdot y))[2]=T$ iff $\val_\mathfrak{A}(x=y\cdot y)[\{\langle x,2\rangle\}+y/a]$ holds for some $a\in\mathbb{Q}$, which is equivalent to $\val_\mathfrak{A}(2)[\{\langle x,2\rangle,\langle y,a\rangle\}]=2=a\cdot a=\val_\mathfrak{A}(y\cdot y)[\{\langle x,2\rangle,\langle y,a\rangle\}]$ holds for some $a\in\mathbb{Q}$, which is false.
\end{customthm}

\begin{customthm}{II.7.18}
  Let $\mathcal{L}=\mathcal{P}_0$ be a lexicon for a propositional logic. We identify each function $f:\mathcal{P}_0\rightarrow2$ with a structure $\mathfrak{A}=(1,f)$ for $\mathcal{L}$. That is, $\mathfrak{A}$ is a structure whose underlying set is $1=\{\emptyset\}$ and whose semantic entities are determined by $p_\mathfrak{A}=f(p)$ for each $p\in\mathcal{P}_0$. Note that any structure $\mathfrak{A}'$ for $\mathcal{L}$ is equivalent to a structure $\mathfrak{A}\in2^{\mathcal{P}_0}$ such that $p_\mathfrak{A}=p_{\mathfrak{A}'}$ for each $p\in\mathcal{P}_0$ as long as we only deal with truth values of formulas.

  Let $\Sigma$ be a set of variable-free, quantifier-free and $=$-free sentences in $\mathcal{L}$. For each formula $\varphi\in\Sigma$, $F_\varphi=\{\mathfrak{A}\in2^{\mathcal{P}_0}:\mathfrak{A}\models\varphi\}$ is a closed set. This is because we can map every truth table $A\in2^{V(\varphi)}$ to a closed set of structures $E_A=A\times2^{\mathcal{P}_0\setminus V(\varphi)}$ and we can represent $F_\varphi$ as a finite union of closed sets $F_\varphi=\bigcup\{E_A:\varphi\text{ is true under }A\in2^{V(\varphi)}\}$ since $V(\varphi)$ is finite. Now assume every finite subset of $\Sigma$ is semantically consistent. This is equivalent to saying that every finite subset of $\mathcal{F}=\{F_\varphi:\varphi\in\Sigma\}$ has a non-empty intersection. Then by the compactness of $2^{\mathcal{P}_0}$, $\bigcap\mathcal{F}$ is non-empty, which is equivalent to the statement that the whole $\Sigma$ is semantically consistent.
\end{customthm}

\begin{customthm}{II.7.19}
  Consider a language $\mathcal{L}=\{p_a^j:a\in A,j<4\}$ and 2 sets of sentences $\Sigma_1=\{\neg(p_a^j\wedge p_b^j):a\neq b,aRb,j<4\}$ and $\Sigma_2=\{(p_a^0\wedge\neg p_a^1\wedge\neg p_a^2\wedge\neg p_a^3)\vee(\neg p_a^0\wedge p_a^1\wedge\neg p_a^2\wedge\neg p_a^3)\vee(\neg p_a^0\wedge\neg p_a^1\wedge p_a^2\wedge\neg p_a^3)\vee(\neg p_a^0\wedge\neg p_a^1\wedge\neg p_a^2\wedge p_a^3):a\in A\}$ over $\mathcal{L}$. $\neg(p_a^j\wedge p_b^j)\in\Sigma_1$ says that the 2 connected nodes $a,b$ are given different colors and each sentence in $\Sigma_2$ says the corresponding node is given exactly 1 color. Let $\Sigma'$ be a finite subset of $\Sigma=\Sigma_1\cup\Sigma_2$ and $A'\subseteq A$ be the set of nodes that appear in $\Sigma'$. Since $A'$ is finite, $A'$ is four-colorable by the assumption and hence $\Sigma'$ is semantically consistent. Then by the Propositional Compactness Theorem, the whole $\Sigma$ is satisfiable, which is equivalent to the four-colorability of $A$.
\end{customthm}

\begin{customthm}{II.7.20}
  Let $A$ be a set. Consider a language $\mathcal{L}=\{p_{a,b}:a,b\in A\}$ and 3 sets of sentences $\Sigma_1=\{\neg p_{a,a}:a\in A\}$, $\Sigma_2=\{p_{a,b}\wedge p_{b,c}\rightarrow p_{a,c}:a,b,c\in A\}$ and  $\Sigma_3=\{p_{a,b}\vee p_{b,a}:a,b\in A,a\neq b\}$ over $\mathcal{L}$. They correspond to irreflexivity, transitivity and trichotomy respectively. Let $\Sigma'$ be a finite subset of $\Sigma=\Sigma_1\cup\Sigma_2\cup\Sigma_3$ and $A'\subseteq A$ be the set of elements that appear in $\Sigma'$. Since $A'$ is finite, $A'$ can be totally ordered by using an injection from $A'$ to some natural number. Therefore $\Sigma'$ is semantically consistent. By the Propositional Compactness Theorem, the whole $\Sigma$ is satisfiable, which means $A$ can be totally ordered.
\end{customthm}

\begin{customthm}{II.7.21}
  Let $\psi_n(y)$ be a formula over $\mathcal{L}=\{<\}$ defined recursively by $\psi_0(y)=\psi_1(y)=\exists x\exists y(x=y)$ and $\psi_n(y)=\exists x(\psi_{n-1}(x)\wedge x<y)$ for $n\geq2$. For $n\geq2$, $\psi_n(y)$ says there are at least $n-1$ elements below $y$. Therefore, for each $n\geq2$, $\varphi_n=\exists y\psi_n(y)$ says there are at least $n$ elements. This is what we want. Note that for $n\leq1$, $\varphi_n$ is true under any structure $\mathfrak{A}=(A,<)$ so we can say $\mathfrak{A}\models\varphi_n$ iff $|A|\geq n$, because $|A|\geq 1$ always holds too.
\end{customthm}

\begin{customthm}{II.8.5}
  Let $\tau_1,\tau_2$ be terms and $S$ be the set of symbols that appear in at least one of $\tau_1,\tau_2$. Since $S$ is a finite set, we have an injection $i:S\rightarrow\omega$. We fix such an $i$ and define a structure $\mathfrak{A}=(HF,\mathcal{I})$ where $\mathcal{I}(f)(x_1,\ldots,x_n)=(i(f),x_1,\ldots,x_n)$, and also define an assignment $\sigma$ by $\sigma(x)=i(x)$. Then we can show that $\val_\mathfrak{A}(\tau_1)[\sigma]=\val_\mathfrak{A}(\tau_2)[\sigma]$ implies $\tau_1=\tau_2$ by induction on the depth of $\tau_1$. Here the depth $d$ is recursively defined as $d(x)=0$ when $x\in VAR$, $d(c)=0$ when $c$ is a constant symbol, and $d(f\tau_1,\ldots,\tau_n)=1+\max(d(\tau_1),\ldots,d(\tau_n))$.
\end{customthm}

\begin{customthm}{II.8.20}
  Assume $\Sigma$ is complete. Then there exists a structure $\mathfrak{A}=(A,\mathcal{I})$ for $\mathcal{L}$ such that $\mathfrak{A}\models\Sigma$. Let $p\in\mathcal{L}'\setminus\mathcal{L}$ be a predicate of arity $n$ and we define structures $\mathfrak{A}_1,\mathfrak{A}_2$ for $\mathcal{L}'$ by adding $p_{\mathfrak{A}_1}=\emptyset$ and $p_{\mathfrak{A}_2}=A^{n}$ to $\mathfrak{A}$ respectively. Then we have $\mathfrak{A}_1\models\varphi$ and $\mathfrak{A}_2\models\neg\varphi$ for $\varphi=\exists x_1,\ldots,x_n\,p(x_1,\ldots,x_n)$, which contradicts $\Sigma$'s completeness. Hence $\Sigma$ cannot be complete.
\end{customthm}

\begin{customthm}{II.8.25}
  A $w$ in $\varphi(w)$ is either a bound $w$ in $\varphi(z)$ or one that originally is a $z$ in $\varphi(z)$ which ends up being substituted by $w$. Therefore a free occurrence of $w$ in $\varphi(w)$ must belong to the second case and it cannot be inside the scope of $\forall z$ nor $\exists z$ because the $z$ that the $w$ would originate from would be bound in $\varphi(z)$, which would not be replaced. Hence $z$ is free for $w$ in $\varphi(w)$. It is also clear that there is no free occurrence of $z$ in $\varphi(w)$ because all free $z$'s are supposed to be replaced by the substitution.

  We show that $\forall z\varphi(z)\rightarrow\forall w\varphi(w)$ is logically valid. Let $\mathfrak{A}=(A,\mathcal{I})$ be a structure for $\mathcal{L}$ and $\sigma$ be an assignment for $\varphi$. $\mathfrak{A}\models(\forall z\varphi(z)\rightarrow\forall w\varphi(w))[\sigma]$ is equivalent to having $\mathfrak{A}\not\models(\forall z\varphi(z))[\sigma]$ or $\mathfrak{A}\models(\forall w\varphi(w))[\sigma]$, so we assume $\mathfrak{A}\models(\forall z\varphi(z))[\sigma]$ and prove $\mathfrak{A}\models(\forall w\varphi(w))[\sigma]$. Since $w$ does not occur free in $\varphi(z)$, it does not in $\forall z\varphi(z)$ either. Hence by Exercise II.7.9, we have $\mathfrak{A}\models(\forall z\varphi(z))[\sigma+(w/a)]$ for any $a\in A$. In addition, $\forall z\varphi(z)\rightarrow\varphi(w)$ is logically valid from Corollary II.8.12 and we get $\mathfrak{A}\models\varphi(w)[\sigma+(w/a)]$. Since $a$ is arbitrary, $\mathfrak{A}\models(\forall w\varphi(w))[\sigma]$ has been proved.

  $\forall w\varphi(w)\rightarrow\forall z\varphi(z)$ is logically valid as well because what is assumed for $z,w,\varphi(w)$ is exactly what we assume for $w,z,\varphi(z)$ respectively, which is proved in the first paragraph. Hence $\forall w\varphi(w)$ and $\forall z\varphi(z)$ are logically equivalent. The logical equivalence of $\exists z\varphi(z)$ and $\exists w\varphi(w)$ can be shown likewise.
\end{customthm}

\begin{customthm}{II.8.26}
  We show that by induction on $\varphi$. The basis is trivial because $\varphi'$ will be syntactically the same as $\varphi$ since there will be no bound occurrence of $z$. For the step cases, let $\varphi_1',\varphi_2'$ be formulas which result from replacing all bound occurrences of $z$ in $\varphi_1,\varphi_2$ by $w$, respectively, and assume that $\varphi_1\leftrightarrow\varphi_1'$ and $\varphi_2\leftrightarrow\varphi_2'$ hold. Then for $\varphi=\varphi_1\wedge\varphi_2$, we have $\varphi'=(\varphi_1\wedge\varphi_2)'=\varphi_1'\wedge\varphi_2'\leftrightarrow\varphi_1\wedge\varphi_2=\varphi$. The cases when $\varphi=\neg\varphi_1$, $\varphi=\varphi_1\vee\varphi_2$, $\varphi=\varphi_1\rightarrow\varphi_2$ or $\varphi=\varphi_1\leftrightarrow\varphi_2$ can be proved similarly. If $\varphi=\forall z\varphi_1(z)$, it holds that $\varphi=\forall w\varphi_1'(w)\leftrightarrow\forall z\varphi_1'(z)\leftrightarrow\forall z\varphi_1(z)=\varphi$, where the first equivalence comes from Exercise II.8.25 and the second one is clear from the inductive hypothesis. The case when $\varphi=\exists z\varphi_1(z)$ can be proved similarly. Thus $\varphi$ and $\varphi'$ are logically equivalent.
\end{customthm}

\begin{customthm}{II.9.3}
  Let $\mathcal{L}$ be a language. Let $\varphi$ be a formula of $\mathcal{L}$ for which we can consider a truth assignment. That is, $\varphi$ is a formula constructed recursively from basic formulas by applying $\neg$, $\wedge$, $\vee$, $\rightarrow$ or $\leftrightarrow$. Let $\mathfrak{A}$ be a structure for $\mathcal{L}$ and let $\sigma$ be an assignment for $\varphi$. We define a truth assignment $v_{\mathfrak{A};\sigma}$ by $v_{\mathfrak{A};\sigma}(\psi)=\val_\mathfrak{A}(\psi)[\sigma]$ for a basic formula $\psi$. Since Definition II.9.2 is consistent with the definition of $\val$, we can show that $\overline{v_{\mathfrak{A};\sigma}}(\varphi)=\val_\mathfrak{A}(\varphi)[\sigma]$ by induction. Now assume that $\varphi$ is a propositional tautology. Then $\overline{v_{\mathfrak{A};\sigma}}(\varphi)=T$ holds and so does $\val_\mathfrak{A}(\varphi)[\sigma]=T$. Since $\mathfrak{A}$ and $\sigma$ are arbitrary, $\varphi$ turns out to be logically valid.
\end{customthm}

\begin{customthm}{II.10.2}
  The validity of the universal closure of a formula is equivalent to the validity of that formula. Therefore, for each formula $\varphi$ of the 11 axioms, we fix an arbitrary structure $\mathfrak{A}=(A,\mathcal{I})$ and an arbitrary assignment $\sigma$ for $\varphi$ to show $\mathfrak{A}\models\varphi[\sigma]$.
  \begin{enumerate}
    \item Already done in Exercise II.9.3.
    \item Since $\val_\mathfrak{A}(\varphi)[\sigma]$ only depends on $\sigma\upharpoonright V(\varphi)$ (see Exercise II.7.9), which does not contain $x$, we have $\val_\mathfrak{A}(\forall x\varphi)[\sigma]=\val_\mathfrak{A}(\varphi)[\sigma]$.
    \item Assume $\mathfrak{A}\models(\forall x\varphi)[\sigma]$, $\mathfrak{A}\models(\forall x(\varphi\rightarrow\psi))[\sigma]$ and let $a\in A$. From the first assumption, we have $\mathfrak{A}\models\varphi[\sigma+(x/a)]$. From the second assumption, we can say that $\mathfrak{A}\models\varphi[\sigma+(x/a)]$ implies $\mathfrak{A}\models\psi[\sigma+(x/a)]$. Thus we have $\mathfrak{A}\models\varphi[\sigma+(x/a)]$ for an arbitrary $a\in A$, which proves $\mathfrak{A}\models(\forall x\psi)[\sigma]$.
    \item See Corollary II.8.12.
    \item See Corollary II.8.12.
    \item $\mathfrak{A}\models(\forall x\neg\varphi)[\sigma]$ iff $\mathfrak{A}\models(\neg\varphi)[\sigma+(x/a)]$ for all $a\in A$, iff $\mathfrak{A}\not\models\varphi[\sigma+(x/a)]$ for all $a\in A$, iff not $\mathfrak{A}\models\varphi[\sigma+(x/a)]$ for any $a\in A$, iff not ($\mathfrak{A}\models\varphi[\sigma+(x/a)]$ for some $a\in A$), iff not ($\mathfrak{A}\models(\exists x\varphi)[\sigma]$), iff $\mathfrak{A}\models(\neg\exists x\varphi)[\sigma]$.
    \item Obvious from the reflexivity of '$=$'.
    \item Obvious from the symmetry of '$=$'.
    \item Obvious from the transitivity of '$=$'.
    \item Assume $\sigma(x_i)=\sigma(y_i)$ for each $i=1,\ldots,n$. Then we have $\val_\mathfrak{A}(f(x_1,\ldots,x_n))[\sigma]=f_\mathfrak{A}(\sigma(x_1),\ldots,\sigma(x_n))=f_\mathfrak{A}(\sigma(y_1),\ldots,\sigma(y_n))=\val_\mathfrak{A}(f(y_1,\ldots,y_n))[\sigma]$.
    \item Almost the same as 10.
  \end{enumerate}
\end{customthm}

\begin{customthm}{II.10.6}
  Below is the proof.
  \begin{center}
    \begin{tabular}{lll}
      0. & $\forall x(x=x\rightarrow\exists x(x=x))$ & type 5 axiom \\
      1. & $\forall x(x=x\rightarrow\exists x(x=x))$ & \\
      & $\rightarrow(\forall x(x=x)\rightarrow\forall x\exists x(x=x))$ & type 3 axiom \\
      2. & $\forall x(x=x)\rightarrow\forall x\exists x(x=x)$ & 0, 1, modus ponens \\
      3. & $\forall x(x=x)$ & type 7 axiom \\
      4. & $\forall x\exists x(x=x)$ & 3, 2, modus ponens \\
      5. & $\forall x\exists x(x=x)\rightarrow\exists x(x=x)$ & type 4 axiom \\
      6. & $\exists x(x=x)$ & 4, 5, modus ponens
    \end{tabular}
  \end{center}
\end{customthm}

\begin{customthm}{II.11.10}
  Below is the proof.
  \begin{center}
    \begin{tabular}{lll}
      1. & $\emptyset\vdash_\mathcal{L}\forall x\neg\neg\psi(x)\leftrightarrow\neg\exists x\neg\psi(x)$ & type 6 axiom \\
      2. & $\psi(c)\vdash_{\mathcal{L}'}\neg\neg\psi(c)$ & tautology \\
      3. & $\forall x\psi(x)\vdash_{\mathcal{L}'}\neg\neg\psi(c)$ & 2, UI \\
      4. & $\forall x\psi(x)\vdash_\mathcal{L}\forall x\neg\neg\psi(x)$ & 3, UG \\
      5. & $\emptyset\vdash_\mathcal{L}\forall x\psi(x)\rightarrow\forall x\neg\neg\psi(x)$ & 4, Deduction Theorem \\
      6. & $\emptyset\vdash_\mathcal{L}\exists x\neg\psi(x)\rightarrow\neg\forall x\psi(x)$ & 1, 5, tautology \\
      7. & $\emptyset\vdash_\mathcal{L}\exists x\neg\psi(x)\leftrightarrow\neg\forall x\psi(x)$ & Example II.11.9, tautology
    \end{tabular}
  \end{center}
\end{customthm}

\begin{customthm}{II.11.11}
  Below is the proof. Here $GP=\{\gamma_1,\gamma_2\}$, $\gamma_1=\forall xyz[x\cdot yz=xy\cdot z]$ and $\gamma_2=\exists u[\forall x[xu=ux=x]\wedge\forall x\exists y[xy=yx=u]]$ as defined in the textbook.
  \begin{center}
    \begin{tabularx}{\textwidth}{lX}
      0. & $\{\gamma_1,\forall x[xe=ex=x],ad=da=e,ab=ac\}\vdash_{\mathcal{L}'''}$ \\
      & $b=eb=da\cdot b=d\cdot ab=d\cdot ac=da\cdot c=ec=c$ \hfill UI, tautology, type 10 axiom \\
      1. & $\{\gamma_1,\forall x[xe=ex=x],\exists y[ay=ya=e],ab=ac\}\vdash_{\mathcal{L}''}b=c$ \hfill 0, EI, tautology \\
      2. & $\{\gamma_1,\forall x[xe=ex=x],\forall x\exists y[xy=yx=e],ab=ac\}\vdash_{\mathcal{L}''}b=c$ \\
      & \hfill 1, Deduction Theorem, type 4 axiom, tautology \\
      3. & $GP\cup\{ab=ac\}\vdash_{\mathcal{L}'}b=c$ \hfill 2, Deduction Theorem, tautology \\
      4. & $GP\vdash_\mathcal{L}\forall xyz[xy=xz\rightarrow y=z]$ \hfill 3, Deduction Theorem, UG
    \end{tabularx}
  \end{center}
\end{customthm}

\begin{customthm}{II.11.13}
  Let $\varphi_0,\ldots,\varphi_n$ be a formal proof of $\varphi$ from $\Sigma$ in $\mathcal{L}_1$ and let $\mathcal{L}_0'=\mathcal{L}_0\cup\{c\}$ where $c\not\in\mathcal{L}_0$ is a constant symbol. Let $F$ be a function that turns any formula $\varphi\in\mathcal{L}_1$ into a formula of $\mathcal{L}_0'$ by replacing all atomic formulas $p(\tau_1,\ldots,\tau_m)$ in $\varphi$ with $p\in\mathcal{L}_1\setminus\mathcal{L}_0$ by $\forall x(x=x)$, and then replacing all terms $f(\tau_1,\ldots,\tau_m)$ in $\varphi$ with $f\in\mathcal{L}_1\setminus\mathcal{L}_0$ by $c$.

  We prove $\Sigma\vdash_{\mathcal{L}_0'}F(\varphi_i)$ by induction. Assume we have $\Sigma\vdash_{\mathcal{L}_0'}F(\varphi_j)$ for all $j<i$. If $\varphi_i$ is in $\Sigma$, $F(\varphi_i)$ coincides with $\varphi_i$ and $\Sigma\vdash_{\mathcal{L}_0'}F(\varphi_i)$ is obvious. If $\varphi_i$ is obtained from $\varphi_j$ and $\varphi_k=\varphi_j\rightarrow\varphi_i$ by Modus Ponens, $\Sigma\vdash_{\mathcal{L}_0'}F(\varphi_i)$ can be obtained from $F(\varphi_j)$ and $F(\varphi_k)$ by Modus Ponens since $F(\varphi_k)=F(\varphi_j)\rightarrow F(\varphi_i)$ holds. Now assume $\varphi_i$ is a logical axiom. If $\varphi_i$ is a propositional tautology, so is $F(\varphi_i)$. This can be easily shown by induction and the homomorphic property of $F$ such as $F(\psi\wedge\chi)=F(\psi)\wedge F(\chi)$. If $\varphi_i$ is a universal closure of $\psi\rightarrow\forall x\psi$ (that is, it is a type 2 axiom), then $F(\varphi_i)$ is a universal closure of $F(\psi)\rightarrow\forall xF(\psi)$, which is a type 2 axiom too. Likewise, if $\varphi_i$ is an axiom of type 3 or 6, $F(\varphi_i)$ is an axiom of the same type as $\varphi_i$. If $\varphi_i$ is an axiom of type 4 or type 5, considering $F(\psi(x\rightsquigarrow\tau))=F(\psi)(x\rightsquigarrow\tau')$ where $\tau'$ is another term which is free for $x$ in $\psi$, $F(\varphi_i)$ is an axiom of the same type as $\varphi_i$. If $\varphi_i$ is an axiom of type 7, 8 or 9, then $F(\varphi_i)=\varphi_i$ holds and it is obvious that $F(\varphi_i)$ is an axiom of the same type as $\varphi_i$. If $\varphi_i$ is a type 10 axiom, let it be a universal closure of $x_1=y_1\wedge\ldots\wedge x_n=y_n\rightarrow(f(x_1,\ldots,x_n)=f(y_1,\ldots,y_n))$. If $f\in\mathcal{L}_0$, we have $F(\varphi_i)=\varphi_i$ and $\Sigma\vdash_{\mathcal{L}_0'}F(\varphi_i)$ is obvious. If not, $F(\varphi_i)$ is a universal closure of $x_1=y_1\wedge\ldots\wedge x_n=y_n\rightarrow c=c$, which is derived as follows.
  \begin{center}
    \begin{tabular}{lll}
      0. & $c=c$ & type 10 axiom \\
      1. & $c=c\rightarrow(a_1=b_1\wedge\ldots\wedge a_n=b_n\rightarrow c=c)$ & tautology \\
      2. & $a_1=b_1\wedge\ldots\wedge a_n=b_n\rightarrow c=c$ & 0, 1, modus ponens \\
      3. & $F(\varphi_i)$ & 2, UG
    \end{tabular}
  \end{center}
  The case of type 11 axioms can be shown in a similar way. Thus we have proved $\Sigma\vdash_{\mathcal{L}_0'}F(\varphi)$. The rest follows from Lemma II.11.12.
\end{customthm}

\begin{customthm}{II.11.14}
  Below is the proof.
  \begin{center}
    \begin{tabular}{lll}
      0. & $\emptyset\vdash\forall xyz(z=z\wedge x=y\rightarrow(z\in x\leftrightarrow z\in y))$ & type 11 axiom \\
      1. & $\emptyset\vdash c=c\wedge a=b\rightarrow(c\in a\leftrightarrow c\in b)$ & 0, UI \\
      2. & $\emptyset\vdash c=c\rightarrow(a=b\rightarrow(c\in a\leftrightarrow c\in b))$ & 1, tautology \\
      3. & $\emptyset\vdash a=b\rightarrow(c\in a\leftrightarrow c\in b)$ & type 10 axiom, 2 \\
      4. & $\emptyset\vdash\forall xy(x=y\rightarrow\forall z(z\in x\leftrightarrow z\in y))$ & 3, UG
    \end{tabular}
  \end{center}
\end{customthm}

\begin{customthm}{II.11.15}
  Below is the proof. Here $SE(z)=[z=z]$ and $C(x,y,z)=[x\in y\leftrightarrow x\in z\wedge z\neq z]$. Note that $\exists xSE(x)$ is the Set Existence axiom and $\forall z\exists y\forall xC(x,y,z)$ is a Comprehension axiom.
  \begin{center}
    \begin{tabular}{lll}
      0. & $ZF\cup\{SE(c),\forall xC(x,b,c)\}\vdash c=c$ & $SE(c)$ \\
      1. & $ZF\cup\{SE(c),\forall xC(x,b,c)\}\vdash c=c\rightarrow a\not\in b$ & $\forall xC(x,b,c)$, UI, tautology \\
      2. & $ZF\cup\{SE(c),\forall xC(x,b,c)\}\vdash a\not\in b$ & 0, 1, modus ponens \\
      3. & $ZF\cup\{SE(c),\forall xC(x,b,c)\}\vdash\exists y\forall x(x\not\in y)$ & 2, UG, EG \\
      4. & $ZF\cup\{\exists y\forall xC(x,y,c)\}\vdash\exists y\forall x(x\not\in y)$ & 3, EI \\
      5. & $ZF\vdash\exists y\forall x(x\not\in y)$ & type 4 axiom, 4, tautology, \\
      && Deduction Theorem
    \end{tabular}
  \end{center}
\end{customthm}

\begin{customthm}{II.11.16}
  Let $\varphi=\forall y\exists xp(x,y)\rightarrow\exists x\forall yp(x,y)$. Let $\mathfrak{A}=(2,\mathcal{I})$ be a structure for $\mathcal{L}=\{p\}$ where $p_\mathfrak{A}(x,y)=\{(0,0),(1,1)\}$. Then we have $\val_\mathfrak{A}[\forall y\exists xp(x,y)]=T$ and $\val_\mathfrak{A}[\exists x\forall yp(x,y)]=F$, which means $\mathfrak{A}\not\models\varphi$. Thus $\emptyset\not\models\varphi$ holds, which implies $\emptyset\not\vdash\varphi$ by Soundness.
\end{customthm}

\begin{customthm}{II.12.23}
  If $\Sigma\vdash\theta$, then $\Sigma'\vdash\theta$ is obvious. If $\Sigma\not\vdash\theta$, then $\widetilde\Sigma=\Sigma\cup\{\neg\theta\}$ is consistent from Proof by Contradiction. We apply the proof of Lemma II.12.20 to $\widetilde\Sigma$ to get a consistent set of sentences $(\widetilde\Sigma)'=\Sigma'\cup\{\neg\theta\}$, which implies $\Sigma'\not\vdash\theta$.
\end{customthm}

\begin{customthm}{II.13.8}
  Let $A$ be a model of $\Sigma$. By defining a scalar multiplication $m:\mathbb{Z}_p\times A\rightarrow A$ by $m(n,x)=x+x+\cdots+x$ with $n$ copies of $x$, we can view $A$ as a vector space over $\mathbb{Z}_p$. Now let $B$ be a basis of $A$ and assume $A$ is infinite. Then $B$ is infinite as well because a finite basis (say $|B|=n$) would generate only $p^n$ elements. Then we can apply Exercise I.13.21 to get $|A|=|B|$. Hence infinite models of the same cardinality have bases of the same cardinality, which proves those models are isomorphic. Thus $\Sigma$ is $\kappa$-categorical for all $\kappa\geq\aleph_0$.
\end{customthm}

\begin{customthm}{II.13.10}
  We first prove that $\Delta^{(~)}$ is $\aleph_0$-categorical. Let $A=\{a_i:i<\omega\}$, $B=\{b_j:j<\omega\}$ be countable models of $\Delta^{(~)}$. We recursively define $p_k\in A\times B$ and $f_k=\bigcup_{l<k}p_l$ for $k<\omega$ as follows.
  \begin{align*}
    p_0 &= (a_0, b_0) \\
    p_k &= (a_i, b_j) \text{ for } k>0 \text{, where} \\
    &\text{if } k \text{ is odd, } i=\min\{i<\omega:a_i\not\in\dom(f_k)\} \text{, } j=\min\{j<\omega:\psi_j\} \text{ where} \\
    &\psi_j=
    \begin{cases}
      b_j<\min(\ran(f_k)) &\text{ if } a_i<\min(\dom(f_k)) \\
      b_j>\max(\ran(f_k)) &\text{ if } a_i>\max(\dom(f_k)) \\
      f_k(a_i^-)<b_j<f_k(a_i^+) &\text{ otherwise, where } \\
      &a_i^-=\max\{a\in\dom(f_k):a<a_i\}, \\ &a_i^+=\min\{a\in\dom(f_k):a>a_i\}
    \end{cases} \\
    &\text{if } k \text{ is even, } j=\min\{j<\omega:b_j\not\in\ran(f_k)\} \text{, } i=\min\{i<\omega:\varphi_i\} \text{ where} \\
    &\varphi_i=
    \begin{cases}
      a_i<\min(\dom(f_k)) &\text{ if } b_j<\min(\ran(f_k)) \\
      a_i>\max(\dom(f_k)) &\text{ if } b_j>\max(\ran(f_k)) \\
      f_k^{-1}(b_j^-)<a_i<f_k^{-1}(b_j^+) &\text{ otherwise, where } \\
      &b_j^-=\max\{b\in\ran(f_k):b<b_j\}, \\ &b_j^+=\min\{b\in\ran(f_k):b>b_j\}
    \end{cases}
  \end{align*}
  It is easy to see that $\bigcup_kp_k$ is a function from $A$ to $B$ which is bijective and order-preserving. Thus $A\cong B$ holds and $\Delta^{(~)}$ proves to be $\aleph_0$-categorical. For each of the other 3 theories, one can think of the correspondence of the endpoints first and reduce the problem to the case of $\Delta^{(~)}$ to prove that it is also $\aleph_0$-categorical.

  Now we prove that $\Delta^{(~)}$ is not $2^{\aleph_0}$-categorical and use the Morley's theorem mentioned in the textbook to claim that it is not $\aleph_1$-categorical. Let $A=((-1,0]\cap\mathbb{Q})\cup[0,1)$ and $B=(-1,0]\cup([0,1)\cap\mathbb{Q})$. Both of these have cardinality $2^{\aleph_0}$ and are models of $\Delta^{(~)}$, but are not isomorphic. To prove this, we assume there is an isomorphism $f:A\rightarrow B$ and consider $A_0=\{x\in A:x<0\}$ and $B_0=\{y\in B:y<f(0)\}$. $A_0$ and $B_0$ have to be isomorphic but $|A_0|=\aleph_0\neq2^{\aleph_0}=|B_0|$ holds, which is a contradiction. Hence $A$ and $B$ are not isomorphic and $\Delta^{(~)}$ is not $2^{\aleph_0}$-categorical. The other 3 theories can be proved not to be $\aleph_1$-categorical in a similar way.
\end{customthm}

\begin{customthm}{II.13.11}
  We first prove the completeness of $\Sigma$. Let $\Sigma_k$ be the reduct of $\Sigma$ to  $\{<,c_0,c_i,\ldots,c_k\}$. Then each $\Sigma_k$ is $\aleph_0$-categorical because we can always make an isomorphism between 2 models $\mathfrak{A}$, $\mathfrak{B}$ of $\Sigma_k$ by considering an isomorphism between each pair of corresponding open intervals $({c_{i}}_\mathfrak{A},{c_{i+1}}_\mathfrak{A})$ and $({c_{i}}_\mathfrak{B},{c_{i+1}}_\mathfrak{B})$, the existence of which is assured by the fact that $\Delta^{(~)}$ is $\aleph_0$-categorical (see Exercise II.13.10). Therefore $\Sigma_k$ is complete by Łoś–Vaught test. Let $\varphi$ be a formula of $\mathcal{L}$ and $k=\max\{i\in\omega:c_i \text{ appears in } \varphi\}$. If no $c_i$ appear in $\varphi$, then let $k=0$. Since $\Sigma_k$ is complete, we have either $\Sigma_k\models\varphi$ or $\Sigma_k\models\neg\varphi$. If $\Sigma_k\models\varphi$ holds, then $\Sigma\models\varphi$ holds. This is obvious from $\Sigma_k\subseteq\Sigma$ and the definition of $\models$. If $\Sigma_k\models\neg\varphi$ holds, then $\Sigma\models\neg\varphi$ holds. Thus we have either one of $\Sigma\models\varphi$ or $\Sigma\models\neg\varphi$ for any $\varphi$, which shows the completeness of $\Sigma$.

  To see that $\Sigma$ has precisely three models of size $\aleph_0$, we identify a model of $\Sigma$ with $\mathbb{Q}$, which is possible because $\Delta^{(~)}$ is $\aleph_0$-categorical and $\mathbb{Q}$ is a model of it. We also identify each constant symbol $c_i$ with a point $c_i$ in $\mathbb{Q}$. Here we consider the following 3 cases; (i) $\sup_n c_n=\infty$, (ii) $\sup_n c_n\in\mathbb{Q}$ and (iii) $\sup_n c_n\in\mathbb{R}\setminus\mathbb{Q}$. Models within each case are isomorphic by considering correspondences between intervals as we did above. However, models of different cases are not isomorphic since $\{x\in\mathbb{Q}:c_i\leq x \text{ for all } i\in\omega\}$ cannot be isomorphic because it is empty in (i), it has a minimum in (ii), and it is not empty and does not have a minimum in (iii). Thus $\Sigma$ has precisely 3 countable models.
\end{customthm}

\begin{customthm}{II.13.12}
  $\Sigma$ is complete just as $\Sigma$ in Exercise II.13.11 is complete. We first show that $\Sigma$ has at most $2^{\aleph_0}$ models. Note that the underlying set of a model can be viewed as $\omega$. Then we have $2^{|\omega\times\omega|}=2^{\aleph_0}$ possible interpretations of $<$ and $|\omega|^{|\mathbb{Q}|}=2^{\aleph_0}$ possible interpretations of $\{c_q:q\in\mathbb{Q}\}$, ignoring the sentences in $\Sigma$. Therefore we have $2^{\aleph_0}\times2^{\aleph_0}=2^{\aleph_0}$ possible interpretations of $\mathcal{L}$, which implies there are at most $2^{\aleph_0}$ models of $\Sigma$. To prove that $\Sigma$ has at least $2^{\aleph_0}$ countable models, we show $2^{\aleph_0}$ different models below. For any $r\in\mathbb{R}\setminus\mathbb{Q}$, we define a model $A_r$ of $\Sigma$ by $A_r=\mathbb{Q}\cup\{r\}$ where each $c_q\in\mathbb{Q}$ is interpreted as $q\in A_r$. Then for any 2 different irrationals $r$ and $r'$, $A_r$ and $A_{r'}$ are not isomorphic because any isomorphism $f:A_r\rightarrow A_{r'}$ would map $r$ to $r'$ since it would have to be identical on $\mathbb{Q}$, which would invert the order between $r$ and a rational number which lies between $r$ and $r'$. Thus by Schröder–Bernstein theorem, we conclude that $\Sigma$ has exactly $2^{\aleph_0}$ countable models.
\end{customthm}

\begin{customthm}{II.14.3}
  We first show that for any atomic formula $\varphi$ and any assignment $\sigma$, $\mathfrak{A}\times\mathfrak{B}\models\varphi[\sigma]$ is equivalent to $\mathfrak{A}\models\varphi[\sigma]$ and $\mathfrak{B}\models\varphi[\sigma]$. This is proved as follows. Here we assume $\varphi$ is of the form $p(\tau_1,\ldots,\tau_n)$, but the proof also works when it is $\tau_1=\tau_2$.
  \begin{align*}
    &\mathfrak{A}\times\mathfrak{B}\models\varphi[\sigma] \\
    \leftrightarrow~&\textstyle\val_{\mathfrak{A}\times\mathfrak{B}}(\varphi)[\sigma]=T \\
    \leftrightarrow~&\textstyle(\val_{\mathfrak{A}\times\mathfrak{B}}(\tau_1)[\sigma],\ldots,\val_{\mathfrak{A}\times\mathfrak{B}}(\tau_n)[\sigma])\in p_{\mathfrak{A}\times\mathfrak{B}} \\
    \leftrightarrow~&\textstyle((\val_\mathfrak{A}(\tau_1)[\sigma],\val_\mathfrak{B}(\tau_1)[\sigma]),\ldots,(\val_\mathfrak{A}(\tau_n)[\sigma],\val_\mathfrak{B}(\tau_n)[\sigma]))\in p_{\mathfrak{A}\times\mathfrak{B}} \\
    \leftrightarrow~&\textstyle(\val_\mathfrak{A}(\tau_1)[\sigma],\ldots,\val_\mathfrak{A}(\tau_n)[\sigma])\in p_\mathfrak{A}\mbox{ and }(\val_\mathfrak{B}(\tau_1)[\sigma],\ldots,\val_\mathfrak{B}(\tau_n)[\sigma])\in p_\mathfrak{B} \\
    \leftrightarrow~&\textstyle\val_\mathfrak{A}(\varphi)[\sigma]=T\mbox{ and }\val_\mathfrak{B}(\varphi)[\sigma]=T \\
    \leftrightarrow~&\mathfrak{A}\models\varphi[\sigma]\mbox{ and }\mathfrak{B}\models\varphi[\sigma] \\
  \end{align*}
  Here the fourth equivalence comes from the definition of $p_{\mathfrak{A}\times\mathfrak{B}}$, and the third equivalence comes from $\val_{\mathfrak{A}\times\mathfrak{B}}(\tau)[\sigma]=(\val_\mathfrak{A}(\tau)[\sigma],\val_\mathfrak{B}(\tau)[\sigma])$, which can be easily proved by the definition of $\val_{\mathfrak{A}\times\mathfrak{B}}(\tau)[\sigma]$ and induction on $\tau$.

  Now let $\psi=\varphi_1\vee\cdots\vee\varphi_n$ be a Horn clause and $\sigma$ be an assignment for $\psi$. Then it holds that $\mathfrak{A}\models\psi[\sigma]$ and $\mathfrak{B}\models\psi[\sigma]$ imply $\mathfrak{A}\times\mathfrak{B}\models\psi[\sigma]$. To prove that, assume that $\mathfrak{A}\models\psi[\sigma]$ and $\mathfrak{B}\models\psi[\sigma]$. If there is a negative literal $\varphi_i$ that is true either under $\mathfrak{A}$ or $\mathfrak{B}$, then we have $\mathfrak{A}\times\mathfrak{B}\not\models\neg\varphi_i[\sigma]$ by what is proved above, which means $\mathfrak{A}\times\mathfrak{B}\models\psi[\sigma]$. If not, there is a positive literal $\varphi_i$ that is true under both $\mathfrak{A}$ and $\mathfrak{B}$. Then we have $\mathfrak{A}\times\mathfrak{B}\models\varphi_i[\sigma]$ by what is proved above, which means $\mathfrak{A}\times\mathfrak{B}\models\psi[\sigma]$. Therefore in any case, $\mathfrak{A}\models\psi[\sigma]$ and $\mathfrak{B}\models\psi[\sigma]$ imply $\mathfrak{A}\times\mathfrak{B}\models\psi[\sigma]$. By considering a closure of $\psi$, we can conclude that $\mathfrak{A}\models\chi$ and $\mathfrak{B}\models\chi$ imply $\mathfrak{A}\times\mathfrak{B}\models\chi$ for any $\chi$ which is a closure of a Horn clause. Then if $\Sigma$ is universal Horn, $\mathfrak{A}\models\Sigma$ and $\mathfrak{B}\models\Sigma$, we have $\mathfrak{A}\times\mathfrak{B}\models\Sigma$.
\end{customthm}

\begin{customthm}{II.14.4}
  Let $\mathcal{L}$ be a language without predicate symbols. Let $\mathfrak{A}=(A,\mathcal{I})$, $\mathfrak{B}=(B,\mathcal{J})$ be structures for $\mathcal{L}$, and let $\Phi:\mathfrak{A}\rightarrow\mathfrak{B}$ be a surjective homomorphism. That is, $\Phi$ is one that $\Phi(f_\mathfrak{A}(a_1,\ldots,a_n))=f_\mathfrak{B}(\Phi(a_1),\ldots,\Phi(a_n))$ holds for any $n$-ary function symbol $f\in\mathcal{L}$ and $a_1,\ldots,a_n\in A$. One can easily show that $\Phi(\val_\mathfrak{A}(\tau)[\sigma])=\val_\mathfrak{B}(\tau)[\Phi\circ\sigma]$ holds for any term $\tau$ and any assignment $\sigma$ for $\tau$ by induction on $\tau$. Then it directly follows from this that $\mathfrak{A}\models(\tau_1=\tau_2)[\sigma]$ implies $\mathfrak{B}\models(\tau_1=\tau_2)[\Phi\circ\sigma]$. Then for a closure $\varphi$ of any equation, it holds that $\mathfrak{A}\models\varphi\rightarrow\mathfrak{B}\models\varphi$, which follows from the previous fact and that $\Phi$ is surjective. Thus if $\Sigma$ is an equational theory and $\mathfrak{A}\models\Sigma$, we have $\mathfrak{B}\models\Sigma$.
\end{customthm}

\begin{customthm}{II.14.5}
  Let $\varphi(\vec{x})=\neg p_1(\vec{x})\vee\cdots\vee\neg p_{n-1}(\vec{x})\vee p_n(\vec{x})\in\Sigma$ be a Horn clause whose universal closure is $\psi\in\Sigma$. Assume that $\mathfrak{CT}(\mathcal{L},\Sigma)\not\models\psi$. Then by Lemma II.12.10(2), there exist a tuple $\vec{\tau}$ of closed terms such that $\mathfrak{CT}(\mathcal{L},\Sigma)\not\models\varphi(\vec{\tau})$. This is equivalent to having $\mathfrak{CT}(\mathcal{L},\Sigma)\models p_i(\vec{\tau})$ for $1\leq i\leq n-1$ and $\mathfrak{CT}(\mathcal{L},\Sigma)\not\models p_n(\vec{\tau})$, which is further equivalent to having $\Sigma\vdash p_i(\vec{\tau})$ for $1\leq i\leq n-1$ and $\Sigma\not\vdash p_n(\vec{\tau})$ by Lemma II.12.10(3). On the other hand, we have $\Sigma\vdash p_1(\vec{\tau})\rightarrow\cdots\rightarrow p_{n-1}(\vec{\tau})\rightarrow p_n(\vec{\tau})$ by $\psi\in\Sigma$, UI and a tautology. Here $\rightarrow$ is understood as right-associative. Then by applying Modus Ponens $n-1$ times, we get $\Sigma\vdash p_n(\vec{\tau})$, which is a contradiction. Hence $\mathfrak{CT}(\mathcal{L},\Sigma)\models\psi$ holds. In a similar way, one can also prove that $\mathfrak{CT}(\mathcal{L},\Sigma)\models\chi$ if $\chi\in\Sigma$ is a universal closure of a Horn clause without a positive literal. Thus $\mathfrak{CT}(\mathcal{L},\Sigma)\models\Sigma$ holds.
\end{customthm}

\begin{customthm}{II.15.3}
  Let $\Sigma$ be a set of sentences of $\mathcal{L}=\{\vee,\wedge\}$ where each sentence of it is a universal closure of one of the following 8 equations.
  \begin{align*}
    &x\vee(y\vee z)=(x\vee y)\vee z &&x\wedge(y\wedge z)=(x\wedge y)\wedge z \\
    &x\vee y=y\vee x &&x\wedge y=y\wedge x \\
    &x\vee x=x &&x\wedge x=x \\
    &x\wedge(x\vee y)=x &&x\vee(x\wedge y)=x
  \end{align*}
  Then $\Sigma$ is an equational theory. To prove that any lattice is a model of $\Sigma$, it suffices to show that the least upper bound operator $\vee$ and the greatest lower bound operator $\wedge$ satisfy the 8 equations above. One can easily prove it by using the properties that $x\vee y<z\leftrightarrow x<z$ \& $y<z$ and $z<x\wedge y\leftrightarrow z<x$ \& $z<y$. To prove that every model of $\Sigma$ is a lattice, we define the least upper bound operator by $\vee$ and the greatest lower bound operator by $\wedge$, and define the order $x<y$ by $y=x\vee y$ \& $x\neq y$. We have to confirm that $<$ is a partial order. The irreflexivity is obvious. To show the transitivity, let $x<y$ and $y<z$. That is, we have $y=x\vee y$, $x\neq y$, $z=y\vee z$ and $y\neq z$. Then it holds that $z=y\vee z=(x\vee y)\vee z=x\vee(y\vee z)=x\vee z$. If $x=z$ then $y=z\vee y=y\vee z=z$, which contradicts $y\neq z$. Hence we have $x\neq z$ and get $x<z$. Thus $<$ is a partial order and any model of $\Sigma$ is a lattice. Therefore being a lattice is equivalent to being a model of $\Sigma$ and we conclude that $\Sigma$ is an equational theory of lattices.
\end{customthm}

\begin{customthm}{II.15.4}
  Let $\mathcal{L}$ be a language, $\mathfrak{A}=(A,\mathcal{I})$, $\mathfrak{B}=(B,\mathcal{J})$ be structures for $\mathcal{L}$, $\varphi$ be a quantifier-free formula of $\mathcal{L}$ and $\sigma$ be an assignment for $\varphi$ in $A$. Assume that $\mathfrak{A}$ is a substructure of $\mathfrak{B}$ and that $\mathfrak{B}\models\varphi[\sigma]$. We prove $\mathfrak{A}\models\varphi[\sigma]$ by induction on $\varphi$. If $\varphi=p(\tau_1,\ldots,\tau_n)$, we have $(\val_{\mathfrak{B}}(\tau_1)[\sigma],\ldots,\val_{\mathfrak{B}}(\tau_n)[\sigma])\in p_\mathfrak{B}$ from $\mathfrak{B}\models\varphi[\sigma]$. Here we use the fact that $\val_{\mathfrak{A}}(\tau)[\sigma]=\val_{\mathfrak{B}}(\tau)[\sigma]$ holds for any term $\tau$ and any assignment $\sigma$ for $\tau$ in $A$, which can be easily proved by induction on $\tau$, to get $\val_{\mathfrak{A}}(\varphi)[\sigma]\in p_\mathfrak{B}\cap A^n$. Hence we get $\mathfrak{A}\models\varphi[\sigma]$. If $\varphi=\varphi_1\wedge\varphi_2$ and we have  $\mathfrak{A}\models\varphi_1[\sigma]$ \& $\mathfrak{A}\models\varphi_2[\sigma]$, then it is obvious that $\mathfrak{A}\models\varphi[\sigma]$ holds. The same goes for the other logical connectives. Thus we have $\mathfrak{B}\models\varphi[\sigma]\rightarrow\mathfrak{A}\models\varphi[\sigma]$ for any quantifier-free formula of $\mathcal{L}$ and any assignment $\sigma$ for $\varphi$ in $A$. Therefore for a set $\Sigma$ of sentences in $\mathcal{L}$ which is universal, we have $\mathfrak{B}\models\Sigma\rightarrow\mathfrak{A}\models\Sigma$.
\end{customthm}

\begin{customthm}{II.16.5}
  Let $n$ be the number of roots of $\tau$ in $B$. $n$ is finite since $n\leq\textrm{deg}(\tau)$ holds. Assuming we have already discovered $k(<n)$ different roots $b_1,\ldots,b_k$ of $\tau$ in $A$, we can apply II.16.4(2) with $\exists y(\tau(y)=0\wedge y\neq b_1\wedge\cdots\wedge y\neq b_k)$ to obtain another root $b_{k+1}$ in $A$. Repeating this until we get $b_n$, we will get $n$ roots in $A$.
\end{customthm}

\begin{customthm}{II.16.8}
  Let $\mathfrak{C}=(C;\triangleleft,T)$ and assume $\mathfrak{B}\preccurlyeq\mathfrak{C}$ throughout this exercise.
  \begin{enumerate}
    \item Let $P(n)$ say that $\forall x\in C[x\triangleleft n\rightarrow x\in\omega]$. We prove $\forall n\in\omega\,P(n)$ by induction. $P(0)$ holds because there is no $x\in C$ such that $x\triangleleft0$, which is clear from $\mathfrak{B}\models\forall x[x\not<0]$. Now we prove $P(n+1)$ from $P(n)$ and the fact $\mathfrak{B}\models\forall n\forall x[x<S(n)\leftrightarrow x<n\vee x=n]$. That is, for $x\in C$ such that $x\triangleleft T(n)$, it is either $x\triangleleft n$ or $x=n$, which anyway means $x\in\omega$ by inductive hypothesis or simply by $x=n\in\omega$. Hence $\forall n\in\omega\,P(n)$ holds, meaning that $\omega$ is an initial segment of $C$.
    \item We prove the following.
    \begin{enumerate}
      \item\underline{$\Phi$ is injective}. Let $x,y\in C$ and assume $\Phi(x)=\Phi(y)$. Note that it is either both $x$ and $y$ are in $\omega$ or both are in $C\setminus\omega$ because $\ran(\Phi\upharpoonright\omega)\subseteq\omega$ and $\ran(\Phi\upharpoonright C\setminus\omega)\subseteq C\setminus\omega$ hold. If $x$ (and $y$) is in $\omega$, then $x=\Phi(x)=\Phi(y)=y$ holds. If not (i.e. $x,y\in C\setminus\omega$), we have $\Phi(x)=T(x)$ and $\Phi(y)=T(y)$, from which we derive $x=y$ since it holds that $\mathfrak{B}\models\forall x,y[S(x)=S(y)\rightarrow x=y]$.
      \item\underline{$\Phi$ is surjective}. Let $y\in C$. If $y\in\omega$, then $y\in\ran(\Phi)$ since $y=\Phi(y)$. If not, considering $\mathfrak{B}\models\forall y[y\neq0\rightarrow\exists x[y=S(x)]]$, we get an $x\in C$ such that $y=T(x)=\Phi(x)$, which says $y\in\ran(\Phi)$.
      \item\underline{$\forall x,y\in C[x<y\leftrightarrow\Phi(x)\triangleleft\Phi(y)]$}. The full proof is omitted as it would be too long, but can be easily done by arguing by cases (whether $x$ or $y$ belong to $\omega$ or not), using $\mathfrak{B}\models\forall x,y[x<y\leftrightarrow S(x)<S(y)]$.
      \item\underline{$T\circ\Phi=\Phi\circ T$}. Let $x\in C$. If $x\in\omega$, then $T(\Phi(x))=T(x)=\Phi(T(x))$ holds. If not, we have $T(\Phi(x))=T(T(x))=\Phi(T(x))$.
    \end{enumerate}
    \item Assume $\mathfrak{B}\neq\mathfrak{C}$ and that there is a formula $\varphi(x)$ of $\mathcal{L}$ such that $\mathfrak{B}\models\varphi[a]$ iff $a$ is even. Then  $\psi=\forall x[\varphi(x)\leftrightarrow\neg\varphi(S(x))]$ is valid under $\mathfrak{B}$ so it is valid in $\mathfrak{C}$ as well. Now let $c\in C\setminus\omega$ and assume $\mathfrak{C}\models\varphi[c]$. Then $\mathfrak{C}\models\varphi[\Phi(c)]$ holds since $\Phi$ is an automorphism of $\mathfrak{C}$. Therefore we have $\mathfrak{C}\models\varphi[c]\wedge\varphi[S(c)]$, which contradicts $\psi$. If $\mathfrak{C}\not\models\varphi[c]$, then we will have $\mathfrak{C}\models\neg\varphi[c]\wedge\neg\varphi[S(c)]$ in the same way, which anyway leads to a contradiction. Thus such a $\varphi$ does not exist.
  \end{enumerate}
\end{customthm}

\begin{customthm}{II.16.9}
  Let $\varphi(\vec{x})=\exists y\psi(\vec{x},y)$ be any existential formula and let $\vec{a}\in A$ be one such that $\mathfrak{B}\models\varphi[\vec{a}]$. Then there exists $b\in B$ such that $\mathfrak{B}\models\psi[\vec{a},b]$. By using an automorphism of $\mathfrak{B}$ such that $\Phi(\vec{a})=\vec{a}$ and $\Phi(b)\in A$, we can get $b'=\Phi(b)\in A$ that satisfies $\mathfrak{B}\models\psi[\vec{a},b']$ since $\mathfrak{B}\models\psi[\Phi(\vec{a}),\Phi(b)]$ holds. Hence we get $\mathfrak{A}\preccurlyeq\mathfrak{B}$ by Lemma II.16.4.
\end{customthm}

\begin{customthm}{II.16.10}
  Let $\mathfrak{A},\mathfrak{B}$ be infinite abelian group of exponent $p$ such that $\mathfrak{A}\subseteq\mathfrak{B}$. $B$ can be viewed as an infinite dimensional vector space over $\Sigma_p$ and $A$ can be viewed as an infinite dimensional subspace of $B$. Let $a_1,\ldots,a_n\in A$ and $b\in B$. If $b\in A$, the identity map satisfies the condition of Exercise II.16.9. If $b\not\in A$, then let $c\in A$ be a vector such that $a_1,\ldots,a_n,b,c$ are linearly independent. Such $c$ exists because $A$ is infinite dimensional. Let $\mathcal{B}_A$ be a basis of the subspace spanned by $a_1,\ldots,a_n$ and let $\mathcal{B}$ be a basis of $B$ that extends $\mathcal{B}_A\cup\{b,c\}$. Now we define a linear transformation $\Phi:B\rightarrow B$ by $\Phi(b)=c$, $\Phi(c)=b$ and $\Phi(v)=v$ for all other $v\in\mathcal{B}$. Then this $\Phi$ is bijective and hence is an automorphism, which satisfies the condition of Exercise II.16.9. Thus $\mathfrak{A}\preccurlyeq\mathfrak{B}$ holds.
\end{customthm}

\begin{customthm}{II.16.11}
  Let $0<a_1<\ldots<a_n<1\leq b<2$. Then the following function $f:(0,2)\rightarrow(0,2)$ defined by
  \[
    f(x)=
    \begin{cases}
      x & \text{if } 0<x\leq a_n \\
      \frac{1-a_n}{2(b-a_n)}(x-a_n)+a_n & \text{if } a_n<x<b \\
      2-\frac{3-a_n}{2(2-b)}(2-x) & \text{if } b\leq x<2
    \end{cases}
  \]
  satisfies the condition of Exercise II.16.9. Hence we get $(0,1)\preccurlyeq(0,2)$.

  To prove the rest, let $A,B$ be countable dense totally ordered sets such that $A\subseteq B$. Let $a_1,\ldots,a_n\in A$, $a_1<\ldots<a_n$ and $b\in B\setminus A$. We fix an element of $c\in A$ which belongs to the same interval with $b$ partitioned by $a_1,\ldots,a_n$. Then we can construct an automorphism $\Phi$ of $B$, starting from $\Phi(a_1)=a_1,\ldots,\Phi(a_n)=a_n$ and $\Phi(b)=c$ as we saw in Exercise II.13.10. Thus the condition of Exercise II.16.9 is satisfied and it turns out that $\mathfrak{A}\preccurlyeq\mathfrak{B}$ holds whenever $\mathfrak{A}\subseteq\mathfrak{B}$ and $\mathfrak{A},\mathfrak{B}$ are countable models of $\Delta^{(~)}$.
\end{customthm}

\begin{customthm}{II.16.12}
  Assume that $\Sigma$ is not model-complete and let $\mathfrak{A},\mathfrak{B}$ be $\mathcal{L}$-structures such that $\mathfrak{A},\mathfrak{B}\models\Sigma$, $\mathfrak{A}\subseteq\mathfrak{B}$ and $\mathfrak{A}\not\preccurlyeq\mathfrak{B}$.

  Let $p$ be a new unary predicate symbol and let $\mathcal{L}'=\mathcal{L}\cup\{p\}$. We expand $\mathfrak{B}$ to an $\mathcal{L}'$-structure $\mathfrak{B}'$ by letting $p_{\mathfrak{B}'}=A$. Using the Löwenheim-Skolem theorem, we get an $\mathcal{L}'$-structure $\mathfrak{B}_1$ such that $\mathfrak{B}_1\equiv\mathfrak{B}'$ and $|\mathfrak{B}_1|=\aleph_0$. Now let $A_1=p_{\mathfrak{B}'}$. For any $n$-ary function $g\in\mathcal{L}$, it holds that $\ran(g\upharpoonright {A_1}^n)\subseteq A_1$ because $\mathfrak{B}'\models\forall\vec{x}[p(\vec{x})\rightarrow p(g(\vec{x}))]$ holds from $\mathfrak{A}\subseteq\mathfrak{B}$ (here, $p(\vec{x})$ abbreviates $p(x_1)\wedge\cdots\wedge p(x_n)$ for $\vec{x}=(x_1,\ldots,x_n)$). Therefore we can define a substructure $\mathfrak{A}_1\subseteq\mathfrak{B}_1$ whose carrier set is $A_1$. For this substructure $\mathfrak{A}_1$, we also have $\mathfrak{A}_1\models\Sigma$. To prove that, for any formula $\varphi$ of $\mathcal{L}$, consider a corresponding formula $\varphi_p$ of $\mathcal{L}'$ that is relativized to the set defined by $p$. Specifically, $\varphi_p$ is recursively defined by replacing every $\forall x\chi$ in $\varphi$ with $\forall x[p(x)\rightarrow\chi]$ and every $\exists x\chi$ in $\varphi$ with $\exists x[p(x)\wedge\chi]$. Then for any sentence $\varphi$ of $\mathcal{L}$, it holds that $\mathfrak{A}_1\models\varphi$ iff $\mathfrak{B}_1\models\varphi_p$ iff $\mathfrak{B}'\models\varphi_p$ iff $\mathfrak{A}\models\varphi$. The second 'iff' comes from $\mathfrak{B}_1\equiv\mathfrak{B}'$ and the rest are easily proved by induction on the number of quantifiers in $\varphi$, using Lemma II.16.2 and our definition of $\varphi_p$. Finally, we have $\mathfrak{A}_1\not\preccurlyeq\mathfrak{B}_1$. To prove this, we use $\mathfrak{A}\not\preccurlyeq\mathfrak{B}$ and Lemma II.16.4 to get a formula $\psi(\vec{x},y)$ of $\mathcal{L}$ such that there exist $\vec{a}\in A$ and $b\in B$ such that $\mathfrak{B}\models\psi[\vec{a},b]$ but no $a'\in A$ satisfies $\mathfrak{B}\models\psi[\vec{a},a']$. Then here is an $\mathcal{L}'$-formula $\exists\vec{x}[p(\vec{x})\wedge\exists y\psi(\vec{x},y)\wedge\neg\exists y[p(y)\wedge\psi(\vec{x},y)]]$ that is true in in $\mathfrak{B}'$ and hence in $\mathfrak{B}_1$, which means $\mathfrak{A}_1\not\preccurlyeq\mathfrak{B}_1$.

  Now let $f$ be a new unary function symbol and let $\mathcal{L}''=\mathcal{L}'\cup\{f\}$. We expand $\mathfrak{B}_1$ to an $\mathcal{L}''$-structure $\mathfrak{B}_1''$ by letting $p_{\mathfrak{B}_1''}=A_1$ and defining $f_{\mathfrak{B}_1''}$ as an injection to $A_1$. Note that this is possible since $|A_1|=\aleph_0$ holds by the assumption that $\Sigma$ has no finite models. Using the Löwenheim-Skolem theorem, we get an $\mathcal{L}''$-structure $\mathfrak{B}_2$ such that $\mathfrak{B}_2\equiv\mathfrak{B}_1''$ and $|\mathfrak{B}_2|=\kappa$. We have a substructure $\mathfrak{A}_2\subseteq\mathfrak{B}_2$ such that $\mathfrak{A}_2\models\Sigma$ and $\mathfrak{A}_2\not\preccurlyeq\mathfrak{B}_2$ by the same reasoning as we got $\mathfrak{A}_1$ above. This $\mathfrak{A}_2$ has cardinality $\kappa$ because the sentence $\forall x,y[f(x)=f(y)\rightarrow x=y]$ is true in $\mathfrak{B}_1''$ and hence in $\mathfrak{B}_2$, which means $|\mathfrak{A}_2|=\kappa$. Now this pair of models $\mathfrak{A}_2$ and $\mathfrak{B}_2$ contradicts the given assumption and we conclude that $\Sigma$ is model-complete.
\end{customthm}

\begin{customthm}{II.16.13}
  By Exercise II.16.11, we know that $\Delta^{(~)}$ satisfies the condition of Exercise II.16.12. Hence it is model-complete.
\end{customthm}

\begin{customthm}{II.16.14}
  As we saw in Exercise II.13.7, any infinite torsion-free divisible abelian group of cardinality $\aleph_1$ can be viewed as a vector space over $\mathbb{Q}$ which has cardinality $\aleph_1$. Therefore we can prove $\mathfrak{A}\preccurlyeq\mathfrak{B}$ whenever $\mathfrak{A},\mathfrak{B}$ are infinite torsion-free divisible abelian groups such that $\mathfrak{A}\subseteq\mathfrak{B}$ and $|\mathfrak{A}|=|\mathfrak{B}|=\aleph_1$ just as we did in Exercise II.16.10. Thus by Exercise II.16.12, the theory turns out to be model-complete.
\end{customthm}

\begin{customthm}{II.16.15}
  We first show that any boolean function of positive arity can be written in conjunctive normal form. Let $0<n\in\omega$ and $\varphi:2^n\rightarrow 2$ be a boolean function. Then the following formula $\psi:2^n\rightarrow 2$ defined as
  \[
    \psi(p_1,\ldots,p_n)=\bigwedge_{\varphi(b_1,\ldots,b_n)=F}(p_1=\overline{b_1}\vee\cdots\vee p_n=\overline{b_n})
  \]
  is equal to $\varphi$ and is in conjunctive normal form if there are any boolean values $b_1,\ldots,b_n$ such that $\varphi(b_1,\ldots,b_n)=F$. If not, then $\varphi$ is always true and we can simply make $\psi=p_1\vee\neg p_1$.

  Given a quantifier-free formula $\varphi$, we can get a boolean function $\varphi'$ by replacing each atomic formula in it with different boolean variables. We have a boolean function $\psi'$ in conjunctive normal form which is equal to $\varphi'$ from the above. By replacing the boolean variables in $\psi'$ with the corresponding atomic formulas, we can get a formula $\psi$ in conjunctive normal form which is equivalent to $\varphi$.
\end{customthm}

\begin{customthm}{II.16.16}
  We prove it by induction. For basis, note that any atomic formula is quantifier-free and hence is in prenex normal form. Now let $\varphi=Q_1x_1Q_2x_2\cdots Q_nx_n\psi$ where $\psi$ is a quantifier-free formula. $\neg\varphi$ is equivalent to $\overline{Q_1}x_1\overline{Q_2}x_2\cdots\overline{Q_n}x_n\neg\psi$, where each $Q_i$ stands for the opposite quantifier. Let $\psi'$ be a quantifier-free formula, $\varphi'=Q_1'y_1Q_2'y_2\cdots Q_m'y_m\psi'$ be another formula in prenex normal form and consider $\varphi\wedge\varphi'$. We can make all the variables $x_1,\ldots,x_n,y_1,\ldots,y_m$ that occur in the quantifiers in $\varphi\wedge\varphi'$ different from each other and from any of its free variables, keeping the logical equivalence. By assuming that, we have a formula $Q_1x_1Q_2x_2\cdots Q_nx_nQ_1'y_1Q_2'y_2\cdots Q_m'y_m(\psi\wedge\psi')$ in a prenex normal form, which is equivalent to $\varphi\wedge\varphi'$. Now we proved the negation case and the conjunction case. $\forall x\varphi$ and $\exists x\varphi$ are obviously in prenex normal form. For the other types of formulas, one can use the fact that $\varphi_1\vee\varphi_2,\varphi_1\rightarrow\varphi_2,\varphi_1\leftrightarrow\varphi_2$ are equivalent to $\neg(\neg\varphi_1\wedge\neg\varphi_2), \neg\varphi_1\vee\varphi_2,(\varphi_1\rightarrow\varphi_2)\wedge(\varphi_2\rightarrow\varphi_1)$, respectively to reduce them to the cases of negation and conjunction. Thus every formula is logically equivalent to one in prenex normal form.
\end{customthm}

\begin{customthm}{II.16.17}
  By Exercise II.16.16, we may assume that every sentence $\varphi$ in $\Sigma$ is in prenex normal form $\varphi=\forall\vec{x_1}\exists y_1\forall\vec{x_2}\exists y_2\cdots\forall\vec{x_n}\exists y_n\psi(\vec{x_1},y_1,\vec{x_2},y_2,\ldots,\vec{x_n},y_n)$ where $\psi$ is $\forall\vec{x}\chi$ for some quantifier-free formula $\chi$. For such a sentence $\varphi$, we consider a set of new function symbols $\mathcal{C}_\varphi=\{f_1,f_2,\ldots,f_n\}$ where each $f_k$ has arity $\mathrm{dim}(\vec{x_1})+\cdots+\mathrm{dim}(\vec{x_k})$. Using these function symbols, we consider a corresponding universal sentence $\varphi'$ in $\mathcal{L}\cup\mathcal{C}_\varphi$ defined by $\varphi'=\forall\vec{x_1}\forall\vec{x_2}\cdots\forall\vec{x_n}\psi(\vec{x_1},f_1(\vec{x_1}),\vec{x_2},f_2(\vec{x_1},\vec{x_2}),\ldots,\vec{x_n},f_n(\vec{x_1},\vec{x_2},\ldots,\vec{x_n}))$. Now we let $\mathcal{C}=\bigcup_{\varphi\in\Sigma}\mathcal{C}_\varphi$, $\mathcal{L}'=\mathcal{L}\cup\mathcal{C}$ and $\Sigma'=\{\varphi':\varphi\in\Sigma\}$ to prove each item.
  \begin{enumerate}
    \item Obvious.
    \item $\Sigma'$ is clearly a set of $\mathcal{L}'$ sentences. To see that every sentence of $\Sigma$ is provable from $\Sigma'$, one can show that every $\varphi=\forall\vec{x_1}\exists y_1\forall\vec{x_2}\exists y_2\cdots\forall\vec{x_n}\exists y_n\psi(\vec{x_1},y_1,\vec{x_2},y_2,\ldots,\vec{x_n},y_n)\in\Sigma$ is provable from its corresponding $\varphi'\in\Sigma'$ by repeating a sequence of EG, UI and UG rules $n$ times as below.
    \begin{center}
      \begin{tabular}{ll}
        $\psi(\vec{c_1},f_1(\vec{c_1}),\ldots,\vec{c_n},f_n(\vec{c_1},\ldots,\vec{c_n}))\vdash\psi(\vec{c_1},f_1(\vec{c_1}),\ldots,\vec{c_n},f_n(\vec{c_1},\ldots,\vec{c_n}))$ & tautology \\
        $\psi(\vec{c_1},f_1(\vec{c_1}),\ldots,\vec{c_n},f_n(\vec{c_1},\ldots,\vec{c_n}))\vdash\exists y_n\psi(\vec{c_1},f_1(\vec{c_1}),\ldots,\vec{c_n},y_n)$ & EG \\
        $\forall\vec{x_n}\psi(\vec{c_1},f_1(\vec{c_1}),\ldots,\vec{x_n},f_n(\vec{c_1},\ldots,\vec{x_n}))\vdash\exists y_n\psi(\vec{c_1},f_1(\vec{c_1}),\ldots,\vec{c_n},y_n)$ & UI \\
        $\forall\vec{x_n}\psi(\vec{c_1},f_1(\vec{c_1}),\ldots,\vec{x_n},f_n(\vec{c_1},\ldots,\vec{x_n}))\vdash\forall\vec{x_n}\exists y_n\psi(\vec{c_1},f_1(\vec{c_1}),\ldots,\vec{x_n},y_n)$ & UG \\
        $\vdots$ & $\vdots$ \\
        $\varphi'\vdash\varphi$ & UG \\
      \end{tabular}
    \end{center}
    \item Obvious.
    \item We first show a way to expand any model of $\Sigma$ to a model of $\Sigma'$. Let $\mathfrak{A}$ be a model of $\Sigma$, let $\varphi=\forall\vec{x_1}\exists y_1\forall\vec{x_2}\exists y_2\cdots\forall\vec{x_n}\exists y_n\psi(\vec{x_1},y_1,\vec{x_2},y_2,\ldots,\vec{x_n},y_n)\in\Sigma$, and let $\varphi'\in\Sigma'$ be the corresponding sentence. We are going to expand $\mathfrak{A}$ to $\mathfrak{A}_\varphi$ that satisfies $\varphi'$. Since $\mathfrak{A}\models\varphi$ holds, for each $\vec{a}\in A$, we have a non-empty subset $S_{\vec{a}}=\{b\in A:\mathfrak{A}\models\forall\vec{x_2}\exists y_2\cdots\forall\vec{x_n}\exists y_n\psi(\vec{x_1},y_1,\vec{x_2},y_2,\ldots,\vec{x_n},y_n)[\vec{x_1}/\vec{a},y_1/b]\}$, from which we can define ${f_1}_{\mathfrak{A}_\varphi}$ so that $\forall\vec{x_2}\exists y_2\cdots\forall\vec{x_n}\exists y_n\psi(\vec{x_1},f(\vec{x_1}),\vec{x_2},y_2,\ldots,\vec{x_n},y_n)[\vec{x_1}/\vec{a}]$ is true in $\mathfrak{A}_\varphi$, using Axiom of Choice. Likewise, we can define the rest of functions in $\varphi'$ so that $\varphi'$ is finally true in $\mathfrak{A}_\varphi$. Repeating this process for all the formulas in $\Sigma$, we get an expansion of $\mathfrak{A}$ that models the whole $\Sigma'$. Now, let $\varphi$ be any sentence in $\mathcal{L}$ and assume $\Sigma'\vdash\varphi$. Let $\mathfrak{A}$ be a model of $\Sigma$ and let $\mathfrak{A}'$ be an expansion of it that models $\Sigma'$. Then we have $\mathfrak{A}'\models\varphi$ since $\Sigma'\models\varphi$ holds from $\Sigma'\vdash\varphi$ and Soundness. Therefore $\mathfrak{A}\models\varphi$ holds as well because we have $\val_\mathfrak{A}(\varphi)=\val_{\mathfrak{A}'}(\varphi)$. Thus we get $\Sigma\models\varphi$, which means $\Sigma\vdash\varphi$ from Completeness.
  \end{enumerate}
\end{customthm}

\begin{customthm}{II.16.18}
  Let $\mathrm{eDiag}(\mathfrak{A})$ be the elementary diagram of $\mathfrak{A}$, written in $\mathcal{L}_A=\mathcal{L}\cup\{c_a:a\in A\}$. Let $c$ be another new constant symbol and let $\mathcal{L}'=\mathcal{L}\cup\{c\}$. Now consider a set of $\mathcal{L}'$-formulas $\Sigma=\mathrm{eDiag}(\mathfrak{A})\cup{\varphi(c)}\cup\{c\neq c_a:a\in A\}$. Note that $\Sigma$ is consistent by the assumption that $\varphi^\mathfrak{A}$ is infinite. Indeed, $\mathfrak{A}$ is always a model for any finite subset $\Delta\subseteq\Sigma$ when it interprets $c$ as something in $\varphi^\mathfrak{A}\setminus\{a\in A: c_a\in\Delta\}$. Then $\Sigma$ has a model $\mathfrak{B}$ where we may assume that $\mathfrak{B}$ interprets each constant $c_a$ as $a\in A$. Thus we have $\mathfrak{A}\preccurlyeq\mathfrak{B}$ from $\mathfrak{B}\models\mathrm{eDiag}(\mathfrak{A})$ and also have $\varphi^\mathfrak{B}\supseteq\varphi^\mathfrak{A}\cup\{c_\mathfrak{B}\}\supsetneq\varphi^\mathfrak{A}$ from the rest of $\Sigma$.
\end{customthm}

\begin{customthm}{II.16.19}
  Below is the proof for each item.
  \begin{enumerate}
    \item Let $b\in B\setminus\{0\}$. If $b$ is infinitely large, then $1/b$ is infinitesimal because $|b|>1/a$ holds for all positive real numbers $a$, which is equivalent to $|1/b|<a$. If $1/b$ is infinitesimal, then $b$ is infinitely large because $|b|>0$ holds and $|1/b|<|1/a|$ holds for all non-zero real numbers $a$, which is equivalent to having $|b|>|a|$ for all real numbers $a$.
    \item Let $b\in B\setminus\mathbb{R}$. If $b$ is infinitely large, then we can get an infinitesimal $1/b$. If not, $\varepsilon:=b-\sup\{a\in\mathbb{R}:a<b\}$ where sup is calculated in $\mathbb{R}$ is an infinitesimal and we can get an infinitely large number $1/\varepsilon$.
    \item Let $\eta\in\mathbb{R},\eta>0$ and let $\xi$ be a positive real number such that $\forall h(|h|<\xi\rightarrow|(f(r+h)-f(r))/h-g(r)|<\eta)$ holds. Then by interpreting this formula in $\mathfrak{B}$ and instantiating the $h$ by $\varepsilon$, we have $|(f_\mathfrak{B}(r+\varepsilon)-f_\mathfrak{B}(r))/\varepsilon-g_\mathfrak{B}(r)|<\eta$. Since $\eta$ is an arbitrary positive real number, we can conclude that $(f_\mathfrak{B}(r+\varepsilon)-f_\mathfrak{B}(r))/\varepsilon-g_\mathfrak{B}(r)$ is either infinitesimal or $0$.
    \item Let $\psi(x)=x\neq0\wedge x\neq\pm 1\wedge\neg\exists y,z[x=yz\wedge Z(x)\wedge Z(y)\wedge1<|y|<|x|]$. Then $x\in\mathbb{R}$ is a prime number iff $\varphi(x)$ holds. Since there are arbitrarily large prime numbers, $\forall x\exists y(x<y\wedge\psi(y))$ holds. By interpreting it in $\mathfrak{B}$ with an infinitely large positive $x$, we get an infinitely large prime $y$. Note that if $x<y$ holds for an infinitely large number $x$, then $y$ is infinitely large as well.
    \item A maximal element in $P$ would contradict $\forall x\exists y(x<y\wedge\psi(y))$. To prove that $P$ does not have a minimal element, we consider the sentence $\forall x [2<x\wedge Z(x)\rightarrow\exists y[\psi(y)\wedge\floor*{\sqrt[\ceil*{\log_2x}]{x}}\leq y<x]$, which one can easily prove in $\mathbb{R}$ by thinking of the factorization of $x$. A minimal element $x\in P$ would contradict it.
    \item If there are infinitely many twin primes, then $\forall x\exists y[x<y\wedge P(y)\wedge P(y+2)]$ is true. Interpreting this sentence in $\mathfrak{B}$ with an infinitely large number $x$, we get a pair of infinitely large primes $y$ and $y+2$. If there are only finitely many twin primes, let $x$ be the larger one of the largest twin prime pair and consider $\neg\exists y[x<y\wedge P(y)\wedge P(y+2)]$. The interpretation of it in $\mathfrak{B}$ will tell us that there is no infinitely large twin prime.
  \end{enumerate}
\end{customthm}

\begin{customthm}{II.17.2}
  The function $f:R(\gamma)\rightarrow R(\gamma)$ recursively defined by $f(0)=7$ and $f(x)=\{f(y):y\in x\}$ if $0<\rank(x)<\gamma$, is an isomorphic embedding since it obviously satisfies $x\in y\leftrightarrow f(x)\in f(y)$.
\end{customthm}

\begin{customthm}{II.17.14}
  Considering the Axiom of Infinity, we have $\gamma>\omega$. Replacement forces $\gamma$ to be a regular cardinal. Our arguments so far about cardinalities prove $2^\lambda<\gamma$ whenever $\lambda<\gamma$. Hence $\gamma$ is strongly inaccessible. By Exercise I.13.17, we have $\gamma=\beth_\gamma$ and $\type(\{\alpha<\gamma:\alpha=\beth_\alpha\})=\gamma$.
\end{customthm}

\begin{customthm}{II.17.16}
  By Lemma II.17.9 (10), (19), (17) and (18), the following statement $\varphi=\forall x\exists n\exists f[[n\textrm{ is a natural number}]\wedge[f\textrm{ is a bijection from }\dom(f)\textrm{ onto }\ran(f)]\wedge[\dom(f)=x]\wedge[\ran(f)=n]]$ can be viewed as a $\Delta_0$ sentence of $\mathcal{L}=\{\in\}$. Since $HF$ is transitive, the interpretation of $\varphi$ in $HF$ is that every set in it has a bijection onto some natural number, just as we interpret $\varphi$ in $V$. Therefore $HF\models\varphi$ is obvious because every set in $HF$ is finite.
\end{customthm}

\begin{customthm}{II.17.23}
  Let $\theta(x,e)$ say that $x$ is a natural number, $e\subseteq x$, and $e$ contains exactly the even numbers less than $x$. Note that by Lemma II.17.9 and Example II.17.5, the first 2 statements can be expressed as $\Delta_0$ formulas and so can the last one because it is equivalent to $0\in e\wedge\forall y\in e[S(S(y))\in x\rightarrow S(S(y))\in e]\wedge\forall y\in e[y=0\vee\exists z\in e[y=S(S(z))]]$. Therefore we may assume that $\theta(x,e)$ is a $\Delta_0$ formula. Now, for $x\in\omega$, $x$ is odd iff $\exists e[\theta(x,e)\wedge\exists u\in e[x=S(u)]]$ and even iff $x=0\vee\exists e[\theta(x,e)\wedge u\in e[x=S(S(u))]]$. The negation of the first formula gives us a $\Pi_0$ formula and the latter formula gives us a $\Sigma_0$ formula, both defining $E=\{2n:n\in\omega\}$. Thus $E$ is a $\Delta_1$ subset of $HF$.
\end{customthm}

\begin{customthm}{II.17.30}
  Let $A\preccurlyeq H(\aleph_1)$. We prove that $A$ is transitive. First of all, we make sure that $\emptyset\in A$ by considering $\mathrm{emp}(x)=\forall z(z\not\in x)$ both on $A$ and on $H(\aleph_1)$, where $H(\aleph_1)\models\exists!x\,\mathrm{emp}(x)$ and hence $A\models\exists!x\,\mathrm{emp}(x)$. By $A\preccurlyeq H(\aleph_1)$, we find that $\mathrm{emp}^A(a)$ iff $\mathrm{emp}^{H(\aleph_1)}(a)$ for $a\in A$, and that proves $\emptyset\in A$. Likewise, we can show that $\omega\in A$ by considering a formula that uniquely defines $\omega$, which is possible because $H(\aleph_1)$ satisfies the Axiom of Infinity and Comprehension. Now consider the sentence $\varphi=\forall x\neq\emptyset\,\exists f:\omega\xrightarrow{\textrm{onto}}x\,[\forall n\in\omega\,\exists!y\in x[f(n)=y]]$. $H(\aleph_1)\models\varphi$ holds because every set is countable and $\trcl(f)$ is also countable. Considering $A\models\varphi$, we conclude that every element $y$ of a non-empty set $x\in A$ also lies in $A$, which means that $A$ is transitive.

  Now, since $H(\aleph_1)$ is uncountable, there is a $b\in H(\aleph_1)$ such that $\{b\}$ is not $\Delta_0$ on $H(\aleph_1)$. By the Downward Löwenheim-Skolem-Tarski theorem, we have a countable $A\preccurlyeq H(\aleph_1)$ such that $b\in A$. Then $\{b\}$ is not $\Delta_0$ on $A$ either. To see this, let $\varphi(x)$ be a formula that defines $\{b\}$ on $A$. We have $A\models\exists!x\varphi(x)$, and hence $H(\aleph_1)\models\exists!x\varphi(x)$. We also have $\varphi^A(b)$ and hence $\varphi^{H(\aleph_1)}(b)$. Therefore $\varphi$ defines $\{b\}$ on $H(\aleph_1)$ as well, which cannot be $\Delta_0$. Thus we have a countable transitive $A$ with some $b\in A$ such that $\{b\}$ is not $\Delta_0$ on $A$.
\end{customthm}

\begin{customthm}{II.18.9}
  We show that $PAS$ proves Replacement and Choice. The Power Set axiom has already been done in Lemma II.18.7 and the rest are obvious since they are part of $CST$.

  For Replacement, let $\varphi$ be a formula, let $A$ be a set and assume that $\forall x\in A\,\exists!y\,\varphi(x,y)$. We show by induction on $n=|A|$ that there exists a set $B$ such that $\forall x\in A\,\exists y\in B\,\varphi(x,y)$. Note that we assume $A$ is finite since we are working in $PAS$ now. If $n=0$, $B=\emptyset$ will do. Now let $A'=A\cup\{p\}$, $p\not\in A$, $|A|=n$ and assume we have a corresponding $B$ from the inductive hypothesis. Then, using the $y$ such that $\varphi(p,y)$, we define $B'=B\cup\{y\}$, which exists by Lemma II.18.7 and satisfies $\forall x\in A'\,\exists y\in B'\,\varphi(x,y)$.

  For Choice, we induct on the size of the given family of sets. If the family is of size $0$ (i.e. the empty set), $\emptyset$ is a choice set. For the inductive step, consider a family $F'=F\cup\{P\}$ with $P\neq\emptyset$, $P\not\in F$. A choice set for $F'$ can be $C\cup\{p\}$ where $C$ is a choice set for $F$ (inductive hypothesis) and $p$ is some element in $P$. Note that $C\cup\{p\}$ exists by Lemma II.18.7.
\end{customthm}

\begin{customthm}{II.18.10}
  Let $f:\omega\rightarrow N$ be the function defined by (a) and (b). That is, define $f$ by $f(0)=c$ and $f(S(n))=\sigma(f(n))$ for $n\in\omega$. $f$ is surjective because $\ran(f)=N$ follows from (9) and $c\in\ran(f)\wedge\forall x\in\ran(f)[\sigma(x)\in\ran(f)]$. To show the injectivity of $f$, assume $f$ is not injective and let $n\in\omega$ be the least one such that $\exists m<n[f(m)=f(n)]$. Let $l$ be the predecessor of $n$ (i.e. $n=S(l)$). $m$ cannot be $0$ because we have $f(n)=\sigma(f(l))$ and $f(0)=c\not\in\ran(\sigma)$ from (8). Now let $k$ be the predecessor of $m$. Then we get $\sigma(f(k))=\sigma(f(l))$ from $f(m)=f(n)$. Since $\sigma$ is injective from (7), we have $f(k)=f(l)$, which contradicts the minimality of $n$. Thus $f$ is injective and we conclude that $f$ is a bijection from $\omega$ to $N$.
\end{customthm}

\begin{customthm}{II.18.12}
  Below is the proof for each item. Note that we can assume by Lemma II.18.7 that any function between finite sets exists.
  \begin{enumerate}[label=\alph*.]
    \item\underline{$S(x)=S(y)\rightarrow x=y$}. Assume $S(x)=S(y)$ and $x\neq y$. Then $x\in y\in x$ holds, which contradicts Foundation. Hence we have $x=y$ whenever $S(x)=S(y)$.
    \item\underline{$S(x)\neq 0$}. Since $x\in S(x)=x\cup\{x\}$, $S(x)$ is non-empty.
    \item\underline{$x+0=x$}. $x+0=|\{0\}\times x\cup\{1\}\times0|=|\{0\}\times x|=|x|=x$.
    \item\underline{$x+S(y)=S(x+y)$}. It holds that $x+S(y)=|\{0\}\times x\cup\{1\}\times y\cup\{\langle1,y\rangle\}|$ and $S(x+y)=|\{0\}\times x\cup\{1\}\times y|\cup\{x+y\}$. We can extend a bijection $f:\{0\}\times x\cup\{1\}\times y\rightarrow x+y$ to define a bijection $g:\{0\}\times x\cup\{1\}\times y\cup\{\langle1,y\rangle\}\rightarrow S(x+y)$ by $g=f\cup\{\langle\langle1,y\rangle,x+y\rangle\}$. Thus $x+S(y)=S(x+y)$ holds.
    \item\underline{$x\cdot 0=0$}. $x\cdot 0=|x\times 0|=|\emptyset|=0$.
    \item\underline{$x\cdot S(y)=x\cdot y+x$}. It holds that $x\cdot S(y)=|x\times y\cup x\times\{y\}|$ and $x\cdot y+x=|\{0\}\times|x\times y|\cup\{1\}\times x|$. We can use a bijection $f:x\times y\rightarrow|x\times y|$ to define a bijection $g:x\times y\cup x\times\{y\}\rightarrow\{0\}\times|x\times y|\cup\{1\}\times x$ by $g(i,j)=\langle0,f(i,j)\rangle$ and $g(i,y)=\langle1,i\rangle$ where $i<x$ and $j<y$. Thus $x\cdot S(y)=x\cdot y+x$ holds.
    \item\underline{$\varphi(0)\wedge\forall x[\varphi(x)\rightarrow\varphi(S(x))]\rightarrow\forall x\varphi(x)$}. The relativized formula is $\varphi(0)\wedge\forall x[\mathrm{nat}(x)\wedge\varphi(x)\rightarrow\varphi(S(x))]\rightarrow\forall x[\mathrm{nat}(x)\rightarrow\varphi(x)]$. It is a variant of the ordinary induction in $CST$ assuming the first universal quantification is restricted to natural numbers, which can be proved in $CST$ just like the ordinary induction.
  \end{enumerate}
\end{customthm}

\begin{customthm}{II.18.14}
  Skipped for now.
\end{customthm}

\begin{customthm}{II.18.15}
  Skipped for now.
\end{customthm}

\begin{customthm}{II.18.16} (half-done)
  Below is the proof for each item.
  \begin{enumerate}
    \item We can prove that $\omega$ is a proper initial segment of $\mathrm{nat}^\mathfrak{A}$ just like we have done in Exercise II.16.8.1. $\mathrm{nat}^\mathfrak{A}\setminus\omega$ has no least element because $\forall x[\mathrm{nat}(x)\wedge x\neq0\rightarrow\exists w[\mathrm{nat}(w)\wedge w<x]]$ holds.
    \item Skipped for now.
    \item Skipped for now.
  \end{enumerate}
\end{customthm}

\begin{customthm}{II.18.17}
  Skipped for now.
\end{customthm}

\begin{customthm}{IV.3.4}
  Skipped for now.
\end{customthm}

\begin{customthm}{IV.3.19}
  We define $C:HF\rightarrow HF$ by $C(x)=\min_\Delta(x)$, where $\Delta$ is defined by $x\Delta y\leftrightarrow\Gamma(x)<\Gamma(y)$ with $\Gamma(y)=\sum\{2^{\Gamma(x)}:x\in y\}$ and $\min_\Delta(\emptyset)=\emptyset$. This $C$ clearly satisfies $C(x)\in x$ whenever $x\neq\emptyset$. On the other hand, Lemma IV.3.17 shows that $\Gamma$ is computable and so is $C$ because $y=\min_\Delta(x)$ is equivalent to $\Gamma(y)=\min\{\Gamma(w):w\in x\}$, where $\min()$ here is the usual minimum function for a set of natural numbers (i.e. $\min(z)=\bigcap z$ with $\min(\emptyset)=\emptyset$). Note that $z=\{\Gamma(w):w\in x\}$ is also computable since it is equivalent to $\forall v\in z\exists w\in x[v=\Gamma(w)]\wedge\forall w\in x[\Gamma(w)\in z]$.
\end{customthm}

\begin{customthm}{IV.3.28}
  Let $S\subset HF$ be a $\Sigma_1$ set which is not $\Delta_1$ and fix a computable surjection $g:\omega\rightarrow S$ as in Corollary IV.3.18. We define a binary relation by $x\sqsubset y$ iff $\forall w\in\bigcup x~\mathrm{nat}(w)\wedge\exists n\in\bigcup x[x=\langle0,n\rangle\wedge y=\langle1,g(n)\rangle\wedge\neg\exists m<n[g(m)=g(n)]]$, where $\mathrm{nat}(w)$ is a $\Delta_0$ formula that says $w$ is a natural number. This $\sqsubset$ is clearly $\Delta_1$ and satisfies $|y\downarrow|\leq1$ for all $y$ since $y\downarrow=\emptyset$ unless $y=\langle1,a\rangle$ for some $a\in S$, in which case $y\downarrow=\{\langle0,\min g^{-1}(a)\rangle\}$. $\downarrow$ is not decidable because $S$ is not $\Delta_1$ and $S=\{a:\langle1,a\rangle\neq\emptyset\}$ holds, where $\langle1,a\rangle\neq\emptyset$ would be $\Delta_1$ if $y\downarrow$ is decidable.
\end{customthm}

\begin{customthm}{IV.3.29}
  Let $S\subset HF$ be a $\Sigma_1$ set which is not $\Delta_1$ and fix a computable surjection $g:\omega\rightarrow S$ as in Corollary IV.3.18. We define a function $h:S\rightarrow\omega$ by $h(x)=\min g^{-1}(x)+1$. $h$ is computable because $n=h(x)$ is equivalent to $1<n\wedge g(n\dotdiv1)=x\wedge\neg\exists m<n[g(m\dotdiv1)=x]$, which has a $\Delta_1$ graph. Now we extend this $h$ to a total function $f$ on $HF$ defined by $f=h\cup(HF\setminus S)\times\{0\}$. This $f$ is $\Pi_1$ because $n=f(x)$ is equivalent to a $\Pi_1$ formula $n=h(x)\vee[x\not\in S\wedge n=0]$, where $h$ is computable and $x\not\in S$ is $\Pi_1$ since $S$ is $\Sigma_1$. However, $f$ is not $\Sigma_1$ because $S=\{x:f(x)\neq0\}$ holds, which is not $\Pi_1$ by assumption but would be $\Pi_1$ if $f$ is $\Sigma_1$.
\end{customthm}

\begin{customthm}{IV.3.30}
  Skipped for now.
\end{customthm}

\begin{customthm}{IV.3.31} (half-done)
  We define a function $F$ recursively by $F(x)=\{F(v):v\in A\cap x\}$, let $T=\ran(F\upharpoonright A)$ and prove the following.
  \begin{enumerate}
    \item\underline{$F\upharpoonright A:A\rightarrow T$ is bijective}. It is obviously surjective to its range. We prove the injectivity by induction on $\in$. Let $X=\{x\in A:\exists y\in A[y\neq x\wedge F(y)=F(x)]\}$ and assume that $X\neq\emptyset$. Let $x$ be an $\in$-least element of $X$ and take $y\in A$ such that $F(y)=F(x)$. Since $F(x)\subseteq F(y)$, for any $u\in A\cap x$, there is $v\in A\cap y$ such that $F(v)=F(u)$. If $v\neq u$, then we have $u\in X$, contradicting the minimality of $x$. Thus $v=u$ holds and we have $A\cap x\subseteq A\cap y$. Likewise, since $F(y)\subseteq F(x)$, for any $v\in A\cap y$, there is $u\in A\cap x$ such that $u=v$, meaning that $A\cap y\subseteq A\cap x$. Thus we have $A\cap x=A\cap y$, which by Extensionality means $x=y$, contradicting our assumption of $x\neq y$. Hence $X=\emptyset$ holds, meaning that $F\upharpoonright A:A\rightarrow T$ is injective.
    \item\underline{$F\upharpoonright A:A\rightarrow T$ is isomorphic}. We have already seen above that $F\upharpoonright A$ is bijective. For $x,y\in A$, if $x\in y$, we have $F(x)\in\{F(v):v\in A\cap y\}=F(y)$. Conversely, if $F(x)\in F(y)$ holds for $x,y\in A$, we have $F(v)=F(x)$ for some $v\in A\cap y$, which by injectivity means that $v=x$ and hence $x\in y$. Thus $x\in y\leftrightarrow F(x)=F(y)$ holds for any $x,y\in A$, meaning that $F\upharpoonright A:A\rightarrow T$ is isomorphic.
    \item\underline{$F(x)=x$ for all $x\in A$ such that $\trcl(x)\subseteq A$}. We prove this by induction. Let $X=\{x\in A:\trcl(x)\subseteq A\wedge F(x)\neq x\}$, assume $X\neq\emptyset$ and take $x\in X$. Then we have $F(x)=\{F(v):v\in A\cap x\}=\{v:v\in A\cap x\}=x$, where $F(v)=v$ holds for $v\in A\cap x$ by the minimality of $x\in X$ since $\trcl(v)\subseteq\trcl(x)\subseteq A$. Note also that $\{v:v\in A\cap x\}=x$ holds by Extensionality. This $F(x)=x$ contradicts $x\in X$, meaning that $X=\emptyset$ and $F(x)=x$ holds for all $x\in A$ such that $\trcl(x)\subseteq A$.
    \item\underline{There is a countable transitive model of $ZFC-P$ in which $\approx$ is not absolute}. (half-done). Let us assume that there exists a suitable cardinal $\kappa$ such that $\omega_1$ is definable in $H(\kappa)$. Note that $H(\kappa)$ is a model of $ZFC-P$. We take a countable transitive $A\preccurlyeq H(\kappa)$ using the Downward Löwenheim-Skolem-Tarski theorem and let $T=\ran(F\upharpoonright A)$ with $F$ defined above. Then $F(\omega)\approx F(\omega_1)$ holds in $V$ as both of them are countable, but $(F(\omega)\approx F(\omega_1))^T$ does not since $(\omega\approx\omega_1)^A$ is false. This shows that $\approx$ is not absolute for $T$.
  \end{enumerate}
\end{customthm}

\begin{customthm}{IV.3.32}
  Skipped for now.
\end{customthm}

\begin{customthm}{IV.3.33} (half-done)
  If $\varphi(x,y)$ is a formula that defines $|x|=|y|$, then $\exists k\in n\,\varphi(k,n\setminus k)$ defines the set $E$ of even natural numbers. Here, the set difference function is $\Delta_0$ since it holds that $z=x\setminus y$ iff $\forall w\in z[w\in x\wedge w\not\in y]\wedge\forall w\in x[w\in y\vee w\in z]$. Nevertheless, I do not come up with any $\Delta_0$-looking formula that defines $E$. Once we have such a formula, we can conclude that $\varphi$ is not $\Delta_0$ since $E$ is not $\Delta_0$ by Lemma II.17.19.
\end{customthm}

\begin{customthm}{IV.4.8}
  Let $E\subset HF$ be $\Sigma_1$ and not $\Delta_1$, and let $E(x)\leftrightarrow\exists y\,A(x,y)$, where $A$ is $\Delta_1$. Then $F(x)=\mu y\, A(x,y)$ is partially computable by Lemma IV.4.6 but cannot be extended to a total computable function because $F=A\cap\overline{F}$ holds for any extension $\overline{F}$ of $F$, where the computability of $\overline{F}$ would make $F$ also computable.
\end{customthm}

\begin{customthm}{IV.4.13} (half-done)
  We treat $X$ as its homeomorphic space $2^\omega$, which is the product space of countably many copies of the discrete space $2=\{0,1\}$. Let $p_n:2^\omega\rightarrow 2$ be the projection onto the $n$-th coordinate.
  \begin{enumerate}
    \item $S=\{p_n^{-1}(0):n\in\omega\}\cup\{p_n^{-1}(1):n\in\omega\}$ is a countable clopen subbase of $X$. Then, the base $B=\{O_1\cap\cdots\cap O_n:n\in\omega\wedge O_1,\ldots O_n\in S\}$ is also countable and clopen. Therefore, any open set of $X$ is a countable union of clopen sets in $X$. Now, let $U,V\subseteq X$ be open sets and let $(U_n)_{n\in\omega}$ and $(V_n)_{n\in\omega}$ be families of clopen sets such that $\bigcup_{n\in\omega}U_n=U$ and $\bigcup_{n\in\omega}V_n=V$. We define $(W_n)_{n\in\omega}$ by $W_{2n}=U_n$ and $W_{2n+1}=V_n$ for each $n\in\omega$ and further define $(K_n)_{n\in\omega}$ by $K_n=W_n\setminus\bigcup_{m<n}W_m$. Since this $(K_n)_{n\in\omega}$ is a family of disjoint open sets such that $K_{2n}\subseteq U$ and $K_{2n+1}\subseteq V$ for each $n\in\omega$, and $\bigcup_{n\in\omega}K_n=U\cup V$, we have $\hat{U}=\bigcup_{n\in\omega}K_{2n}$ and $\hat{V}=\bigcup_{n\in\omega}K_{2n+1}$ that satisfy $\hat{U}\subseteq U$, $\hat{V}\subseteq V$, $\hat{U}\cup\hat{V}=U\cup V$ and $\hat{U}\cap\hat{V}=\emptyset$.
    \item Let $H,K\subseteq X$ be disjoint closed sets. We apply (1) to $U=H^c$ and $V=K^c$ to get $\hat{U},\hat{V}$ such that $\hat{U}\subseteq H^c$, $\hat{V}\subseteq K^c$, $\hat{U}\cup\hat{V}=H^c\cup K^c=X$ and $\hat{U}\cap\hat{V}=\emptyset$. Therefore this $\hat{V}(=\hat{U}^c)$ is a clopen set such that $H\subseteq\hat{V}$ and $\hat{V}\cap K=\emptyset$.
    \item Let $U=\bigcup_{n\in\omega}\left(p_{2n}^{-1}(0)\cap\bigcap_{k\in2n}p_k^{-1}(1)\right)$ and $V=\bigcup_{n\in\omega}\left(p_{2n+1}^{-1}(0)\cap\bigcap_{k\in2n+1}p_k^{-1}(1)\right)$. $U$ (resp. $V$) is the set of countably infinite sequences of $2=\{0,1\}$ in which the first $0$ occurs at an even (resp. odd) index. It is clear that $U,V\subseteq X$, $U\cap V=\emptyset$ and both of $U$ and $V$ are open. Moreover, $\mathrm{Cl}(U)\cap\mathrm{Cl}(V)=\{(1,1,\ldots)\}$ holds. Therefore, if $C\subseteq X$ is a clopen set such that $U\subseteq C$ and $C\cap V=\emptyset$, it holds that $\mathrm{Cl}(U)\subseteq C$ and $\mathrm{Cl}(V)\subseteq C^c$, which leads to a contradiction $(1,1,\ldots)\in C\cap C^c=\emptyset$.
    \item Skipped for now.
  \end{enumerate}
\end{customthm}

\begin{customthm}{IV.4.14}
  We first note that $J$ is well-defined as some numbers have two decimal expansions, where one could conceive that one expansion is computable and the other is not. Since such a number is rational, it is sufficient to show that any rational number is computable. For a given rational number, we consider its repeating decimal expansion $D=d_0d_1\cdots d_{k-1}\overline{d_kd_{k+1}\cdots d_{k+l-1}}$. It holds that $\langle n,d\rangle\in D$ iff $\bigvee_{i<k}[n=i\wedge d=d_i]\vee[n\geq k\wedge[\bigvee_{i<l}[l|(n-i)\wedge d=d_i]]]$, which is $\Delta_1$. Therefore, when we want to prove the computability of a given real number, we may assume that it is irrational.

  Furthermore, we assume that no decimal expansion has an infinite consecutive repetition of $9$. This is possible since the repeating $9$'s of any such expansion can be all replaced by $0$'s, incrementing the digit right before the first $9$ of the infinite repetition.

  Below is the proof for each item. Note that to prove the computability of a given decimal expansion, it suffices to show that it is $\Sigma_1$ as any expansion is a function, meaning that it is also $\Pi_1$ by Lemma IV.3.2. We denote by $\equiv$ the equivalence modulo 10 (i.e. $x\equiv y\leftrightarrow10|(x-y)$).
  \begin{enumerate}
    \item\underline{$J$ is a subfield of $\mathbb{R}$}. We prove the following.
    \begin{enumerate}
      \item\underline{$0\in J$}. Its decimal expansion $(0,0,0,\cdots)=\{\langle0,0\rangle,\langle0,1\rangle,\langle0,2\rangle,\ldots\}\subseteq HF$ is computable since $x\in(0,0,0,\cdots)$ iff $\mathrm{op}(x)\wedge\mathrm{nat}(x_0)\wedge x_1=0$.
      \item\underline{$x,y\in J\rightarrow x+y\in J$}. Let $x,y\in J$. The decimal expansion of $x+y$ can be either the sum or the difference of the two decimal expansions of $x$ and $y$ depending on their signs, but this does not matter when considering the computability due to the following reason. For a given decimal expansion $D=(d_0,d_1,\ldots)$ of a computable real number, the reals that have the complementary decimal expansion $\overline{D}=(10-d_0,10-d_1,\ldots)$ are also computable as it holds that $\langle n,d\rangle\in\overline{D}$ iff $d<10\wedge\langle n,10\dot{-}d\rangle\in D$. Therefore, we may assume that the decimal expansion of $x+y$ is the sum of the two decimal expansions of $x$ and $y$. Let $(x_0,x_1,\ldots)$ and $(y_0,y_1,\ldots)$ be the decimal expansions of $x$ and $y$, respectively. Then it holds that $d$ is the $n$-th digit of $x+y$ iff $d<10\wedge[d\equiv x_n+y_n\vee d\equiv x_n+y_n+1]\wedge[d\equiv x_n+y_n\rightarrow\exists m[n<m\wedge\forall i<m[n<i\rightarrow x_i+y_i=9]\wedge x_m+y_m<9]]$, which is $\Sigma_1$. Hence $x+y\in J$.
      \item\underline{$x\in J\rightarrow-x\in J$}. This is obvious since $x$ and $-x$ have the same decimal expansion.
      \item\underline{$1\in J$}. This is obvious since $1$ has the same decimal expansion as $0\in J$.
      \item\underline{$x,y\in J\rightarrow x\cdot y\in J$}. Let $x,y\in J$. We show $x\cdot y\in J$ by gradually generalizing $y$ as follows.
      \begin{enumerate}
        \item\underline{$y\in10$}. It holds that $\forall n\in10\,\forall w\in J[nw\in J]$ since $0\in J$ and the addition function is closed under $J$ as shown above.
        \item\underline{$y=10^k d$ for some $k\in\mathbb{Z}$ and $d\in10$}. It holds that $\forall k\in\mathbb{Z}\,\forall w\in J[10^k w\in J]$ in general since $10^k w$ is just a shifted one of $w$.
        \item\underline{The decimal expansion of $y$ has finite length}. Decompose $y$ into the digits and reduce it to the previous case.
        \item\underline{$y\in J$}. We may assume $|x|\leq1$ since $x\cdot y=10^k((10^{-k}x)\cdot y)$ holds for $k\in\omega$, where a large enough $k$ makes $|10^{-k}x|\leq1$ true. Let $v=T(i,u)$ say that $v$ is the real number that truncates $u$ at the $i$-th digit, followed by all $0$'s after that. It is obvious that $T$ defines a computable function. Now, if $x\cdot y$ is irrational, it holds that $d$ is the $n$-th digit of $x\cdot y$ iff $d<10\wedge[d\equiv(x\cdot T(n,y))(n)\vee d\equiv(x\cdot T(n,y))(n)+1]\wedge[d\equiv(x\cdot T(n,y))(n)+1\rightarrow\exists m[n<m\wedge d\equiv(x\cdot T(m,y))(n)]]$, which is $\Sigma_1$. Intuitively, this says that $x\cdot T(n,y)$ almost determines the $n$-th digit of $x\cdot y$ and the digit increases by 1 modulo 10 if and only if there are some more digits in $y$ that contribute to the increment. To see that it really works, think of the inequality $|x\cdot T(n,y)-x\cdot y|=|x||T(n,y)-y|\leq|T(n,y)-y|<10^{-n}$ that holds under our assumption of $|x|\leq1$.
      \end{enumerate}
      \item\underline{$x\in J\wedge x\neq0\rightarrow 1/x\in J$}. Let $x\in J\setminus\{0\}$. We may assume $|x|\geq1$ and use the computable function $T$ defined above. Note that for any $n\in\omega$ and $w\in J$, $1/T(n,w)$ is computable as it is a rational number. Now, if $1/x$ is irrational, it holds that $d$ is the $n$-th digit of $1/x$ iff $d<10\wedge[d\equiv(1/T(n,x))(n)\vee d\equiv(1/T(n,x))(n)-1]\wedge[d\equiv(1/T(n,x))(n)-1\rightarrow\exists m[n<m\wedge d\equiv(1/T(m,x))(n)]]$, which is $\Sigma_1$. Intuitively, this says that $1/T(n,x)$ almost determines the $n$-th digit of $x\cdot y$ and the digit decreases by 1 modulo 10 if and only if there are some more digits in $y$ that contribute to the decrement. To see that it really works, think of the following inequality $|1/T(n,x)-1/x|=|T(n,x)-x|/|T(n,x)||x|\leq|T(n,x)-x|<10^{-n}$ that holds under our assumption of $|x|\geq1$.
    \end{enumerate}
    \item\underline{$x\in J\rightarrow\sin(x),e^x\in J$}. Let $x\in J$. We first prove that $\sin(x)\in J$. By coupling adjacent terms of its Taylor series, we have $\sin(x)=\sum_{i=0}^\infty(-1)^ix^{2i+1}/(2i+1)!=\sum_{j=0}^\infty s(j,x)$, where $s(j,x)=(1-x^2/(2j+1)(2j+3))x^{2j+1}/(2j+1)!$. For any $n\in\omega$, the partial sum $S(n,x)=\sum_{j=0}^n s(j,x)$ is computable as it is constructed from computable numbers $n$ and $x$ using the four basic arithmetic operations, which we know from above make the results of the operations also computable. Now, we consider a sufficiently large $k\in\omega$ such that $\forall l>k[1-x^2/(2l+1)(2l+3)>0\wedge|s(l,x)|<10^{-l}]$, which exists since $x^2/(2j+1)(2j+3),10^js(j,x)\rightarrow0$ as $j\rightarrow\infty$. Note that for $l>k$, the sign of $S(l,x)$ is the same as that of $x$ and we have $|S(l,x)-\sin(x)|=|\sum_{j=l+1}^\infty s(j,x)|<\sum_{j=m+1}^\infty|s(j,x)|<\sum_{j=l+1}^\infty10^{-j}=10^{-l}/9<10^{-l}$. Then, if $\sin(x)$ is irrational, it holds that $d$ is the $n$-th digit of $\sin(x)$ iff $d<10\wedge[d\equiv S(\max(k,n),x)(n)\vee d\equiv S(\max(k,n),x)(n)\pm1]\wedge[d\equiv S(\max(k,n),x)(n)\pm1\rightarrow\exists m[m>\max(k,n)\wedge d\equiv S(m,x)(n)]]$, which is $\Sigma_1$.

    $e^x\in J$ can be proved in a similar manner. We think of its "coupled" Taylor expansion $e^x=\sum_{j=0}^\infty(1+x/(2j+1))x^{2j}/(2j)!$, its partial sum $S(n,x)=\sum_{j=0}^n(1+x/(2j+1))x^{2j}/(2j)!$ and a large enough $k$ such that $\forall l>k[1+x/(2l+1)>0\wedge|S(l,x)-e^x|<10^{-l}\wedge S(l,x)>0]$. Then, if $e^x$ is irrational, it holds that $d$ is the $n$-th digit of $e^x$ iff $d<10\wedge[d\equiv S(\max(k,n),x)(n)\vee d\equiv S(\max(k,n),x)(n)+1]\wedge[d\equiv S(\max(k,n),x)(n)+1\rightarrow\exists m[m>\max(k,n)\wedge d\equiv S(m,x)(n)]]$, which is $\Sigma_1$.
   \end{enumerate}
\end{customthm}

\begin{customthm}{IV.4.15}
  Fix $S\subset\omega$ which is $\Sigma_1$ but not $\Delta_1$. Note that such an $S$ can be easily obtained from a subset of $HF$ which is $\Sigma_1$ but not $\Delta_1$ via the computable bijection from $\omega$ to $HF$ defined in Exercise I.14.14. By Corollary IV.3.18, there is a computable $g:\omega\rightarrow HF$ with $S=\{g(n):n\in\omega\}$. Now we define $f:\omega\times\omega\rightarrow\{5,6\}$ by $f(j,n)=5$ iff $j=g(n)$. This is computable since $d=f(j,n)$ iff $\mathrm{nat}(j)\wedge\mathrm{nat}(n)\wedge[[n=g(j)\wedge d=5]\vee[n\neq g(j)\wedge d=6]]$, which is $\Delta_1$. However, the map $d\mapsto d_j^f$ is not $\Delta_1$ as it is the characteristic function of $\{j\in\omega:r_j=0.0666\cdots\}=\omega\setminus S$.
\end{customthm}

\begin{customthm}{IV.4.16}
  We define $\varphi(x,y,s)$ as $s\in HF^{<\omega}\wedge \mathrm{lh(s)}\neq0\wedge s(0)=x\wedge\forall i<\mathrm{lh}(s)\dot{-}1[s(i+1)=\mathrm{sd}(s(i))]\wedge s(\mathrm{lh}(s)\dot{-}1)=y$, which is $\Delta_1$ by Lemma IV.3.22. Then $f(x)=y$ is equivalent to the $\Sigma_1$ formula $\exists s\,\varphi(x,y,s)$, meaning that $f$ is partially computable.
\end{customthm}

\end{document}
